Os materiais utilizados para o experimento foram: goniômetro, lâmpadas de diferentes elementos químicos, lupa e prisma de numeração 17.

para  o início da coleta de dados, o primeiro passo se resumiu a calibrar o goniômetro. Para tal, o prisma foi retirado do goniômetro, a luneta de leitura foi destravada e alinhada com com a entrada de luz da lâmpada de Sódio, travando a luneta. Na sequência, o foco da luneta foi ajustado de para que a cruz no interior da mesma pudesse ser visto de forma nítida, assim como a fenda de entrada de luz; o parafuso de ajuste fino foi empregado para garantir que a cruz se alinhasse com o lado fixo da fenda. Para finalizar o ajuste, o disco graduado do goniômetro foi liberado seu "zero" foi alinhado com o "zero" da luneta.

Para próxima etapa do experimento, foi preciso determinar o ângulo $\alpha$ do ápice do prisma, isto é, a separação angular entre as duas faces do prisma mais próximas à fonte de luz. Para isto, considerou-se os dois raios luz $L_1$ e $L_2$ refletidos pelas faces do prisma, e o resultado geométrico que mostra que $\Delta L = 2\alpha$, como ilustrado abaixo.

\begin{figure}[H]
	\centering	    
	\includegraphics[scale=0.35]{figuras/alpha.png}
	\caption{Diagrama para a obtenção de $\alpha$}
	\label{fig:alpha}
\end{figure}

