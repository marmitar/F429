Para a realização da coleta de dados, o experimento foi dividido em duas partes. sendo a primeira parte dedicada à analise dos filtros RC e segunda do filtro RLC.

\subsection{Filtros RC}

Para a análise de filtros RC como circuitos integradores e diferenciadores, primeiramente montou-se um circuito RC com componentes $R_1=\SI{150}{\ohm}$ e $C_1=\SI{220}{\nano\farad}$, em valores nominais. Para a integração, foi utilizado o circuito \ref{fig:integrador}. Analogamente, para a diferenciação, o circuito \ref{fig:diferenciador} foi empregado. Em sequência, o gerador de onda \texttt{BK Precision 4052} \cite{ref:gerador} foi ajustado para tensão $V_{pp}=\SI{1}{\volt}$ e \textit{offset} de $\SI{0,5}{\volt}$ e os sinais de entrada e saída, $V_{in}$ e $V_{out}$, foram ligados aos osciloscópio \texttt{Tektronix TBS 1062} \cite{ref:osciloscopio}. Com o sistema em questão devidamente montado, valores de tensão foram coletados para frequências de $\SI{50}{\kilo\hertz}$ e $\SI{500}{\hertz}$ em três situações, com ondas senoidais, quadradas e triangulares. Ao final, os dados de tensão $V_{in}$ e $V_{out}$ foram coletados utilizando o \textit{software} \texttt{Pylab}.

\subsection{Filtro RLC}

Para a análise do filtro RLC, foi montado um circuito como pode ser visto em \ref{fig:amortecido}, com componentes de valor nominal $L=\SI{2,42}{\milli\henry}$ e $C_{2}=\SI{47}{\nano\farad}$. Para este circuito, o resistor de década $R_{d}$, uma assumiu valores variados com o decorrer da coleta de dados. Na sequência, o circuito foi conectado com os mesmos gerador de onda e osciloscópio utilizados anteriormente, com o gerador ajustado no modo pulso, com frequência de $\SI{500}{\hertz}$ e tensão de saída $V_{in}=\SI{1}{\volt}$. A fim de obter os regimes oscilatórios mencionados na seção \ref{introducao}, valores de $R_d$ foram variados até que se pudesse observar no osciloscópio, de forma bem definidas os regimes desejados. Para o caso sub-amortecido e sobreamortecido, respectivamente, $\SI{50}{\ohm}$  $\SI{2}{\kilo\ohm}$. É importante mencionar que, de forma que a tensão de saída se estabilizasse ao redor da tensão de entrada, como deseja
Por fim, novamente, as tensões $V_{in}$ e $V_{out}$ foram coletados utilizando o \textit{software} \texttt{Pylab}

%-----------------%
% Filtros Simples %
%-----------------%
\centering

%% parte 1
\begin{subfigure}[t]{0.3\textwidth}
    \centering
    \input{figuras/passa-alta}
    \caption{Passa-altas}
    \label{fig:rc}
    
\end{subfigure}
\qquad
%% parte 2
\begin{subfigure}[t]{0.3\textwidth}
    \centering
    \begin{circuitikz}[scale=1]
  	\draw (0,0)		% linha superior
    node(Gi) {}
    to [short, o-o] ++(3,0)
    to ++(0,0) node(Go) {};
    \draw (2,0)	    % conexão
    to [capacitor, l=$C$] ++(0,2)
    to [inductor, l=$L$] ++(0,2);
    \draw (0, 4)    % linha inferior
    node(Vi) {}
    to [resistor, o-, l=$R$] ++(2,0)
    to [short, -o] ++(1,0)
    to ++(0,0) node(Vo) {};
    \draw			% tensões
    (Gi) to [open,v=$V_{in}$] (Vi)
    (Go) to [open,v=$V_{out}$] (Vo);
\end{circuitikz}
    \caption{Rejeita-banda}
    \label{fig:rlc}
\end{subfigure}

\caption{Circuitaria.}

