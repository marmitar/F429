Para a primeira parte do experimento, foi montado, em uma \textit{protoboard}, um filtro passa-alta na forma de  circuito RC, como representado em \cref{fig:passaalta} . No sistema em questão foi utilizado um capacitor de capacitância nominal \SI{47}{\nano\farad} e um resistor de resistência nominal \SI{470}{\ohm}. Com o circuito devidamente pronto, utilizando o osciloscópio \texttt{Tektronix TBS1000 Series}, juntamente com o gerador de função \texttt{BK Precision 4052}, ajustado para ondas senoidais, foram coletados $100$ pontos igualmente espaçados para um \textit{sweep} de frequência de \SI{10}{\hertz} a \SI{10}{\kilo\hertz} registrando as tensões de saída $V_{out}$ em cada ponto. Em um computador, foi utilizado o \textit{software} \texttt{PyLab}, automatizando as medições. É importante salientar que o \textit{range} de frequências analisado foi escolhido de forma que abrangesse pelo menos uma ordem de grandeza acima e abaixo do que o experimento buscava analisar, permitindo verificar a eficácia do filtro.
\par
Ressalva-se que a escolha dos componentes não foi arbitrária, seguindo o modelo teórico e visando cumprir o objetivo do experimento. A frequência de corte deveria ocorrer em $\omega_0=\frac{1}{RC}$; em que $\omega_0$ representa a frequência de corte, R a resistência nominal do resistor e C a capacitância do capacitor.

\par
Para a segunda parte do experimento, seguindo o método da primeira parte, em uma \textit{protoboard}, foi montado um filtro rejeita-banda na forma de circuito RLC, assim como representado em \ref{fig:rejeitabanda}. Para tal circuito foi utilizado indutor de indutância nominal \SI{48,60}{\milli\henry}, capacitor de \SI{470}{\nano\farad} e resistor de \SI{470}{\ohm}. Em seguida, da mesma forma realizada anteriormente, foram coletados dados de tensão de saída ($V_{out}$) para um \textit{sweep} de frequência de \SI{10}{\hertz} a \SI{10}{\kilo\hertz}. Sendo que o \textit{range} em questão foi empregado de formar a cumprir os objetivos do experimento, tendo a mesma justificativa da seção anterior. Ressalva-se que o fato do \textit{range} de frequências ser o mesmo em ambas as partes é uma coincidência.
\par
Novamente, os componentes do circuito foram selecionados seguindo o modelo teórico e buscando atingir os objetivos do experimento. A frequência central, ou seja, aquela que sofreria maior atenuação pode ser calculada como $\omega_c=\frac{1}{\sqrt{LC}}$; em que $\omega_c$ corresponde à frequência central, L à indutância do indutor e C a capacitância do capacitor.
\par
Em sequência, os dados anteriormente coletados foram utilizados para a construção de diagramas de Bode para os dois filtros. A seguir, na seção \ref{resultados} os mesmos serão abordados com maior profundidade.

%-----------------%
% Filtros Simples %
%-----------------%
\centering

%% parte 1
\begin{subfigure}[t]{0.3\textwidth}
    \centering
    \input{figuras/passa-alta}
    \caption{Passa-altas}
    \label{fig:rc}
    
\end{subfigure}
\qquad
%% parte 2
\begin{subfigure}[t]{0.3\textwidth}
    \centering
    \begin{circuitikz}[scale=1]
  	\draw (0,0)		% linha superior
    node(Gi) {}
    to [short, o-o] ++(3,0)
    to ++(0,0) node(Go) {};
    \draw (2,0)	    % conexão
    to [capacitor, l=$C$] ++(0,2)
    to [inductor, l=$L$] ++(0,2);
    \draw (0, 4)    % linha inferior
    node(Vi) {}
    to [resistor, o-, l=$R$] ++(2,0)
    to [short, -o] ++(1,0)
    to ++(0,0) node(Vo) {};
    \draw			% tensões
    (Gi) to [open,v=$V_{in}$] (Vi)
    (Go) to [open,v=$V_{out}$] (Vo);
\end{circuitikz}
    \caption{Rejeita-banda}
    \label{fig:rlc}
\end{subfigure}

\caption{Circuitaria.}