Os materiais utilizados para o experimento foram: goniômetro, lâmpadas de diferentes elementos químicos, lupa e prisma de numeração 17.

para  o início da coleta de dados, o primeiro passo se resumiu a calibrar o goniômetro. Para tal, o prisma foi retirado do goniômetro, a luneta de leitura foi destravada e alinhada com com a entrada de luz da lâmpada de Sódio, travando a luneta. Na sequência, o foco da luneta foi ajustado de para que a cruz no interior da mesma pudesse ser visto de forma nítida, assim como a fenda de entrada de luz; o parafuso de ajuste fino foi empregado para garantir que a cruz se alinhasse com o lado fixo da fenda, \textbf{que possuia um ajuste de abertura}. Para finalizar o ajuste, o disco graduado do goniômetro foi liberado seu "zero" foi alinhado com o "zero" da luneta.

