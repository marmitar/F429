Ao fim do experimento o grupo determinou que o procedimento experimental empregado foi efetivo para averiguar a validade do modelo de difração de Fraunhofer.

Seguindo o modelo proposto e utilizando os dados coletados no laboratório, os objetivos do experimento puderam ser atingidos com resultados satisfatórios. Tais resultados foram averiguados, pela proximidade das medidas dimensionais das fendas de difração,realizadas utilizando as medidas do anteparo e com microscópio metrológico.

Em suma, o experimento obteve resultados satisfatórios. O mesmo permitiu que os objetivos fossem atingidos, verificando as hipóteses, em especial, da validade do uso de padrões de difração como forma de medir dimensões das fendas que foram utilizadas para gera-los. Além disso, também foi possível entender o procedimento experimental para que o diâmetro de um fio de cabelo pudesse ser obtido da mesma maneira e averiguado pelo microscópio. 