Dentre todos os fenômenos ondulatórios, a difração possui uma posição de destaque. Sendo uma conclusão direta do princípio de Hyugens-Fresnel, em que uma frente de onda funciona como várias novas fontes, a difração da luz em conjunto com a refração foram as grandes afrontas à teoria corpuscular da luz, concedida por Isaac Newton.

Utilizando da difração, Thomas Young derrubou, no início do século XIX, a teoria da luz como partícula. Ao montar um experimento com duas fendas de difração, ele mostrou um exemplo de interferência luminosa, com interferências contrutivas e destrutivas, um fenômeno que não podia ser explicado pela teoria de Newton e provava a natureza ondulatória da luz.

Este experimento traz uma reconstrução de parte do experimento de Young, conhecido por “Experimento da Dupla Fenda”, bem como uma extensão dos resultados para a difração da luz em fendas simples, duplas e múltiplas.

Além da comprovação da natureza ondulatória da luz é possível determinar o comprimento de onda da luz incidente a partir dos padrões de interferência formados pela rede de difração.

Assim, este experimento tem como objetivos: observar os efeitos de difração em fendas simples e os efeitos de interferência em fendas simples e múltiplas; verificar as previsões do modelo de difração de Fraunhofer para fendas simples e sua validade para múltiplas fendas, comparando a medida da largura de uma fenda a partir de padrões de difração com a medida realizada com microscópio metrológico.

