%% estilo do documento e medidas
\documentclass[a4paper,portuguese,utf8x,T1]{article}
\usepackage[top=3cm,bottom=2cm,left=2cm,right=2cm,marginparwidth=1.75cm]{geometry}
\renewcommand{\baselinestretch}{1.2}

%% linguagem e fontes
\usepackage[utf8x]{inputenc}
\usepackage[T1]{fontenc}
\usepackage[portuguese]{babel}

%% pacotes para desenhos e ambientes diferenciadas
\usepackage{amsmath}						% símbolos e ambientes matemáticos
\usepackage{booktabs}						% tabelas melhoradas
\usepackage{datatool}						% recolhimento de dados
\usepackage[table,xcdraw]{xcolor}			% mais cores
\usepackage{graphicx}						% figuras melhoradas
\usepackage{subcaption}						% subfiguras
\usepackage[colorinlistoftodos]{todonotes}	% notas para a escrita
\usepackage[colorlinks=true]{hyperref}		% links
\usepackage{circuitikz}						% desenho de circuitos
\usepackage{csvsimple}						% dados em CSV
\usepackage{cleveref}						% referenciamento melhorado
\usepackage{multirow}						% tabelas de multíplas linhas ou colunas
\usepackage{float}							% fixar posição de figuras

% unidades do SI
\usepackage[
	exponent-to-prefix = true,
	round-mode = figures,
	% round-precision = 3,
	scientific-notation = engineering,
	zero-decimal-to-integer = false,
	separate-uncertainty = true,
	multi-part-units = single
]{siunitx}

% setup para usar o cleveref com o subcaption
\captionsetup[subfigure]{subrefformat=simple,labelformat=simple}
    \renewcommand\thesubfigure{(\alph{subfigure})}

%% dados do relatório
\title{
	\Huge Experimento 1 			\\
    \Large Física Experimental IV
}
\author{
	\small Grupo 3:					\\
	Felipe Mack (81335),			\\
    Gabriel Ferrauche (197314) e	\\
    Tiago de Paula (187679)
}
\date{\small\today}

%%%%%%%%%%%%%%%%%%%%%%%%%%%%%
%% Relatório, propriamente %%
%%%%%%%%%%%%%%%%%%%%%%%%%%%%%
\begin{document}

	%% Cabeçalho %%
	\maketitle

	%%%% Resumo %%%%
	\begin{abstract}
    	\textit{Calculamos a partir da montagem de circuitos RC e RLC a resistência interna do osciloscópio, bem como os valores de capacitância dos componentes utilizados. Analisamos as curvas dos circuitos passa-altas, passa-baixas e passa banda, juntamente com seus respectivos diagramas de bode, comparando-as com curvas esperadas fornecidas na literatura sobre o assunto. Utilizando um método baseado em assíntotas, estimamos valores para as frequências de ressonância e de corte, comparando-os com valores experimentais e teóricos.}

\todo[color=cyan]{Exemplo}
	\end{abstract}

	%%%%% Introdução %%%%%
	\section{Introdução} \label{introducao}
		Dentre todos os fenômenos ondulatórios, a difração possui uma posição de destaque. Através dela, Thomas Young no início do século XIX comprovou a natureza ondulatória da Luz. A partir disso, a teoria corpuscular da luz, concedida por Isaac Newton, deixou de ser considerada. 

Este experimento traz uma reconstrução de parte do experimento de Thomas Young, conhecido por “Experimento da Dupla Fenda”, bem como uma extensão dos resultados para a difração da luz em fendas simples, duplas e múltiplas. 

Além da comprovação da natureza ondulatória da luz é possível determinar o comprimento de onda da luz incidente a partir dos padrões de interferência formados pela rede de difração. 

Assim, este experimento tem como objetivos: observar os efeitos de difração em fendas simples e os efeitos de interferência em fendas simples e múltiplas; verificar as previsões do modelo de difração de Fraunhofer para fendas simples e sua validade para múltiplas fendas, comparando a medida da largura de uma fenda a partir de padrões de difração com a medida realizada com microscópio metrológico; avaliar a capacidade de empregar redes de difração como instrumento para medir comprimentos de onda.



	%%%%% Materiais e Métodos %%%%%
	\section{Materiais e Métodos} \label{metodos}
    	Os materiais utilizados para o experimento foram: goniômetro, lâmpadas de diferentes elementos químicos, lupa e prisma de numeração 17.

para  o início da coleta de dados, o primeiro passo se resumiu a calibrar o goniômetro. Para tal, o prisma foi retirado do goniômetro, a luneta de leitura foi destravada e alinhada com com a entrada de luz da lâmpada de Sódio, travando a luneta. Na sequência, o foco da luneta foi ajustado de para que a cruz no interior da mesma pudesse ser visto de forma nítida, assim como a fenda de entrada de luz; o parafuso de ajuste fino foi empregado para garantir que a cruz se alinhasse com o lado fixo da fenda. Para finalizar o ajuste, o disco graduado do goniômetro foi liberado seu "zero" foi alinhado com o "zero" da luneta.

Para próxima etapa do experimento, foi preciso determinar o ângulo $\alpha$ do ápice do prisma, isto é, a separação angular entre as duas faces do prisma mais próximas à fonte de luz. Para isto, considerou-se os dois raios luz $L_1$ e $L_2$ refletidos pelas faces do prisma, e o resultado geométrico que mostra que $\Delta L = 2\alpha$, como ilustrado abaixo.

\begin{figure}[H]
	\centering	    
	\includegraphics[scale=0.35]{figuras/alpha.png}
	\caption{Diagrama para a obtenção de $\alpha$}
	\label{fig:alpha}
\end{figure}



	%%%%% Resultados %%%%%
	\section{Resultados} \label{resultados}
	 	\todo[inline,color=yellow]{Listar equipamentos (modelo e número de identificação quando possível) e componentes (valores nominais) utilizados.}

\begin{figure}
	\centering
	\includegraphics[width=0.9\textwidth]{figuras/diagrama-bode.png}
	\caption{
    	Diagrama de Bode
	}
    \label{fig:frog}
\end{figure}

\begin{center}
	\begin{tabular}{l|c}
    	\bfseries Frequência (Hz) & \bfseries Transmitância (dB)
     	\csvreader[head to column names]{dados/parte2.csv}{}
      	{\\\hline \freq & \TdB}
	\end{tabular}
\end{center}

	%%%%% Discussão %%%%%
	\section{Discussão} \label{discussao}
    	\subsection{Incertezas}
    Começando pela incerteza das medidas dimensionais do experimento, a trena usada para medir a distância $z$ da fenda ao anteparo (papel milimetrado). Como são medições analógicas, considerou-se uma distribuição triangular, fazendo a incerteza da medição ser:
\begin{equation*}
    u_\text{medição} = \text{resolução} \times \frac{\sqrt{6}}{12}
\end{equation*}

Também assumiu-se uma incerteza de calibração de posicionamento da trena, como a incerteza de medição do ponto \SI{0}{\centi\meter}. Esta incerteza poderia ser determinada de forma idêntica à anterior, dessa forma, $u_\text{calibração} = u_\text{medição}$. A incerteza total então seria:
\begin{align*}
    u_\text{total}
        &= \sqrt{u_\text{calibração}^2 + u_\text{medição}^2} \\
        &= u_\text{medição} \sqrt{2} \\
        &= \text{resolução} \times \frac{\sqrt{3}}{6}
\end{align*}

Já no caso do papel milimetrado do anteparo, esse tipo de incerteza não é aplicável, porém como a medição é dada por dois pontos, teremos duas incertezas $u_\text{medição}$, o que torna a incerteza total igual a da trena. Um ponto a expor sobre essa medida é que ela é encontrada a partir de duas dimensões, o que complica os cálculos da incerteza, porém, esse valor é encontrado pelo \texttt{ImageJ}\cite{ref:imagej}, onde o resultado é bem mais preciso, o que deixa válida a assunção de apenas a incerteza de medição de maneira linear.

Para ambos os casos acima $\text{resolução} = \SI{1}{\milli\meter}$, e, então, $u_\text{total} = \SI{.3}{\milli\meter}$. Dentre as medidas calculadas com isso, existe o $z$, que então fica $\Delta z = \SI{.3}{\milli\meter}$.

Agora, para os valores calculados, foram feitas as devidas propagações de incertezas, tendo como referência o padrão do INMETRO, o GUM\cite{ref:gum}. O primeiro desses valores é o $b$, da equação \ref{eq:difr}, cujas derivadas parciais são:
\begin{gather*}
    \frac{\partial b}{\partial z} = \frac{2 \lambda}{\Delta y} \\
    \frac{\partial b}{\partial \lambda} = \frac{2 z}{\Delta y} \\
    \frac{\partial b}{\partial (\Delta y)} = -\frac{2 \lambda z}{(\Delta y)^2}
\end{gather*}

Porém, como não não existe uma incerteza associada a $\lambda$, tem-se:
\begin{align*}
    \Delta b
        &= \sqrt{\left(\frac{\partial b}{\partial z}\right)^2 (\Delta z)^2 + \left(\frac{\partial b}{\partial (\Delta y)}\right)^2 (\Delta (\Delta y))^2} \\
        &= \frac{2 \lambda}{\Delta y} \sqrt{(\Delta z)^2 + z^2 \left(\frac{\Delta (\Delta y)}{\Delta y}\right)^2}
\end{align*}

Um passo importante para reduzir a incerteza numericamente, que também faz parte do procedimento do experimento, é o cálculo de $n_y$ larguras de máximo, em vez de apenas um. Assim, a medida passa a ser $n_y \Delta y$, com a mesma incerteza do $\Delta y$ original. A nova incerteza passar a ser:
\begin{equation*}
    \Delta (\Delta y) = \frac{\Delta (n_y \Delta y)}{n_y} = \frac{u_\text{total}}{n_y}
\end{equation*}

Com um proccesso bem similar aplicado na separação $h$ das fendas duplas, pela eq. \ref{eq:duplas}, chega-se em:
\begin{equation*}
    \Delta h = \frac{\lambda}{\Lambda} \sqrt{(\Delta z)^2 + z^2 \left(\frac{\Delta \Lambda}{\Lambda}\right)^2}
\end{equation*}

Aplicando o mesmo processo de redução da incerteza, isto é, medindo $n_\Lambda \Lambda$, no lugar de somente $\Lambda$, a incerteza reduz para $\Delta \Lambda = u_\text{total}/n_\Lambda$ também.

Para a separação de múltiplas fendas (eq. \ref{eq:mult}), a propagação é novamente muito similar, resultando em:
\begin{equation*}
    \Delta h = \frac{\lambda}{N \delta y} \sqrt{(\Delta z)^2 + z^2 \left(\frac{\Delta \delta y}{\delta y}\right)^2}
\end{equation*}

Nesse caso, no entanto, a redução de incerteza usada anteriormente não é aplicável.

Para a medida final, com o micrômetro metrológico, voltam as incertezas de medidas experimentais. A incerteza de leitura é parecido com a usada na trena, porém a incerteza de paralaxe, devido a leitura com a lupa, assume uma distribuição retangular, pois pode facilmente mudar a leitura uma casa acima ou abaixo, de acordo com a posição do observador. Assim:
\begin{gather*}
    u_{\text{medição}} = \frac{\text{resolução}}{2 \sqrt{6}} = \text{resolução} \times \frac{\sqrt{6}}{12} \\
    u_{\text{paralaxe}} = \frac{2\ \text{resolução}}{2 \sqrt{3}} = \text{resolução} \times \frac{\sqrt{3}}{3} \\
    \Delta L_i = u_{\text{total}}
        = \sqrt{u_{\text{medição}}^2 + u_{\text{paralaxe}}^2}
        = \text{resolução} \times \frac{\sqrt{6}}{4}
\end{gather*}

Então, como $b = L_2 - L_1$ e $h = L_3 - L_2$:
\begin{equation*}
    \Delta b = \Delta h = \sqrt{(\Delta L_i)^2 + (\Delta L_j)^2} = u_\text{total} \sqrt{2} = \text{resolução} \times \frac{\sqrt{3}}{2}
\end{equation*}

Sendo que a resolução do equipamento é \SI{1}{\micro\meter}, a incerteza é $\Delta b = \Delta h = \SI{.9}{\micro\meter}$.
   

\subsection{Análise de resultados}    


    %%%%% Conclusão %%%%%
    \section{Conclusão} \label{conclusao}
    	Por mais que a resolução espectral possa ser melhorada, assim como as fontes de incerteza como um todo, o experimento foi capaz de relacionar muito bem os dados laboratoriais com o valores teóricos e as referências experimentais. O espectrômetro montado se provou mais do que capaz de ser usado para diferenciar linhas espectrais, no caso até $11$ faixas, o que serve para casos nem tão peocupados com a acurácia do resultado. Porém, o mais importante de tudo é que esse experimento foi capaz de confirmar que esses mesmos materiais são mais do que o necessário para realizar espectometrias ainda mais precisas, dado tempo e esforço o suficiente.

	%%%%%% Referências %%%%%%
	\bibliographystyle{alpha}
	\bibliography{dados/referencias}

\end{document}