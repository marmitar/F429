\begin{table}[H]
	\centering
	\sisetup{
        round-precision = 5,
        arc-separator = \,,
        minimum-integer-digits = 2
	}
	\begin{tabular}{cc}
		\toprule\toprule
            {\bfseries Medida} & {\bfseries Valor}
        \\\midrule
            $\theta_{L_1}$ & $\ang{56;30;00}\pm\ang{;;36}$ \\
            $\theta_{L_2}$ & $\ang{298;20;00}\pm\ang{;;36}$
        \\\midrule
            $\alpha$ & $\ang{59;05;00}\pm\ang{;;25}$
        \\\bottomrule\bottomrule
	\end{tabular}

	\caption{Medidas para o cálculo da abertura do prisma}
	\label{tab:apice}
\end{table}

As primeiras medições feitas após toda a calibração do equipamento foram do ápice do prisma, cujos valores estão detalhados na tabela \ref{tab:apice}. A partir desse valor da abertura $\alpha$ e dos desvios mínimos tabelados em \ref{tab:desvios}, foram encontrados os índices de refração do prisma para cada linha espectral (pela relação \ref{eq:n}), apresentados na mesma tabela.

\DTLloaddb{desvios}{dados/desvio.csv}

\begin{table}[H]
	\centering
	\sisetup{
		table-figures-uncertainty = 1
	}
	\begin{tabular}{cccc}
		\toprule\toprule
            {\bfseries Lâmpada}
				& {\bfseries Comprimento de onda}
                & {\bfseries Desvio mínimo}
                & {\bfseries Índice de refração}

		\DTLforeach*{desvios}{\cp=composto, \lm=lambda,\dv=dm_d,\dm=dm_m,\n=n,\nr=nr}{
			\DTLiffirstrow{\\\midrule}{\\}

			\cp
				& \SI[round-precision = 5]{\lm}{\nano\meter}
				& $\ang[arc-separator = \,, minimum-integer-digits = 2]{\dv;\dm;00} \pm\ang{;;37}$
				& $\num[round-precision = 4]{\n}\pm\nr$
		}
        \\\bottomrule\bottomrule
	\end{tabular}

	\caption{Desvio mínimo medido para cada comprimento de onda}
	\label{tab:desvios}
\end{table}

Com isso, os últimos dados restantes vieram do roteiro do experimento~\cite{ref:roteiro}, que são os comprimentos de onda associados às linhas, permitindo a montagem do gráfico \ref{fig:reta} e os valores recolhidos da regressão (na tab. \ref{tab:regres}).

\begin{figure}[H]
	\centering

	%% Creator: Matplotlib, PGF backend
%%
%% To include the figure in your LaTeX document, write
%%   \input{<filename>.pgf}
%%
%% Make sure the required packages are loaded in your preamble
%%   \usepackage{pgf}
%%
%% Figures using additional raster images can only be included by \input if
%% they are in the same directory as the main LaTeX file. For loading figures
%% from other directories you can use the `import` package
%%   \usepackage{import}
%% and then include the figures with
%%   \import{<path to file>}{<filename>.pgf}
%%
%% Matplotlib used the following preamble
%%   \usepackage{fontspec}
%%   \setmainfont{DejaVuSerif.ttf}[Path=/home/marmis/.virtualenvs/default/lib/python3.7/site-packages/matplotlib/mpl-data/fonts/ttf/]
%%   \setsansfont{arial.ttf}[Path=/usr/share/fonts/TTF/]
%%   \setmonofont{DejaVuSansMono.ttf}[Path=/home/marmis/.virtualenvs/default/lib/python3.7/site-packages/matplotlib/mpl-data/fonts/ttf/]
%%
\begingroup%
\makeatletter%
\begin{pgfpicture}%
\pgfpathrectangle{\pgfpointorigin}{\pgfqpoint{6.400000in}{4.800000in}}%
\pgfusepath{use as bounding box, clip}%
\begin{pgfscope}%
\pgfsetbuttcap%
\pgfsetmiterjoin%
\definecolor{currentfill}{rgb}{1.000000,1.000000,1.000000}%
\pgfsetfillcolor{currentfill}%
\pgfsetlinewidth{0.000000pt}%
\definecolor{currentstroke}{rgb}{1.000000,1.000000,1.000000}%
\pgfsetstrokecolor{currentstroke}%
\pgfsetdash{}{0pt}%
\pgfpathmoveto{\pgfqpoint{0.000000in}{0.000000in}}%
\pgfpathlineto{\pgfqpoint{6.400000in}{0.000000in}}%
\pgfpathlineto{\pgfqpoint{6.400000in}{4.800000in}}%
\pgfpathlineto{\pgfqpoint{0.000000in}{4.800000in}}%
\pgfpathclose%
\pgfusepath{fill}%
\end{pgfscope}%
\begin{pgfscope}%
\pgfsetbuttcap%
\pgfsetmiterjoin%
\definecolor{currentfill}{rgb}{0.917647,0.917647,0.949020}%
\pgfsetfillcolor{currentfill}%
\pgfsetlinewidth{0.000000pt}%
\definecolor{currentstroke}{rgb}{0.000000,0.000000,0.000000}%
\pgfsetstrokecolor{currentstroke}%
\pgfsetstrokeopacity{0.000000}%
\pgfsetdash{}{0pt}%
\pgfpathmoveto{\pgfqpoint{0.800000in}{0.528000in}}%
\pgfpathlineto{\pgfqpoint{5.760000in}{0.528000in}}%
\pgfpathlineto{\pgfqpoint{5.760000in}{4.224000in}}%
\pgfpathlineto{\pgfqpoint{0.800000in}{4.224000in}}%
\pgfpathclose%
\pgfusepath{fill}%
\end{pgfscope}%
\begin{pgfscope}%
\pgfpathrectangle{\pgfqpoint{0.800000in}{0.528000in}}{\pgfqpoint{4.960000in}{3.696000in}}%
\pgfusepath{clip}%
\pgfsetroundcap%
\pgfsetroundjoin%
\pgfsetlinewidth{0.803000pt}%
\definecolor{currentstroke}{rgb}{1.000000,1.000000,1.000000}%
\pgfsetstrokecolor{currentstroke}%
\pgfsetdash{}{0pt}%
\pgfpathmoveto{\pgfqpoint{1.140524in}{0.528000in}}%
\pgfpathlineto{\pgfqpoint{1.140524in}{4.224000in}}%
\pgfusepath{stroke}%
\end{pgfscope}%
\begin{pgfscope}%
\definecolor{textcolor}{rgb}{0.150000,0.150000,0.150000}%
\pgfsetstrokecolor{textcolor}%
\pgfsetfillcolor{textcolor}%
\pgftext[x=1.140524in,y=0.412722in,,top]{\color{textcolor}\rmfamily\fontsize{8.800000}{10.560000}\selectfont 2.0}%
\end{pgfscope}%
\begin{pgfscope}%
\pgfpathrectangle{\pgfqpoint{0.800000in}{0.528000in}}{\pgfqpoint{4.960000in}{3.696000in}}%
\pgfusepath{clip}%
\pgfsetroundcap%
\pgfsetroundjoin%
\pgfsetlinewidth{0.803000pt}%
\definecolor{currentstroke}{rgb}{1.000000,1.000000,1.000000}%
\pgfsetstrokecolor{currentstroke}%
\pgfsetdash{}{0pt}%
\pgfpathmoveto{\pgfqpoint{1.750587in}{0.528000in}}%
\pgfpathlineto{\pgfqpoint{1.750587in}{4.224000in}}%
\pgfusepath{stroke}%
\end{pgfscope}%
\begin{pgfscope}%
\definecolor{textcolor}{rgb}{0.150000,0.150000,0.150000}%
\pgfsetstrokecolor{textcolor}%
\pgfsetfillcolor{textcolor}%
\pgftext[x=1.750587in,y=0.412722in,,top]{\color{textcolor}\rmfamily\fontsize{8.800000}{10.560000}\selectfont 2.5}%
\end{pgfscope}%
\begin{pgfscope}%
\pgfpathrectangle{\pgfqpoint{0.800000in}{0.528000in}}{\pgfqpoint{4.960000in}{3.696000in}}%
\pgfusepath{clip}%
\pgfsetroundcap%
\pgfsetroundjoin%
\pgfsetlinewidth{0.803000pt}%
\definecolor{currentstroke}{rgb}{1.000000,1.000000,1.000000}%
\pgfsetstrokecolor{currentstroke}%
\pgfsetdash{}{0pt}%
\pgfpathmoveto{\pgfqpoint{2.360650in}{0.528000in}}%
\pgfpathlineto{\pgfqpoint{2.360650in}{4.224000in}}%
\pgfusepath{stroke}%
\end{pgfscope}%
\begin{pgfscope}%
\definecolor{textcolor}{rgb}{0.150000,0.150000,0.150000}%
\pgfsetstrokecolor{textcolor}%
\pgfsetfillcolor{textcolor}%
\pgftext[x=2.360650in,y=0.412722in,,top]{\color{textcolor}\rmfamily\fontsize{8.800000}{10.560000}\selectfont 3.0}%
\end{pgfscope}%
\begin{pgfscope}%
\pgfpathrectangle{\pgfqpoint{0.800000in}{0.528000in}}{\pgfqpoint{4.960000in}{3.696000in}}%
\pgfusepath{clip}%
\pgfsetroundcap%
\pgfsetroundjoin%
\pgfsetlinewidth{0.803000pt}%
\definecolor{currentstroke}{rgb}{1.000000,1.000000,1.000000}%
\pgfsetstrokecolor{currentstroke}%
\pgfsetdash{}{0pt}%
\pgfpathmoveto{\pgfqpoint{2.970714in}{0.528000in}}%
\pgfpathlineto{\pgfqpoint{2.970714in}{4.224000in}}%
\pgfusepath{stroke}%
\end{pgfscope}%
\begin{pgfscope}%
\definecolor{textcolor}{rgb}{0.150000,0.150000,0.150000}%
\pgfsetstrokecolor{textcolor}%
\pgfsetfillcolor{textcolor}%
\pgftext[x=2.970714in,y=0.412722in,,top]{\color{textcolor}\rmfamily\fontsize{8.800000}{10.560000}\selectfont 3.5}%
\end{pgfscope}%
\begin{pgfscope}%
\pgfpathrectangle{\pgfqpoint{0.800000in}{0.528000in}}{\pgfqpoint{4.960000in}{3.696000in}}%
\pgfusepath{clip}%
\pgfsetroundcap%
\pgfsetroundjoin%
\pgfsetlinewidth{0.803000pt}%
\definecolor{currentstroke}{rgb}{1.000000,1.000000,1.000000}%
\pgfsetstrokecolor{currentstroke}%
\pgfsetdash{}{0pt}%
\pgfpathmoveto{\pgfqpoint{3.580777in}{0.528000in}}%
\pgfpathlineto{\pgfqpoint{3.580777in}{4.224000in}}%
\pgfusepath{stroke}%
\end{pgfscope}%
\begin{pgfscope}%
\definecolor{textcolor}{rgb}{0.150000,0.150000,0.150000}%
\pgfsetstrokecolor{textcolor}%
\pgfsetfillcolor{textcolor}%
\pgftext[x=3.580777in,y=0.412722in,,top]{\color{textcolor}\rmfamily\fontsize{8.800000}{10.560000}\selectfont 4.0}%
\end{pgfscope}%
\begin{pgfscope}%
\pgfpathrectangle{\pgfqpoint{0.800000in}{0.528000in}}{\pgfqpoint{4.960000in}{3.696000in}}%
\pgfusepath{clip}%
\pgfsetroundcap%
\pgfsetroundjoin%
\pgfsetlinewidth{0.803000pt}%
\definecolor{currentstroke}{rgb}{1.000000,1.000000,1.000000}%
\pgfsetstrokecolor{currentstroke}%
\pgfsetdash{}{0pt}%
\pgfpathmoveto{\pgfqpoint{4.190840in}{0.528000in}}%
\pgfpathlineto{\pgfqpoint{4.190840in}{4.224000in}}%
\pgfusepath{stroke}%
\end{pgfscope}%
\begin{pgfscope}%
\definecolor{textcolor}{rgb}{0.150000,0.150000,0.150000}%
\pgfsetstrokecolor{textcolor}%
\pgfsetfillcolor{textcolor}%
\pgftext[x=4.190840in,y=0.412722in,,top]{\color{textcolor}\rmfamily\fontsize{8.800000}{10.560000}\selectfont 4.5}%
\end{pgfscope}%
\begin{pgfscope}%
\pgfpathrectangle{\pgfqpoint{0.800000in}{0.528000in}}{\pgfqpoint{4.960000in}{3.696000in}}%
\pgfusepath{clip}%
\pgfsetroundcap%
\pgfsetroundjoin%
\pgfsetlinewidth{0.803000pt}%
\definecolor{currentstroke}{rgb}{1.000000,1.000000,1.000000}%
\pgfsetstrokecolor{currentstroke}%
\pgfsetdash{}{0pt}%
\pgfpathmoveto{\pgfqpoint{4.800904in}{0.528000in}}%
\pgfpathlineto{\pgfqpoint{4.800904in}{4.224000in}}%
\pgfusepath{stroke}%
\end{pgfscope}%
\begin{pgfscope}%
\definecolor{textcolor}{rgb}{0.150000,0.150000,0.150000}%
\pgfsetstrokecolor{textcolor}%
\pgfsetfillcolor{textcolor}%
\pgftext[x=4.800904in,y=0.412722in,,top]{\color{textcolor}\rmfamily\fontsize{8.800000}{10.560000}\selectfont 5.0}%
\end{pgfscope}%
\begin{pgfscope}%
\pgfpathrectangle{\pgfqpoint{0.800000in}{0.528000in}}{\pgfqpoint{4.960000in}{3.696000in}}%
\pgfusepath{clip}%
\pgfsetroundcap%
\pgfsetroundjoin%
\pgfsetlinewidth{0.803000pt}%
\definecolor{currentstroke}{rgb}{1.000000,1.000000,1.000000}%
\pgfsetstrokecolor{currentstroke}%
\pgfsetdash{}{0pt}%
\pgfpathmoveto{\pgfqpoint{5.410967in}{0.528000in}}%
\pgfpathlineto{\pgfqpoint{5.410967in}{4.224000in}}%
\pgfusepath{stroke}%
\end{pgfscope}%
\begin{pgfscope}%
\definecolor{textcolor}{rgb}{0.150000,0.150000,0.150000}%
\pgfsetstrokecolor{textcolor}%
\pgfsetfillcolor{textcolor}%
\pgftext[x=5.410967in,y=0.412722in,,top]{\color{textcolor}\rmfamily\fontsize{8.800000}{10.560000}\selectfont 5.5}%
\end{pgfscope}%
\begin{pgfscope}%
\definecolor{textcolor}{rgb}{0.150000,0.150000,0.150000}%
\pgfsetstrokecolor{textcolor}%
\pgfsetfillcolor{textcolor}%
\pgftext[x=3.280000in,y=0.238883in,,top]{\color{textcolor}\rmfamily\fontsize{9.600000}{11.520000}\selectfont il2}%
\end{pgfscope}%
\begin{pgfscope}%
\pgfpathrectangle{\pgfqpoint{0.800000in}{0.528000in}}{\pgfqpoint{4.960000in}{3.696000in}}%
\pgfusepath{clip}%
\pgfsetroundcap%
\pgfsetroundjoin%
\pgfsetlinewidth{0.803000pt}%
\definecolor{currentstroke}{rgb}{1.000000,1.000000,1.000000}%
\pgfsetstrokecolor{currentstroke}%
\pgfsetdash{}{0pt}%
\pgfpathmoveto{\pgfqpoint{0.800000in}{0.947053in}}%
\pgfpathlineto{\pgfqpoint{5.760000in}{0.947053in}}%
\pgfusepath{stroke}%
\end{pgfscope}%
\begin{pgfscope}%
\definecolor{textcolor}{rgb}{0.150000,0.150000,0.150000}%
\pgfsetstrokecolor{textcolor}%
\pgfsetfillcolor{textcolor}%
\pgftext[x=0.334825in,y=0.900622in,left,base]{\color{textcolor}\rmfamily\fontsize{8.800000}{10.560000}\selectfont 1.625}%
\end{pgfscope}%
\begin{pgfscope}%
\pgfpathrectangle{\pgfqpoint{0.800000in}{0.528000in}}{\pgfqpoint{4.960000in}{3.696000in}}%
\pgfusepath{clip}%
\pgfsetroundcap%
\pgfsetroundjoin%
\pgfsetlinewidth{0.803000pt}%
\definecolor{currentstroke}{rgb}{1.000000,1.000000,1.000000}%
\pgfsetstrokecolor{currentstroke}%
\pgfsetdash{}{0pt}%
\pgfpathmoveto{\pgfqpoint{0.800000in}{1.498078in}}%
\pgfpathlineto{\pgfqpoint{5.760000in}{1.498078in}}%
\pgfusepath{stroke}%
\end{pgfscope}%
\begin{pgfscope}%
\definecolor{textcolor}{rgb}{0.150000,0.150000,0.150000}%
\pgfsetstrokecolor{textcolor}%
\pgfsetfillcolor{textcolor}%
\pgftext[x=0.334825in,y=1.451647in,left,base]{\color{textcolor}\rmfamily\fontsize{8.800000}{10.560000}\selectfont 1.630}%
\end{pgfscope}%
\begin{pgfscope}%
\pgfpathrectangle{\pgfqpoint{0.800000in}{0.528000in}}{\pgfqpoint{4.960000in}{3.696000in}}%
\pgfusepath{clip}%
\pgfsetroundcap%
\pgfsetroundjoin%
\pgfsetlinewidth{0.803000pt}%
\definecolor{currentstroke}{rgb}{1.000000,1.000000,1.000000}%
\pgfsetstrokecolor{currentstroke}%
\pgfsetdash{}{0pt}%
\pgfpathmoveto{\pgfqpoint{0.800000in}{2.049102in}}%
\pgfpathlineto{\pgfqpoint{5.760000in}{2.049102in}}%
\pgfusepath{stroke}%
\end{pgfscope}%
\begin{pgfscope}%
\definecolor{textcolor}{rgb}{0.150000,0.150000,0.150000}%
\pgfsetstrokecolor{textcolor}%
\pgfsetfillcolor{textcolor}%
\pgftext[x=0.334825in,y=2.002672in,left,base]{\color{textcolor}\rmfamily\fontsize{8.800000}{10.560000}\selectfont 1.635}%
\end{pgfscope}%
\begin{pgfscope}%
\pgfpathrectangle{\pgfqpoint{0.800000in}{0.528000in}}{\pgfqpoint{4.960000in}{3.696000in}}%
\pgfusepath{clip}%
\pgfsetroundcap%
\pgfsetroundjoin%
\pgfsetlinewidth{0.803000pt}%
\definecolor{currentstroke}{rgb}{1.000000,1.000000,1.000000}%
\pgfsetstrokecolor{currentstroke}%
\pgfsetdash{}{0pt}%
\pgfpathmoveto{\pgfqpoint{0.800000in}{2.600127in}}%
\pgfpathlineto{\pgfqpoint{5.760000in}{2.600127in}}%
\pgfusepath{stroke}%
\end{pgfscope}%
\begin{pgfscope}%
\definecolor{textcolor}{rgb}{0.150000,0.150000,0.150000}%
\pgfsetstrokecolor{textcolor}%
\pgfsetfillcolor{textcolor}%
\pgftext[x=0.334825in,y=2.553697in,left,base]{\color{textcolor}\rmfamily\fontsize{8.800000}{10.560000}\selectfont 1.640}%
\end{pgfscope}%
\begin{pgfscope}%
\pgfpathrectangle{\pgfqpoint{0.800000in}{0.528000in}}{\pgfqpoint{4.960000in}{3.696000in}}%
\pgfusepath{clip}%
\pgfsetroundcap%
\pgfsetroundjoin%
\pgfsetlinewidth{0.803000pt}%
\definecolor{currentstroke}{rgb}{1.000000,1.000000,1.000000}%
\pgfsetstrokecolor{currentstroke}%
\pgfsetdash{}{0pt}%
\pgfpathmoveto{\pgfqpoint{0.800000in}{3.151152in}}%
\pgfpathlineto{\pgfqpoint{5.760000in}{3.151152in}}%
\pgfusepath{stroke}%
\end{pgfscope}%
\begin{pgfscope}%
\definecolor{textcolor}{rgb}{0.150000,0.150000,0.150000}%
\pgfsetstrokecolor{textcolor}%
\pgfsetfillcolor{textcolor}%
\pgftext[x=0.334825in,y=3.104722in,left,base]{\color{textcolor}\rmfamily\fontsize{8.800000}{10.560000}\selectfont 1.645}%
\end{pgfscope}%
\begin{pgfscope}%
\pgfpathrectangle{\pgfqpoint{0.800000in}{0.528000in}}{\pgfqpoint{4.960000in}{3.696000in}}%
\pgfusepath{clip}%
\pgfsetroundcap%
\pgfsetroundjoin%
\pgfsetlinewidth{0.803000pt}%
\definecolor{currentstroke}{rgb}{1.000000,1.000000,1.000000}%
\pgfsetstrokecolor{currentstroke}%
\pgfsetdash{}{0pt}%
\pgfpathmoveto{\pgfqpoint{0.800000in}{3.702177in}}%
\pgfpathlineto{\pgfqpoint{5.760000in}{3.702177in}}%
\pgfusepath{stroke}%
\end{pgfscope}%
\begin{pgfscope}%
\definecolor{textcolor}{rgb}{0.150000,0.150000,0.150000}%
\pgfsetstrokecolor{textcolor}%
\pgfsetfillcolor{textcolor}%
\pgftext[x=0.334825in,y=3.655747in,left,base]{\color{textcolor}\rmfamily\fontsize{8.800000}{10.560000}\selectfont 1.650}%
\end{pgfscope}%
\begin{pgfscope}%
\definecolor{textcolor}{rgb}{0.150000,0.150000,0.150000}%
\pgfsetstrokecolor{textcolor}%
\pgfsetfillcolor{textcolor}%
\pgftext[x=0.279270in,y=2.376000in,,bottom,rotate=90.000000]{\color{textcolor}\rmfamily\fontsize{9.600000}{11.520000}\selectfont n}%
\end{pgfscope}%
\begin{pgfscope}%
\pgfpathrectangle{\pgfqpoint{0.800000in}{0.528000in}}{\pgfqpoint{4.960000in}{3.696000in}}%
\pgfusepath{clip}%
\pgfsetbuttcap%
\pgfsetroundjoin%
\definecolor{currentfill}{rgb}{0.933333,0.521569,0.290196}%
\pgfsetfillcolor{currentfill}%
\pgfsetlinewidth{0.752812pt}%
\definecolor{currentstroke}{rgb}{1.000000,1.000000,1.000000}%
\pgfsetstrokecolor{currentstroke}%
\pgfsetdash{}{0pt}%
\pgfpathmoveto{\pgfqpoint{2.213826in}{1.517243in}}%
\pgfpathcurveto{\pgfqpoint{2.222666in}{1.517243in}}{\pgfqpoint{2.231145in}{1.520755in}}{\pgfqpoint{2.237396in}{1.527006in}}%
\pgfpathcurveto{\pgfqpoint{2.243647in}{1.533257in}}{\pgfqpoint{2.247159in}{1.541736in}}{\pgfqpoint{2.247159in}{1.550576in}}%
\pgfpathcurveto{\pgfqpoint{2.247159in}{1.559416in}}{\pgfqpoint{2.243647in}{1.567895in}}{\pgfqpoint{2.237396in}{1.574146in}}%
\pgfpathcurveto{\pgfqpoint{2.231145in}{1.580397in}}{\pgfqpoint{2.222666in}{1.583909in}}{\pgfqpoint{2.213826in}{1.583909in}}%
\pgfpathcurveto{\pgfqpoint{2.204985in}{1.583909in}}{\pgfqpoint{2.196506in}{1.580397in}}{\pgfqpoint{2.190255in}{1.574146in}}%
\pgfpathcurveto{\pgfqpoint{2.184004in}{1.567895in}}{\pgfqpoint{2.180492in}{1.559416in}}{\pgfqpoint{2.180492in}{1.550576in}}%
\pgfpathcurveto{\pgfqpoint{2.180492in}{1.541736in}}{\pgfqpoint{2.184004in}{1.533257in}}{\pgfqpoint{2.190255in}{1.527006in}}%
\pgfpathcurveto{\pgfqpoint{2.196506in}{1.520755in}}{\pgfqpoint{2.204985in}{1.517243in}}{\pgfqpoint{2.213826in}{1.517243in}}%
\pgfpathclose%
\pgfusepath{stroke,fill}%
\end{pgfscope}%
\begin{pgfscope}%
\pgfpathrectangle{\pgfqpoint{0.800000in}{0.528000in}}{\pgfqpoint{4.960000in}{3.696000in}}%
\pgfusepath{clip}%
\pgfsetbuttcap%
\pgfsetroundjoin%
\definecolor{currentfill}{rgb}{0.933333,0.521569,0.290196}%
\pgfsetfillcolor{currentfill}%
\pgfsetlinewidth{0.752812pt}%
\definecolor{currentstroke}{rgb}{1.000000,1.000000,1.000000}%
\pgfsetstrokecolor{currentstroke}%
\pgfsetdash{}{0pt}%
\pgfpathmoveto{\pgfqpoint{2.471259in}{1.710466in}}%
\pgfpathcurveto{\pgfqpoint{2.480099in}{1.710466in}}{\pgfqpoint{2.488578in}{1.713978in}}{\pgfqpoint{2.494829in}{1.720229in}}%
\pgfpathcurveto{\pgfqpoint{2.501080in}{1.726480in}}{\pgfqpoint{2.504592in}{1.734959in}}{\pgfqpoint{2.504592in}{1.743799in}}%
\pgfpathcurveto{\pgfqpoint{2.504592in}{1.752639in}}{\pgfqpoint{2.501080in}{1.761119in}}{\pgfqpoint{2.494829in}{1.767369in}}%
\pgfpathcurveto{\pgfqpoint{2.488578in}{1.773620in}}{\pgfqpoint{2.480099in}{1.777133in}}{\pgfqpoint{2.471259in}{1.777133in}}%
\pgfpathcurveto{\pgfqpoint{2.462419in}{1.777133in}}{\pgfqpoint{2.453940in}{1.773620in}}{\pgfqpoint{2.447689in}{1.767369in}}%
\pgfpathcurveto{\pgfqpoint{2.441438in}{1.761119in}}{\pgfqpoint{2.437926in}{1.752639in}}{\pgfqpoint{2.437926in}{1.743799in}}%
\pgfpathcurveto{\pgfqpoint{2.437926in}{1.734959in}}{\pgfqpoint{2.441438in}{1.726480in}}{\pgfqpoint{2.447689in}{1.720229in}}%
\pgfpathcurveto{\pgfqpoint{2.453940in}{1.713978in}}{\pgfqpoint{2.462419in}{1.710466in}}{\pgfqpoint{2.471259in}{1.710466in}}%
\pgfpathclose%
\pgfusepath{stroke,fill}%
\end{pgfscope}%
\begin{pgfscope}%
\pgfpathrectangle{\pgfqpoint{0.800000in}{0.528000in}}{\pgfqpoint{4.960000in}{3.696000in}}%
\pgfusepath{clip}%
\pgfsetbuttcap%
\pgfsetroundjoin%
\definecolor{currentfill}{rgb}{0.933333,0.521569,0.290196}%
\pgfsetfillcolor{currentfill}%
\pgfsetlinewidth{0.752812pt}%
\definecolor{currentstroke}{rgb}{1.000000,1.000000,1.000000}%
\pgfsetstrokecolor{currentstroke}%
\pgfsetdash{}{0pt}%
\pgfpathmoveto{\pgfqpoint{4.732019in}{3.242520in}}%
\pgfpathcurveto{\pgfqpoint{4.740860in}{3.242520in}}{\pgfqpoint{4.749339in}{3.246032in}}{\pgfqpoint{4.755590in}{3.252283in}}%
\pgfpathcurveto{\pgfqpoint{4.761841in}{3.258534in}}{\pgfqpoint{4.765353in}{3.267013in}}{\pgfqpoint{4.765353in}{3.275854in}}%
\pgfpathcurveto{\pgfqpoint{4.765353in}{3.284694in}}{\pgfqpoint{4.761841in}{3.293173in}}{\pgfqpoint{4.755590in}{3.299424in}}%
\pgfpathcurveto{\pgfqpoint{4.749339in}{3.305675in}}{\pgfqpoint{4.740860in}{3.309187in}}{\pgfqpoint{4.732019in}{3.309187in}}%
\pgfpathcurveto{\pgfqpoint{4.723179in}{3.309187in}}{\pgfqpoint{4.714700in}{3.305675in}}{\pgfqpoint{4.708449in}{3.299424in}}%
\pgfpathcurveto{\pgfqpoint{4.702198in}{3.293173in}}{\pgfqpoint{4.698686in}{3.284694in}}{\pgfqpoint{4.698686in}{3.275854in}}%
\pgfpathcurveto{\pgfqpoint{4.698686in}{3.267013in}}{\pgfqpoint{4.702198in}{3.258534in}}{\pgfqpoint{4.708449in}{3.252283in}}%
\pgfpathcurveto{\pgfqpoint{4.714700in}{3.246032in}}{\pgfqpoint{4.723179in}{3.242520in}}{\pgfqpoint{4.732019in}{3.242520in}}%
\pgfpathclose%
\pgfusepath{stroke,fill}%
\end{pgfscope}%
\begin{pgfscope}%
\pgfpathrectangle{\pgfqpoint{0.800000in}{0.528000in}}{\pgfqpoint{4.960000in}{3.696000in}}%
\pgfusepath{clip}%
\pgfsetbuttcap%
\pgfsetroundjoin%
\definecolor{currentfill}{rgb}{0.415686,0.800000,0.392157}%
\pgfsetfillcolor{currentfill}%
\pgfsetlinewidth{0.752812pt}%
\definecolor{currentstroke}{rgb}{1.000000,1.000000,1.000000}%
\pgfsetstrokecolor{currentstroke}%
\pgfsetdash{}{0pt}%
\pgfpathmoveto{\pgfqpoint{2.792013in}{2.095771in}}%
\pgfpathcurveto{\pgfqpoint{2.800853in}{2.095771in}}{\pgfqpoint{2.809332in}{2.099283in}}{\pgfqpoint{2.815583in}{2.105534in}}%
\pgfpathcurveto{\pgfqpoint{2.821834in}{2.111785in}}{\pgfqpoint{2.825346in}{2.120264in}}{\pgfqpoint{2.825346in}{2.129104in}}%
\pgfpathcurveto{\pgfqpoint{2.825346in}{2.137944in}}{\pgfqpoint{2.821834in}{2.146423in}}{\pgfqpoint{2.815583in}{2.152674in}}%
\pgfpathcurveto{\pgfqpoint{2.809332in}{2.158925in}}{\pgfqpoint{2.800853in}{2.162437in}}{\pgfqpoint{2.792013in}{2.162437in}}%
\pgfpathcurveto{\pgfqpoint{2.783173in}{2.162437in}}{\pgfqpoint{2.774693in}{2.158925in}}{\pgfqpoint{2.768443in}{2.152674in}}%
\pgfpathcurveto{\pgfqpoint{2.762192in}{2.146423in}}{\pgfqpoint{2.758679in}{2.137944in}}{\pgfqpoint{2.758679in}{2.129104in}}%
\pgfpathcurveto{\pgfqpoint{2.758679in}{2.120264in}}{\pgfqpoint{2.762192in}{2.111785in}}{\pgfqpoint{2.768443in}{2.105534in}}%
\pgfpathcurveto{\pgfqpoint{2.774693in}{2.099283in}}{\pgfqpoint{2.783173in}{2.095771in}}{\pgfqpoint{2.792013in}{2.095771in}}%
\pgfpathclose%
\pgfusepath{stroke,fill}%
\end{pgfscope}%
\begin{pgfscope}%
\pgfpathrectangle{\pgfqpoint{0.800000in}{0.528000in}}{\pgfqpoint{4.960000in}{3.696000in}}%
\pgfusepath{clip}%
\pgfsetbuttcap%
\pgfsetroundjoin%
\definecolor{currentfill}{rgb}{0.415686,0.800000,0.392157}%
\pgfsetfillcolor{currentfill}%
\pgfsetlinewidth{0.752812pt}%
\definecolor{currentstroke}{rgb}{1.000000,1.000000,1.000000}%
\pgfsetstrokecolor{currentstroke}%
\pgfsetdash{}{0pt}%
\pgfpathmoveto{\pgfqpoint{2.352299in}{1.903309in}}%
\pgfpathcurveto{\pgfqpoint{2.361139in}{1.903309in}}{\pgfqpoint{2.369619in}{1.906821in}}{\pgfqpoint{2.375870in}{1.913072in}}%
\pgfpathcurveto{\pgfqpoint{2.382120in}{1.919323in}}{\pgfqpoint{2.385633in}{1.927802in}}{\pgfqpoint{2.385633in}{1.936642in}}%
\pgfpathcurveto{\pgfqpoint{2.385633in}{1.945482in}}{\pgfqpoint{2.382120in}{1.953961in}}{\pgfqpoint{2.375870in}{1.960212in}}%
\pgfpathcurveto{\pgfqpoint{2.369619in}{1.966463in}}{\pgfqpoint{2.361139in}{1.969975in}}{\pgfqpoint{2.352299in}{1.969975in}}%
\pgfpathcurveto{\pgfqpoint{2.343459in}{1.969975in}}{\pgfqpoint{2.334980in}{1.966463in}}{\pgfqpoint{2.328729in}{1.960212in}}%
\pgfpathcurveto{\pgfqpoint{2.322478in}{1.953961in}}{\pgfqpoint{2.318966in}{1.945482in}}{\pgfqpoint{2.318966in}{1.936642in}}%
\pgfpathcurveto{\pgfqpoint{2.318966in}{1.927802in}}{\pgfqpoint{2.322478in}{1.919323in}}{\pgfqpoint{2.328729in}{1.913072in}}%
\pgfpathcurveto{\pgfqpoint{2.334980in}{1.906821in}}{\pgfqpoint{2.343459in}{1.903309in}}{\pgfqpoint{2.352299in}{1.903309in}}%
\pgfpathclose%
\pgfusepath{stroke,fill}%
\end{pgfscope}%
\begin{pgfscope}%
\pgfpathrectangle{\pgfqpoint{0.800000in}{0.528000in}}{\pgfqpoint{4.960000in}{3.696000in}}%
\pgfusepath{clip}%
\pgfsetbuttcap%
\pgfsetroundjoin%
\definecolor{currentfill}{rgb}{0.415686,0.800000,0.392157}%
\pgfsetfillcolor{currentfill}%
\pgfsetlinewidth{0.752812pt}%
\definecolor{currentstroke}{rgb}{1.000000,1.000000,1.000000}%
\pgfsetstrokecolor{currentstroke}%
\pgfsetdash{}{0pt}%
\pgfpathmoveto{\pgfqpoint{5.123755in}{3.810720in}}%
\pgfpathcurveto{\pgfqpoint{5.132595in}{3.810720in}}{\pgfqpoint{5.141074in}{3.814232in}}{\pgfqpoint{5.147325in}{3.820483in}}%
\pgfpathcurveto{\pgfqpoint{5.153576in}{3.826734in}}{\pgfqpoint{5.157088in}{3.835213in}}{\pgfqpoint{5.157088in}{3.844053in}}%
\pgfpathcurveto{\pgfqpoint{5.157088in}{3.852893in}}{\pgfqpoint{5.153576in}{3.861372in}}{\pgfqpoint{5.147325in}{3.867623in}}%
\pgfpathcurveto{\pgfqpoint{5.141074in}{3.873874in}}{\pgfqpoint{5.132595in}{3.877386in}}{\pgfqpoint{5.123755in}{3.877386in}}%
\pgfpathcurveto{\pgfqpoint{5.114915in}{3.877386in}}{\pgfqpoint{5.106436in}{3.873874in}}{\pgfqpoint{5.100185in}{3.867623in}}%
\pgfpathcurveto{\pgfqpoint{5.093934in}{3.861372in}}{\pgfqpoint{5.090422in}{3.852893in}}{\pgfqpoint{5.090422in}{3.844053in}}%
\pgfpathcurveto{\pgfqpoint{5.090422in}{3.835213in}}{\pgfqpoint{5.093934in}{3.826734in}}{\pgfqpoint{5.100185in}{3.820483in}}%
\pgfpathcurveto{\pgfqpoint{5.106436in}{3.814232in}}{\pgfqpoint{5.114915in}{3.810720in}}{\pgfqpoint{5.123755in}{3.810720in}}%
\pgfpathclose%
\pgfusepath{stroke,fill}%
\end{pgfscope}%
\begin{pgfscope}%
\pgfpathrectangle{\pgfqpoint{0.800000in}{0.528000in}}{\pgfqpoint{4.960000in}{3.696000in}}%
\pgfusepath{clip}%
\pgfsetbuttcap%
\pgfsetroundjoin%
\definecolor{currentfill}{rgb}{0.839216,0.372549,0.372549}%
\pgfsetfillcolor{currentfill}%
\pgfsetlinewidth{0.752812pt}%
\definecolor{currentstroke}{rgb}{1.000000,1.000000,1.000000}%
\pgfsetstrokecolor{currentstroke}%
\pgfsetdash{}{0pt}%
\pgfpathmoveto{\pgfqpoint{2.234546in}{1.710466in}}%
\pgfpathcurveto{\pgfqpoint{2.243387in}{1.710466in}}{\pgfqpoint{2.251866in}{1.713978in}}{\pgfqpoint{2.258117in}{1.720229in}}%
\pgfpathcurveto{\pgfqpoint{2.264368in}{1.726480in}}{\pgfqpoint{2.267880in}{1.734959in}}{\pgfqpoint{2.267880in}{1.743799in}}%
\pgfpathcurveto{\pgfqpoint{2.267880in}{1.752639in}}{\pgfqpoint{2.264368in}{1.761119in}}{\pgfqpoint{2.258117in}{1.767369in}}%
\pgfpathcurveto{\pgfqpoint{2.251866in}{1.773620in}}{\pgfqpoint{2.243387in}{1.777133in}}{\pgfqpoint{2.234546in}{1.777133in}}%
\pgfpathcurveto{\pgfqpoint{2.225706in}{1.777133in}}{\pgfqpoint{2.217227in}{1.773620in}}{\pgfqpoint{2.210976in}{1.767369in}}%
\pgfpathcurveto{\pgfqpoint{2.204725in}{1.761119in}}{\pgfqpoint{2.201213in}{1.752639in}}{\pgfqpoint{2.201213in}{1.743799in}}%
\pgfpathcurveto{\pgfqpoint{2.201213in}{1.734959in}}{\pgfqpoint{2.204725in}{1.726480in}}{\pgfqpoint{2.210976in}{1.720229in}}%
\pgfpathcurveto{\pgfqpoint{2.217227in}{1.713978in}}{\pgfqpoint{2.225706in}{1.710466in}}{\pgfqpoint{2.234546in}{1.710466in}}%
\pgfpathclose%
\pgfusepath{stroke,fill}%
\end{pgfscope}%
\begin{pgfscope}%
\pgfpathrectangle{\pgfqpoint{0.800000in}{0.528000in}}{\pgfqpoint{4.960000in}{3.696000in}}%
\pgfusepath{clip}%
\pgfsetbuttcap%
\pgfsetroundjoin%
\definecolor{currentfill}{rgb}{0.839216,0.372549,0.372549}%
\pgfsetfillcolor{currentfill}%
\pgfsetlinewidth{0.752812pt}%
\definecolor{currentstroke}{rgb}{1.000000,1.000000,1.000000}%
\pgfsetstrokecolor{currentstroke}%
\pgfsetdash{}{0pt}%
\pgfpathmoveto{\pgfqpoint{3.549691in}{2.575256in}}%
\pgfpathcurveto{\pgfqpoint{3.558531in}{2.575256in}}{\pgfqpoint{3.567010in}{2.578769in}}{\pgfqpoint{3.573261in}{2.585019in}}%
\pgfpathcurveto{\pgfqpoint{3.579512in}{2.591270in}}{\pgfqpoint{3.583024in}{2.599750in}}{\pgfqpoint{3.583024in}{2.608590in}}%
\pgfpathcurveto{\pgfqpoint{3.583024in}{2.617430in}}{\pgfqpoint{3.579512in}{2.625909in}}{\pgfqpoint{3.573261in}{2.632160in}}%
\pgfpathcurveto{\pgfqpoint{3.567010in}{2.638411in}}{\pgfqpoint{3.558531in}{2.641923in}}{\pgfqpoint{3.549691in}{2.641923in}}%
\pgfpathcurveto{\pgfqpoint{3.540851in}{2.641923in}}{\pgfqpoint{3.532372in}{2.638411in}}{\pgfqpoint{3.526121in}{2.632160in}}%
\pgfpathcurveto{\pgfqpoint{3.519870in}{2.625909in}}{\pgfqpoint{3.516358in}{2.617430in}}{\pgfqpoint{3.516358in}{2.608590in}}%
\pgfpathcurveto{\pgfqpoint{3.516358in}{2.599750in}}{\pgfqpoint{3.519870in}{2.591270in}}{\pgfqpoint{3.526121in}{2.585019in}}%
\pgfpathcurveto{\pgfqpoint{3.532372in}{2.578769in}}{\pgfqpoint{3.540851in}{2.575256in}}{\pgfqpoint{3.549691in}{2.575256in}}%
\pgfpathclose%
\pgfusepath{stroke,fill}%
\end{pgfscope}%
\begin{pgfscope}%
\pgfpathrectangle{\pgfqpoint{0.800000in}{0.528000in}}{\pgfqpoint{4.960000in}{3.696000in}}%
\pgfusepath{clip}%
\pgfsetbuttcap%
\pgfsetroundjoin%
\definecolor{currentfill}{rgb}{0.839216,0.372549,0.372549}%
\pgfsetfillcolor{currentfill}%
\pgfsetlinewidth{0.752812pt}%
\definecolor{currentstroke}{rgb}{1.000000,1.000000,1.000000}%
\pgfsetstrokecolor{currentstroke}%
\pgfsetdash{}{0pt}%
\pgfpathmoveto{\pgfqpoint{4.804004in}{3.432304in}}%
\pgfpathcurveto{\pgfqpoint{4.812844in}{3.432304in}}{\pgfqpoint{4.821323in}{3.435817in}}{\pgfqpoint{4.827574in}{3.442067in}}%
\pgfpathcurveto{\pgfqpoint{4.833825in}{3.448318in}}{\pgfqpoint{4.837337in}{3.456798in}}{\pgfqpoint{4.837337in}{3.465638in}}%
\pgfpathcurveto{\pgfqpoint{4.837337in}{3.474478in}}{\pgfqpoint{4.833825in}{3.482957in}}{\pgfqpoint{4.827574in}{3.489208in}}%
\pgfpathcurveto{\pgfqpoint{4.821323in}{3.495459in}}{\pgfqpoint{4.812844in}{3.498971in}}{\pgfqpoint{4.804004in}{3.498971in}}%
\pgfpathcurveto{\pgfqpoint{4.795164in}{3.498971in}}{\pgfqpoint{4.786685in}{3.495459in}}{\pgfqpoint{4.780434in}{3.489208in}}%
\pgfpathcurveto{\pgfqpoint{4.774183in}{3.482957in}}{\pgfqpoint{4.770671in}{3.474478in}}{\pgfqpoint{4.770671in}{3.465638in}}%
\pgfpathcurveto{\pgfqpoint{4.770671in}{3.456798in}}{\pgfqpoint{4.774183in}{3.448318in}}{\pgfqpoint{4.780434in}{3.442067in}}%
\pgfpathcurveto{\pgfqpoint{4.786685in}{3.435817in}}{\pgfqpoint{4.795164in}{3.432304in}}{\pgfqpoint{4.804004in}{3.432304in}}%
\pgfpathclose%
\pgfusepath{stroke,fill}%
\end{pgfscope}%
\begin{pgfscope}%
\pgfpathrectangle{\pgfqpoint{0.800000in}{0.528000in}}{\pgfqpoint{4.960000in}{3.696000in}}%
\pgfusepath{clip}%
\pgfsetbuttcap%
\pgfsetroundjoin%
\definecolor{currentfill}{rgb}{0.839216,0.372549,0.372549}%
\pgfsetfillcolor{currentfill}%
\pgfsetlinewidth{0.752812pt}%
\definecolor{currentstroke}{rgb}{1.000000,1.000000,1.000000}%
\pgfsetstrokecolor{currentstroke}%
\pgfsetdash{}{0pt}%
\pgfpathmoveto{\pgfqpoint{1.436245in}{0.935293in}}%
\pgfpathcurveto{\pgfqpoint{1.445085in}{0.935293in}}{\pgfqpoint{1.453564in}{0.938806in}}{\pgfqpoint{1.459815in}{0.945056in}}%
\pgfpathcurveto{\pgfqpoint{1.466066in}{0.951307in}}{\pgfqpoint{1.469578in}{0.959787in}}{\pgfqpoint{1.469578in}{0.968627in}}%
\pgfpathcurveto{\pgfqpoint{1.469578in}{0.977467in}}{\pgfqpoint{1.466066in}{0.985946in}}{\pgfqpoint{1.459815in}{0.992197in}}%
\pgfpathcurveto{\pgfqpoint{1.453564in}{0.998448in}}{\pgfqpoint{1.445085in}{1.001960in}}{\pgfqpoint{1.436245in}{1.001960in}}%
\pgfpathcurveto{\pgfqpoint{1.427405in}{1.001960in}}{\pgfqpoint{1.418926in}{0.998448in}}{\pgfqpoint{1.412675in}{0.992197in}}%
\pgfpathcurveto{\pgfqpoint{1.406424in}{0.985946in}}{\pgfqpoint{1.402912in}{0.977467in}}{\pgfqpoint{1.402912in}{0.968627in}}%
\pgfpathcurveto{\pgfqpoint{1.402912in}{0.959787in}}{\pgfqpoint{1.406424in}{0.951307in}}{\pgfqpoint{1.412675in}{0.945056in}}%
\pgfpathcurveto{\pgfqpoint{1.418926in}{0.938806in}}{\pgfqpoint{1.427405in}{0.935293in}}{\pgfqpoint{1.436245in}{0.935293in}}%
\pgfpathclose%
\pgfusepath{stroke,fill}%
\end{pgfscope}%
\begin{pgfscope}%
\pgfpathrectangle{\pgfqpoint{0.800000in}{0.528000in}}{\pgfqpoint{4.960000in}{3.696000in}}%
\pgfusepath{clip}%
\pgfsetbuttcap%
\pgfsetroundjoin%
\definecolor{currentfill}{rgb}{0.282353,0.470588,0.815686}%
\pgfsetfillcolor{currentfill}%
\pgfsetlinewidth{0.752812pt}%
\definecolor{currentstroke}{rgb}{1.000000,1.000000,1.000000}%
\pgfsetstrokecolor{currentstroke}%
\pgfsetdash{}{0pt}%
\pgfpathmoveto{\pgfqpoint{3.417493in}{2.575256in}}%
\pgfpathcurveto{\pgfqpoint{3.426333in}{2.575256in}}{\pgfqpoint{3.434812in}{2.578769in}}{\pgfqpoint{3.441063in}{2.585019in}}%
\pgfpathcurveto{\pgfqpoint{3.447314in}{2.591270in}}{\pgfqpoint{3.450826in}{2.599750in}}{\pgfqpoint{3.450826in}{2.608590in}}%
\pgfpathcurveto{\pgfqpoint{3.450826in}{2.617430in}}{\pgfqpoint{3.447314in}{2.625909in}}{\pgfqpoint{3.441063in}{2.632160in}}%
\pgfpathcurveto{\pgfqpoint{3.434812in}{2.638411in}}{\pgfqpoint{3.426333in}{2.641923in}}{\pgfqpoint{3.417493in}{2.641923in}}%
\pgfpathcurveto{\pgfqpoint{3.408653in}{2.641923in}}{\pgfqpoint{3.400174in}{2.638411in}}{\pgfqpoint{3.393923in}{2.632160in}}%
\pgfpathcurveto{\pgfqpoint{3.387672in}{2.625909in}}{\pgfqpoint{3.384160in}{2.617430in}}{\pgfqpoint{3.384160in}{2.608590in}}%
\pgfpathcurveto{\pgfqpoint{3.384160in}{2.599750in}}{\pgfqpoint{3.387672in}{2.591270in}}{\pgfqpoint{3.393923in}{2.585019in}}%
\pgfpathcurveto{\pgfqpoint{3.400174in}{2.578769in}}{\pgfqpoint{3.408653in}{2.575256in}}{\pgfqpoint{3.417493in}{2.575256in}}%
\pgfpathclose%
\pgfusepath{stroke,fill}%
\end{pgfscope}%
\begin{pgfscope}%
\pgfpathrectangle{\pgfqpoint{0.800000in}{0.528000in}}{\pgfqpoint{4.960000in}{3.696000in}}%
\pgfusepath{clip}%
\pgfsetbuttcap%
\pgfsetroundjoin%
\definecolor{currentfill}{rgb}{0.282353,0.470588,0.815686}%
\pgfsetfillcolor{currentfill}%
\pgfsetlinewidth{0.752812pt}%
\definecolor{currentstroke}{rgb}{1.000000,1.000000,1.000000}%
\pgfsetstrokecolor{currentstroke}%
\pgfsetdash{}{0pt}%
\pgfpathmoveto{\pgfqpoint{1.643577in}{1.129656in}}%
\pgfpathcurveto{\pgfqpoint{1.652417in}{1.129656in}}{\pgfqpoint{1.660896in}{1.133168in}}{\pgfqpoint{1.667147in}{1.139419in}}%
\pgfpathcurveto{\pgfqpoint{1.673398in}{1.145670in}}{\pgfqpoint{1.676910in}{1.154149in}}{\pgfqpoint{1.676910in}{1.162989in}}%
\pgfpathcurveto{\pgfqpoint{1.676910in}{1.171829in}}{\pgfqpoint{1.673398in}{1.180308in}}{\pgfqpoint{1.667147in}{1.186559in}}%
\pgfpathcurveto{\pgfqpoint{1.660896in}{1.192810in}}{\pgfqpoint{1.652417in}{1.196322in}}{\pgfqpoint{1.643577in}{1.196322in}}%
\pgfpathcurveto{\pgfqpoint{1.634737in}{1.196322in}}{\pgfqpoint{1.626258in}{1.192810in}}{\pgfqpoint{1.620007in}{1.186559in}}%
\pgfpathcurveto{\pgfqpoint{1.613756in}{1.180308in}}{\pgfqpoint{1.610244in}{1.171829in}}{\pgfqpoint{1.610244in}{1.162989in}}%
\pgfpathcurveto{\pgfqpoint{1.610244in}{1.154149in}}{\pgfqpoint{1.613756in}{1.145670in}}{\pgfqpoint{1.620007in}{1.139419in}}%
\pgfpathcurveto{\pgfqpoint{1.626258in}{1.133168in}}{\pgfqpoint{1.634737in}{1.129656in}}{\pgfqpoint{1.643577in}{1.129656in}}%
\pgfpathclose%
\pgfusepath{stroke,fill}%
\end{pgfscope}%
\begin{pgfscope}%
\pgfpathrectangle{\pgfqpoint{0.800000in}{0.528000in}}{\pgfqpoint{4.960000in}{3.696000in}}%
\pgfusepath{clip}%
\pgfsetbuttcap%
\pgfsetroundjoin%
\definecolor{currentfill}{rgb}{0.282353,0.470588,0.815686}%
\pgfsetfillcolor{currentfill}%
\pgfsetlinewidth{0.752812pt}%
\definecolor{currentstroke}{rgb}{1.000000,1.000000,1.000000}%
\pgfsetstrokecolor{currentstroke}%
\pgfsetdash{}{0pt}%
\pgfpathmoveto{\pgfqpoint{3.996180in}{2.861801in}}%
\pgfpathcurveto{\pgfqpoint{4.005020in}{2.861801in}}{\pgfqpoint{4.013499in}{2.865313in}}{\pgfqpoint{4.019750in}{2.871564in}}%
\pgfpathcurveto{\pgfqpoint{4.026001in}{2.877815in}}{\pgfqpoint{4.029513in}{2.886294in}}{\pgfqpoint{4.029513in}{2.895134in}}%
\pgfpathcurveto{\pgfqpoint{4.029513in}{2.903975in}}{\pgfqpoint{4.026001in}{2.912454in}}{\pgfqpoint{4.019750in}{2.918705in}}%
\pgfpathcurveto{\pgfqpoint{4.013499in}{2.924956in}}{\pgfqpoint{4.005020in}{2.928468in}}{\pgfqpoint{3.996180in}{2.928468in}}%
\pgfpathcurveto{\pgfqpoint{3.987340in}{2.928468in}}{\pgfqpoint{3.978860in}{2.924956in}}{\pgfqpoint{3.972609in}{2.918705in}}%
\pgfpathcurveto{\pgfqpoint{3.966359in}{2.912454in}}{\pgfqpoint{3.962846in}{2.903975in}}{\pgfqpoint{3.962846in}{2.895134in}}%
\pgfpathcurveto{\pgfqpoint{3.962846in}{2.886294in}}{\pgfqpoint{3.966359in}{2.877815in}}{\pgfqpoint{3.972609in}{2.871564in}}%
\pgfpathcurveto{\pgfqpoint{3.978860in}{2.865313in}}{\pgfqpoint{3.987340in}{2.861801in}}{\pgfqpoint{3.996180in}{2.861801in}}%
\pgfpathclose%
\pgfusepath{stroke,fill}%
\end{pgfscope}%
\begin{pgfscope}%
\pgfpathrectangle{\pgfqpoint{0.800000in}{0.528000in}}{\pgfqpoint{4.960000in}{3.696000in}}%
\pgfusepath{clip}%
\pgfsetbuttcap%
\pgfsetroundjoin%
\definecolor{currentfill}{rgb}{1.000000,1.000000,1.000000}%
\pgfsetfillcolor{currentfill}%
\pgfsetlinewidth{1.204500pt}%
\definecolor{currentstroke}{rgb}{1.000000,1.000000,1.000000}%
\pgfsetstrokecolor{currentstroke}%
\pgfsetdash{}{0pt}%
\pgfsys@defobject{currentmarker}{\pgfqpoint{infin}{infin}}{\pgfqpoint{-infin}{-infin}}{%
\pgfusepath{stroke,fill}%
}%
\end{pgfscope}%
\begin{pgfscope}%
\pgfpathrectangle{\pgfqpoint{0.800000in}{0.528000in}}{\pgfqpoint{4.960000in}{3.696000in}}%
\pgfusepath{clip}%
\pgfsetbuttcap%
\pgfsetroundjoin%
\definecolor{currentfill}{rgb}{0.282353,0.470588,0.815686}%
\pgfsetfillcolor{currentfill}%
\pgfsetlinewidth{0.803000pt}%
\definecolor{currentstroke}{rgb}{0.282353,0.470588,0.815686}%
\pgfsetstrokecolor{currentstroke}%
\pgfsetdash{}{0pt}%
\pgfpathmoveto{\pgfqpoint{0.000000in}{-0.033333in}}%
\pgfpathcurveto{\pgfqpoint{0.008840in}{-0.033333in}}{\pgfqpoint{0.017319in}{-0.029821in}}{\pgfqpoint{0.023570in}{-0.023570in}}%
\pgfpathcurveto{\pgfqpoint{0.029821in}{-0.017319in}}{\pgfqpoint{0.033333in}{-0.008840in}}{\pgfqpoint{0.033333in}{0.000000in}}%
\pgfpathcurveto{\pgfqpoint{0.033333in}{0.008840in}}{\pgfqpoint{0.029821in}{0.017319in}}{\pgfqpoint{0.023570in}{0.023570in}}%
\pgfpathcurveto{\pgfqpoint{0.017319in}{0.029821in}}{\pgfqpoint{0.008840in}{0.033333in}}{\pgfqpoint{0.000000in}{0.033333in}}%
\pgfpathcurveto{\pgfqpoint{-0.008840in}{0.033333in}}{\pgfqpoint{-0.017319in}{0.029821in}}{\pgfqpoint{-0.023570in}{0.023570in}}%
\pgfpathcurveto{\pgfqpoint{-0.029821in}{0.017319in}}{\pgfqpoint{-0.033333in}{0.008840in}}{\pgfqpoint{-0.033333in}{0.000000in}}%
\pgfpathcurveto{\pgfqpoint{-0.033333in}{-0.008840in}}{\pgfqpoint{-0.029821in}{-0.017319in}}{\pgfqpoint{-0.023570in}{-0.023570in}}%
\pgfpathcurveto{\pgfqpoint{-0.017319in}{-0.029821in}}{\pgfqpoint{-0.008840in}{-0.033333in}}{\pgfqpoint{0.000000in}{-0.033333in}}%
\pgfpathclose%
\pgfusepath{stroke,fill}%
\end{pgfscope}%
\begin{pgfscope}%
\pgfpathrectangle{\pgfqpoint{0.800000in}{0.528000in}}{\pgfqpoint{4.960000in}{3.696000in}}%
\pgfusepath{clip}%
\pgfsetbuttcap%
\pgfsetroundjoin%
\definecolor{currentfill}{rgb}{0.933333,0.521569,0.290196}%
\pgfsetfillcolor{currentfill}%
\pgfsetlinewidth{0.803000pt}%
\definecolor{currentstroke}{rgb}{0.933333,0.521569,0.290196}%
\pgfsetstrokecolor{currentstroke}%
\pgfsetdash{}{0pt}%
\pgfpathmoveto{\pgfqpoint{0.000000in}{-0.033333in}}%
\pgfpathcurveto{\pgfqpoint{0.008840in}{-0.033333in}}{\pgfqpoint{0.017319in}{-0.029821in}}{\pgfqpoint{0.023570in}{-0.023570in}}%
\pgfpathcurveto{\pgfqpoint{0.029821in}{-0.017319in}}{\pgfqpoint{0.033333in}{-0.008840in}}{\pgfqpoint{0.033333in}{0.000000in}}%
\pgfpathcurveto{\pgfqpoint{0.033333in}{0.008840in}}{\pgfqpoint{0.029821in}{0.017319in}}{\pgfqpoint{0.023570in}{0.023570in}}%
\pgfpathcurveto{\pgfqpoint{0.017319in}{0.029821in}}{\pgfqpoint{0.008840in}{0.033333in}}{\pgfqpoint{0.000000in}{0.033333in}}%
\pgfpathcurveto{\pgfqpoint{-0.008840in}{0.033333in}}{\pgfqpoint{-0.017319in}{0.029821in}}{\pgfqpoint{-0.023570in}{0.023570in}}%
\pgfpathcurveto{\pgfqpoint{-0.029821in}{0.017319in}}{\pgfqpoint{-0.033333in}{0.008840in}}{\pgfqpoint{-0.033333in}{0.000000in}}%
\pgfpathcurveto{\pgfqpoint{-0.033333in}{-0.008840in}}{\pgfqpoint{-0.029821in}{-0.017319in}}{\pgfqpoint{-0.023570in}{-0.023570in}}%
\pgfpathcurveto{\pgfqpoint{-0.017319in}{-0.029821in}}{\pgfqpoint{-0.008840in}{-0.033333in}}{\pgfqpoint{0.000000in}{-0.033333in}}%
\pgfpathclose%
\pgfusepath{stroke,fill}%
\end{pgfscope}%
\begin{pgfscope}%
\pgfpathrectangle{\pgfqpoint{0.800000in}{0.528000in}}{\pgfqpoint{4.960000in}{3.696000in}}%
\pgfusepath{clip}%
\pgfsetbuttcap%
\pgfsetroundjoin%
\definecolor{currentfill}{rgb}{0.415686,0.800000,0.392157}%
\pgfsetfillcolor{currentfill}%
\pgfsetlinewidth{0.803000pt}%
\definecolor{currentstroke}{rgb}{0.415686,0.800000,0.392157}%
\pgfsetstrokecolor{currentstroke}%
\pgfsetdash{}{0pt}%
\pgfpathmoveto{\pgfqpoint{0.000000in}{-0.033333in}}%
\pgfpathcurveto{\pgfqpoint{0.008840in}{-0.033333in}}{\pgfqpoint{0.017319in}{-0.029821in}}{\pgfqpoint{0.023570in}{-0.023570in}}%
\pgfpathcurveto{\pgfqpoint{0.029821in}{-0.017319in}}{\pgfqpoint{0.033333in}{-0.008840in}}{\pgfqpoint{0.033333in}{0.000000in}}%
\pgfpathcurveto{\pgfqpoint{0.033333in}{0.008840in}}{\pgfqpoint{0.029821in}{0.017319in}}{\pgfqpoint{0.023570in}{0.023570in}}%
\pgfpathcurveto{\pgfqpoint{0.017319in}{0.029821in}}{\pgfqpoint{0.008840in}{0.033333in}}{\pgfqpoint{0.000000in}{0.033333in}}%
\pgfpathcurveto{\pgfqpoint{-0.008840in}{0.033333in}}{\pgfqpoint{-0.017319in}{0.029821in}}{\pgfqpoint{-0.023570in}{0.023570in}}%
\pgfpathcurveto{\pgfqpoint{-0.029821in}{0.017319in}}{\pgfqpoint{-0.033333in}{0.008840in}}{\pgfqpoint{-0.033333in}{0.000000in}}%
\pgfpathcurveto{\pgfqpoint{-0.033333in}{-0.008840in}}{\pgfqpoint{-0.029821in}{-0.017319in}}{\pgfqpoint{-0.023570in}{-0.023570in}}%
\pgfpathcurveto{\pgfqpoint{-0.017319in}{-0.029821in}}{\pgfqpoint{-0.008840in}{-0.033333in}}{\pgfqpoint{0.000000in}{-0.033333in}}%
\pgfpathclose%
\pgfusepath{stroke,fill}%
\end{pgfscope}%
\begin{pgfscope}%
\pgfpathrectangle{\pgfqpoint{0.800000in}{0.528000in}}{\pgfqpoint{4.960000in}{3.696000in}}%
\pgfusepath{clip}%
\pgfsetbuttcap%
\pgfsetroundjoin%
\definecolor{currentfill}{rgb}{0.839216,0.372549,0.372549}%
\pgfsetfillcolor{currentfill}%
\pgfsetlinewidth{0.803000pt}%
\definecolor{currentstroke}{rgb}{0.839216,0.372549,0.372549}%
\pgfsetstrokecolor{currentstroke}%
\pgfsetdash{}{0pt}%
\pgfpathmoveto{\pgfqpoint{0.000000in}{-0.033333in}}%
\pgfpathcurveto{\pgfqpoint{0.008840in}{-0.033333in}}{\pgfqpoint{0.017319in}{-0.029821in}}{\pgfqpoint{0.023570in}{-0.023570in}}%
\pgfpathcurveto{\pgfqpoint{0.029821in}{-0.017319in}}{\pgfqpoint{0.033333in}{-0.008840in}}{\pgfqpoint{0.033333in}{0.000000in}}%
\pgfpathcurveto{\pgfqpoint{0.033333in}{0.008840in}}{\pgfqpoint{0.029821in}{0.017319in}}{\pgfqpoint{0.023570in}{0.023570in}}%
\pgfpathcurveto{\pgfqpoint{0.017319in}{0.029821in}}{\pgfqpoint{0.008840in}{0.033333in}}{\pgfqpoint{0.000000in}{0.033333in}}%
\pgfpathcurveto{\pgfqpoint{-0.008840in}{0.033333in}}{\pgfqpoint{-0.017319in}{0.029821in}}{\pgfqpoint{-0.023570in}{0.023570in}}%
\pgfpathcurveto{\pgfqpoint{-0.029821in}{0.017319in}}{\pgfqpoint{-0.033333in}{0.008840in}}{\pgfqpoint{-0.033333in}{0.000000in}}%
\pgfpathcurveto{\pgfqpoint{-0.033333in}{-0.008840in}}{\pgfqpoint{-0.029821in}{-0.017319in}}{\pgfqpoint{-0.023570in}{-0.023570in}}%
\pgfpathcurveto{\pgfqpoint{-0.017319in}{-0.029821in}}{\pgfqpoint{-0.008840in}{-0.033333in}}{\pgfqpoint{0.000000in}{-0.033333in}}%
\pgfpathclose%
\pgfusepath{stroke,fill}%
\end{pgfscope}%
\begin{pgfscope}%
\pgfpathrectangle{\pgfqpoint{0.800000in}{0.528000in}}{\pgfqpoint{4.960000in}{3.696000in}}%
\pgfusepath{clip}%
\pgfsetroundcap%
\pgfsetroundjoin%
\pgfsetlinewidth{1.204500pt}%
\definecolor{currentstroke}{rgb}{0.100000,0.100000,0.100000}%
\pgfsetstrokecolor{currentstroke}%
\pgfsetstrokeopacity{0.600000}%
\pgfsetdash{}{0pt}%
\pgfpathmoveto{\pgfqpoint{1.025455in}{0.778268in}}%
\pgfpathlineto{\pgfqpoint{5.534545in}{4.035002in}}%
\pgfpathlineto{\pgfqpoint{5.534545in}{4.035002in}}%
\pgfusepath{stroke}%
\end{pgfscope}%
\begin{pgfscope}%
\pgfpathrectangle{\pgfqpoint{0.800000in}{0.528000in}}{\pgfqpoint{4.960000in}{3.696000in}}%
\pgfusepath{clip}%
\pgfsetroundcap%
\pgfsetroundjoin%
\pgfsetlinewidth{1.204500pt}%
\definecolor{currentstroke}{rgb}{0.100000,0.100000,0.100000}%
\pgfsetstrokecolor{currentstroke}%
\pgfsetstrokeopacity{0.300000}%
\pgfsetdash{}{0pt}%
\pgfpathmoveto{\pgfqpoint{1.025455in}{0.696000in}}%
\pgfpathlineto{\pgfqpoint{5.534545in}{4.056000in}}%
\pgfpathlineto{\pgfqpoint{5.534545in}{4.056000in}}%
\pgfusepath{stroke}%
\end{pgfscope}%
\begin{pgfscope}%
\pgfsetrectcap%
\pgfsetmiterjoin%
\pgfsetlinewidth{1.003750pt}%
\definecolor{currentstroke}{rgb}{1.000000,1.000000,1.000000}%
\pgfsetstrokecolor{currentstroke}%
\pgfsetdash{}{0pt}%
\pgfpathmoveto{\pgfqpoint{0.800000in}{0.528000in}}%
\pgfpathlineto{\pgfqpoint{0.800000in}{4.224000in}}%
\pgfusepath{stroke}%
\end{pgfscope}%
\begin{pgfscope}%
\pgfsetrectcap%
\pgfsetmiterjoin%
\pgfsetlinewidth{1.003750pt}%
\definecolor{currentstroke}{rgb}{1.000000,1.000000,1.000000}%
\pgfsetstrokecolor{currentstroke}%
\pgfsetdash{}{0pt}%
\pgfpathmoveto{\pgfqpoint{5.760000in}{0.528000in}}%
\pgfpathlineto{\pgfqpoint{5.760000in}{4.224000in}}%
\pgfusepath{stroke}%
\end{pgfscope}%
\begin{pgfscope}%
\pgfsetrectcap%
\pgfsetmiterjoin%
\pgfsetlinewidth{1.003750pt}%
\definecolor{currentstroke}{rgb}{1.000000,1.000000,1.000000}%
\pgfsetstrokecolor{currentstroke}%
\pgfsetdash{}{0pt}%
\pgfpathmoveto{\pgfqpoint{0.800000in}{0.528000in}}%
\pgfpathlineto{\pgfqpoint{5.760000in}{0.528000in}}%
\pgfusepath{stroke}%
\end{pgfscope}%
\begin{pgfscope}%
\pgfsetrectcap%
\pgfsetmiterjoin%
\pgfsetlinewidth{1.003750pt}%
\definecolor{currentstroke}{rgb}{1.000000,1.000000,1.000000}%
\pgfsetstrokecolor{currentstroke}%
\pgfsetdash{}{0pt}%
\pgfpathmoveto{\pgfqpoint{0.800000in}{4.224000in}}%
\pgfpathlineto{\pgfqpoint{5.760000in}{4.224000in}}%
\pgfusepath{stroke}%
\end{pgfscope}%
\begin{pgfscope}%
\pgfsetbuttcap%
\pgfsetmiterjoin%
\definecolor{currentfill}{rgb}{0.917647,0.917647,0.949020}%
\pgfsetfillcolor{currentfill}%
\pgfsetfillopacity{0.800000}%
\pgfsetlinewidth{0.803000pt}%
\definecolor{currentstroke}{rgb}{0.800000,0.800000,0.800000}%
\pgfsetstrokecolor{currentstroke}%
\pgfsetstrokeopacity{0.800000}%
\pgfsetdash{}{0pt}%
\pgfpathmoveto{\pgfqpoint{0.885556in}{3.227519in}}%
\pgfpathlineto{\pgfqpoint{1.871844in}{3.227519in}}%
\pgfpathquadraticcurveto{\pgfqpoint{1.896288in}{3.227519in}}{\pgfqpoint{1.896288in}{3.251964in}}%
\pgfpathlineto{\pgfqpoint{1.896288in}{4.138444in}}%
\pgfpathquadraticcurveto{\pgfqpoint{1.896288in}{4.162889in}}{\pgfqpoint{1.871844in}{4.162889in}}%
\pgfpathlineto{\pgfqpoint{0.885556in}{4.162889in}}%
\pgfpathquadraticcurveto{\pgfqpoint{0.861111in}{4.162889in}}{\pgfqpoint{0.861111in}{4.138444in}}%
\pgfpathlineto{\pgfqpoint{0.861111in}{3.251964in}}%
\pgfpathquadraticcurveto{\pgfqpoint{0.861111in}{3.227519in}}{\pgfqpoint{0.885556in}{3.227519in}}%
\pgfpathclose%
\pgfusepath{stroke,fill}%
\end{pgfscope}%
\begin{pgfscope}%
\pgfsetbuttcap%
\pgfsetroundjoin%
\definecolor{currentfill}{rgb}{1.000000,1.000000,1.000000}%
\pgfsetfillcolor{currentfill}%
\pgfsetlinewidth{1.204500pt}%
\definecolor{currentstroke}{rgb}{1.000000,1.000000,1.000000}%
\pgfsetstrokecolor{currentstroke}%
\pgfsetdash{}{0pt}%
\pgfsys@defobject{currentmarker}{\pgfqpoint{infin}{infin}}{\pgfqpoint{-infin}{-infin}}{%
\pgfusepath{stroke,fill}%
}%
\begin{pgfscope}%
\pgfsys@transformshift{1.032222in}{4.053223in}%
\pgfsys@useobject{currentmarker}{}%
\end{pgfscope}%
\end{pgfscope}%
\begin{pgfscope}%
\definecolor{textcolor}{rgb}{0.150000,0.150000,0.150000}%
\pgfsetstrokecolor{textcolor}%
\pgfsetfillcolor{textcolor}%
\pgftext[x=1.252222in,y=4.021140in,left,base]{\color{textcolor}\rmfamily\fontsize{8.800000}{10.560000}\selectfont composto}%
\end{pgfscope}%
\begin{pgfscope}%
\pgfsetbuttcap%
\pgfsetroundjoin%
\definecolor{currentfill}{rgb}{0.282353,0.470588,0.815686}%
\pgfsetfillcolor{currentfill}%
\pgfsetlinewidth{0.803000pt}%
\definecolor{currentstroke}{rgb}{0.282353,0.470588,0.815686}%
\pgfsetstrokecolor{currentstroke}%
\pgfsetdash{}{0pt}%
\pgfpathmoveto{\pgfqpoint{1.032222in}{3.840495in}}%
\pgfpathcurveto{\pgfqpoint{1.041062in}{3.840495in}}{\pgfqpoint{1.049542in}{3.844007in}}{\pgfqpoint{1.055792in}{3.850258in}}%
\pgfpathcurveto{\pgfqpoint{1.062043in}{3.856509in}}{\pgfqpoint{1.065556in}{3.864989in}}{\pgfqpoint{1.065556in}{3.873829in}}%
\pgfpathcurveto{\pgfqpoint{1.065556in}{3.882669in}}{\pgfqpoint{1.062043in}{3.891148in}}{\pgfqpoint{1.055792in}{3.897399in}}%
\pgfpathcurveto{\pgfqpoint{1.049542in}{3.903650in}}{\pgfqpoint{1.041062in}{3.907162in}}{\pgfqpoint{1.032222in}{3.907162in}}%
\pgfpathcurveto{\pgfqpoint{1.023382in}{3.907162in}}{\pgfqpoint{1.014903in}{3.903650in}}{\pgfqpoint{1.008652in}{3.897399in}}%
\pgfpathcurveto{\pgfqpoint{1.002401in}{3.891148in}}{\pgfqpoint{0.998889in}{3.882669in}}{\pgfqpoint{0.998889in}{3.873829in}}%
\pgfpathcurveto{\pgfqpoint{0.998889in}{3.864989in}}{\pgfqpoint{1.002401in}{3.856509in}}{\pgfqpoint{1.008652in}{3.850258in}}%
\pgfpathcurveto{\pgfqpoint{1.014903in}{3.844007in}}{\pgfqpoint{1.023382in}{3.840495in}}{\pgfqpoint{1.032222in}{3.840495in}}%
\pgfpathclose%
\pgfusepath{stroke,fill}%
\end{pgfscope}%
\begin{pgfscope}%
\definecolor{textcolor}{rgb}{0.150000,0.150000,0.150000}%
\pgfsetstrokecolor{textcolor}%
\pgfsetfillcolor{textcolor}%
\pgftext[x=1.252222in,y=3.841745in,left,base]{\color{textcolor}\rmfamily\fontsize{8.800000}{10.560000}\selectfont Cd}%
\end{pgfscope}%
\begin{pgfscope}%
\pgfsetbuttcap%
\pgfsetroundjoin%
\definecolor{currentfill}{rgb}{0.933333,0.521569,0.290196}%
\pgfsetfillcolor{currentfill}%
\pgfsetlinewidth{0.803000pt}%
\definecolor{currentstroke}{rgb}{0.933333,0.521569,0.290196}%
\pgfsetstrokecolor{currentstroke}%
\pgfsetdash{}{0pt}%
\pgfpathmoveto{\pgfqpoint{1.032222in}{3.661101in}}%
\pgfpathcurveto{\pgfqpoint{1.041062in}{3.661101in}}{\pgfqpoint{1.049542in}{3.664613in}}{\pgfqpoint{1.055792in}{3.670864in}}%
\pgfpathcurveto{\pgfqpoint{1.062043in}{3.677115in}}{\pgfqpoint{1.065556in}{3.685594in}}{\pgfqpoint{1.065556in}{3.694434in}}%
\pgfpathcurveto{\pgfqpoint{1.065556in}{3.703274in}}{\pgfqpoint{1.062043in}{3.711753in}}{\pgfqpoint{1.055792in}{3.718004in}}%
\pgfpathcurveto{\pgfqpoint{1.049542in}{3.724255in}}{\pgfqpoint{1.041062in}{3.727767in}}{\pgfqpoint{1.032222in}{3.727767in}}%
\pgfpathcurveto{\pgfqpoint{1.023382in}{3.727767in}}{\pgfqpoint{1.014903in}{3.724255in}}{\pgfqpoint{1.008652in}{3.718004in}}%
\pgfpathcurveto{\pgfqpoint{1.002401in}{3.711753in}}{\pgfqpoint{0.998889in}{3.703274in}}{\pgfqpoint{0.998889in}{3.694434in}}%
\pgfpathcurveto{\pgfqpoint{0.998889in}{3.685594in}}{\pgfqpoint{1.002401in}{3.677115in}}{\pgfqpoint{1.008652in}{3.670864in}}%
\pgfpathcurveto{\pgfqpoint{1.014903in}{3.664613in}}{\pgfqpoint{1.023382in}{3.661101in}}{\pgfqpoint{1.032222in}{3.661101in}}%
\pgfpathclose%
\pgfusepath{stroke,fill}%
\end{pgfscope}%
\begin{pgfscope}%
\definecolor{textcolor}{rgb}{0.150000,0.150000,0.150000}%
\pgfsetstrokecolor{textcolor}%
\pgfsetfillcolor{textcolor}%
\pgftext[x=1.252222in,y=3.662351in,left,base]{\color{textcolor}\rmfamily\fontsize{8.800000}{10.560000}\selectfont Na}%
\end{pgfscope}%
\begin{pgfscope}%
\pgfsetbuttcap%
\pgfsetroundjoin%
\definecolor{currentfill}{rgb}{0.415686,0.800000,0.392157}%
\pgfsetfillcolor{currentfill}%
\pgfsetlinewidth{0.803000pt}%
\definecolor{currentstroke}{rgb}{0.415686,0.800000,0.392157}%
\pgfsetstrokecolor{currentstroke}%
\pgfsetdash{}{0pt}%
\pgfpathmoveto{\pgfqpoint{1.032222in}{3.481706in}}%
\pgfpathcurveto{\pgfqpoint{1.041062in}{3.481706in}}{\pgfqpoint{1.049542in}{3.485219in}}{\pgfqpoint{1.055792in}{3.491469in}}%
\pgfpathcurveto{\pgfqpoint{1.062043in}{3.497720in}}{\pgfqpoint{1.065556in}{3.506200in}}{\pgfqpoint{1.065556in}{3.515040in}}%
\pgfpathcurveto{\pgfqpoint{1.065556in}{3.523880in}}{\pgfqpoint{1.062043in}{3.532359in}}{\pgfqpoint{1.055792in}{3.538610in}}%
\pgfpathcurveto{\pgfqpoint{1.049542in}{3.544861in}}{\pgfqpoint{1.041062in}{3.548373in}}{\pgfqpoint{1.032222in}{3.548373in}}%
\pgfpathcurveto{\pgfqpoint{1.023382in}{3.548373in}}{\pgfqpoint{1.014903in}{3.544861in}}{\pgfqpoint{1.008652in}{3.538610in}}%
\pgfpathcurveto{\pgfqpoint{1.002401in}{3.532359in}}{\pgfqpoint{0.998889in}{3.523880in}}{\pgfqpoint{0.998889in}{3.515040in}}%
\pgfpathcurveto{\pgfqpoint{0.998889in}{3.506200in}}{\pgfqpoint{1.002401in}{3.497720in}}{\pgfqpoint{1.008652in}{3.491469in}}%
\pgfpathcurveto{\pgfqpoint{1.014903in}{3.485219in}}{\pgfqpoint{1.023382in}{3.481706in}}{\pgfqpoint{1.032222in}{3.481706in}}%
\pgfpathclose%
\pgfusepath{stroke,fill}%
\end{pgfscope}%
\begin{pgfscope}%
\definecolor{textcolor}{rgb}{0.150000,0.150000,0.150000}%
\pgfsetstrokecolor{textcolor}%
\pgfsetfillcolor{textcolor}%
\pgftext[x=1.252222in,y=3.482956in,left,base]{\color{textcolor}\rmfamily\fontsize{8.800000}{10.560000}\selectfont Hg}%
\end{pgfscope}%
\begin{pgfscope}%
\pgfsetbuttcap%
\pgfsetroundjoin%
\definecolor{currentfill}{rgb}{0.839216,0.372549,0.372549}%
\pgfsetfillcolor{currentfill}%
\pgfsetlinewidth{0.803000pt}%
\definecolor{currentstroke}{rgb}{0.839216,0.372549,0.372549}%
\pgfsetstrokecolor{currentstroke}%
\pgfsetdash{}{0pt}%
\pgfpathmoveto{\pgfqpoint{1.032222in}{3.300581in}}%
\pgfpathcurveto{\pgfqpoint{1.041062in}{3.300581in}}{\pgfqpoint{1.049542in}{3.304094in}}{\pgfqpoint{1.055792in}{3.310344in}}%
\pgfpathcurveto{\pgfqpoint{1.062043in}{3.316595in}}{\pgfqpoint{1.065556in}{3.325075in}}{\pgfqpoint{1.065556in}{3.333915in}}%
\pgfpathcurveto{\pgfqpoint{1.065556in}{3.342755in}}{\pgfqpoint{1.062043in}{3.351234in}}{\pgfqpoint{1.055792in}{3.357485in}}%
\pgfpathcurveto{\pgfqpoint{1.049542in}{3.363736in}}{\pgfqpoint{1.041062in}{3.367248in}}{\pgfqpoint{1.032222in}{3.367248in}}%
\pgfpathcurveto{\pgfqpoint{1.023382in}{3.367248in}}{\pgfqpoint{1.014903in}{3.363736in}}{\pgfqpoint{1.008652in}{3.357485in}}%
\pgfpathcurveto{\pgfqpoint{1.002401in}{3.351234in}}{\pgfqpoint{0.998889in}{3.342755in}}{\pgfqpoint{0.998889in}{3.333915in}}%
\pgfpathcurveto{\pgfqpoint{0.998889in}{3.325075in}}{\pgfqpoint{1.002401in}{3.316595in}}{\pgfqpoint{1.008652in}{3.310344in}}%
\pgfpathcurveto{\pgfqpoint{1.014903in}{3.304094in}}{\pgfqpoint{1.023382in}{3.300581in}}{\pgfqpoint{1.032222in}{3.300581in}}%
\pgfpathclose%
\pgfusepath{stroke,fill}%
\end{pgfscope}%
\begin{pgfscope}%
\definecolor{textcolor}{rgb}{0.150000,0.150000,0.150000}%
\pgfsetstrokecolor{textcolor}%
\pgfsetfillcolor{textcolor}%
\pgftext[x=1.252222in,y=3.301831in,left,base]{\color{textcolor}\rmfamily\fontsize{8.800000}{10.560000}\selectfont He}%
\end{pgfscope}%
\end{pgfpicture}%
\makeatother%
\endgroup%


	\caption{Reta}
	\label{fig:reta}
\end{figure}

\begin{table}[H]
	\centering
	\sisetup{
        round-precision = 5,
        minimum-integer-digits = 2
	}
	\begin{tabular}{cc}
		\toprule\toprule
            {\bfseries Coeficiente} & {\bfseries Medida}
        \\\midrule
            $A$ & $1.607 \pm 0.008$ \\
            $B$ & $0.0080 \pm 0.0004 \si{\micro\meter^2}$
        \\\bottomrule\bottomrule
	\end{tabular}

	\caption{Coeficientes da fórmula de Cauchy}
	\label{tab:regres}
\end{table}

Por fim, com os valores da regressão, a fórmula de Cauchy pôde ser utilizada na forma inversa (\ref{eq:lambda}) para montar a relação de dispersão na figura \ref{fig:cauchy}, que pode ser usado em conjunto com com os materiais utilizados como um espectrômetro. As faixas de incerteza foram calculadas pela eq. \ref{eq:u:lambda}.

\begin{figure}[H]
	\centering

	%% Creator: Matplotlib, PGF backend
%%
%% To include the figure in your LaTeX document, write
%%   \input{<filename>.pgf}
%%
%% Make sure the required packages are loaded in your preamble
%%   \usepackage{pgf}
%%
%% Figures using additional raster images can only be included by \input if
%% they are in the same directory as the main LaTeX file. For loading figures
%% from other directories you can use the `import` package
%%   \usepackage{import}
%% and then include the figures with
%%   \import{<path to file>}{<filename>.pgf}
%%
%% Matplotlib used the following preamble
%%   \usepackage[utf8]{inputenc}
%%   \usepackage[T1]{fontenc}
%%   \usepackage[portuguese]{babel}
%%   \usepackage{siunitx}
%%   \usepackage{gensymb}
%%   \usepackage{fontspec}
%%   \setmainfont{DejaVuSerif.ttf}[Path=/home/marmis/.virtualenvs/default/lib/python3.7/site-packages/matplotlib/mpl-data/fonts/ttf/]
%%   \setsansfont{arial.ttf}[Path=/usr/share/fonts/TTF/]
%%   \setmonofont{DejaVuSansMono.ttf}[Path=/home/marmis/.virtualenvs/default/lib/python3.7/site-packages/matplotlib/mpl-data/fonts/ttf/]
%%
\begingroup%
\makeatletter%
\begin{pgfpicture}%
\pgfpathrectangle{\pgfpointorigin}{\pgfqpoint{5.000000in}{3.000000in}}%
\pgfusepath{use as bounding box, clip}%
\begin{pgfscope}%
\pgfsetbuttcap%
\pgfsetmiterjoin%
\definecolor{currentfill}{rgb}{1.000000,1.000000,1.000000}%
\pgfsetfillcolor{currentfill}%
\pgfsetlinewidth{0.000000pt}%
\definecolor{currentstroke}{rgb}{1.000000,1.000000,1.000000}%
\pgfsetstrokecolor{currentstroke}%
\pgfsetdash{}{0pt}%
\pgfpathmoveto{\pgfqpoint{0.000000in}{0.000000in}}%
\pgfpathlineto{\pgfqpoint{5.000000in}{0.000000in}}%
\pgfpathlineto{\pgfqpoint{5.000000in}{3.000000in}}%
\pgfpathlineto{\pgfqpoint{0.000000in}{3.000000in}}%
\pgfpathclose%
\pgfusepath{fill}%
\end{pgfscope}%
\begin{pgfscope}%
\pgfsetbuttcap%
\pgfsetmiterjoin%
\definecolor{currentfill}{rgb}{0.917647,0.917647,0.949020}%
\pgfsetfillcolor{currentfill}%
\pgfsetlinewidth{0.000000pt}%
\definecolor{currentstroke}{rgb}{0.000000,0.000000,0.000000}%
\pgfsetstrokecolor{currentstroke}%
\pgfsetstrokeopacity{0.000000}%
\pgfsetdash{}{0pt}%
\pgfpathmoveto{\pgfqpoint{0.636577in}{0.562153in}}%
\pgfpathlineto{\pgfqpoint{4.808000in}{0.562153in}}%
\pgfpathlineto{\pgfqpoint{4.808000in}{2.666026in}}%
\pgfpathlineto{\pgfqpoint{0.636577in}{2.666026in}}%
\pgfpathclose%
\pgfusepath{fill}%
\end{pgfscope}%
\begin{pgfscope}%
\pgfpathrectangle{\pgfqpoint{0.636577in}{0.562153in}}{\pgfqpoint{4.171423in}{2.103873in}}%
\pgfusepath{clip}%
\pgfsetroundcap%
\pgfsetroundjoin%
\pgfsetlinewidth{0.803000pt}%
\definecolor{currentstroke}{rgb}{1.000000,1.000000,1.000000}%
\pgfsetstrokecolor{currentstroke}%
\pgfsetdash{}{0pt}%
\pgfpathmoveto{\pgfqpoint{1.257115in}{0.562153in}}%
\pgfpathlineto{\pgfqpoint{1.257115in}{2.666026in}}%
\pgfusepath{stroke}%
\end{pgfscope}%
\begin{pgfscope}%
\pgfsetbuttcap%
\pgfsetroundjoin%
\definecolor{currentfill}{rgb}{0.411765,0.411765,0.411765}%
\pgfsetfillcolor{currentfill}%
\pgfsetlinewidth{1.003750pt}%
\definecolor{currentstroke}{rgb}{0.411765,0.411765,0.411765}%
\pgfsetstrokecolor{currentstroke}%
\pgfsetdash{}{0pt}%
\pgfsys@defobject{currentmarker}{\pgfqpoint{0.000000in}{-0.066667in}}{\pgfqpoint{0.000000in}{0.000000in}}{%
\pgfpathmoveto{\pgfqpoint{0.000000in}{0.000000in}}%
\pgfpathlineto{\pgfqpoint{0.000000in}{-0.066667in}}%
\pgfusepath{stroke,fill}%
}%
\begin{pgfscope}%
\pgfsys@transformshift{1.257115in}{0.562153in}%
\pgfsys@useobject{currentmarker}{}%
\end{pgfscope}%
\end{pgfscope}%
\begin{pgfscope}%
\definecolor{textcolor}{rgb}{0.411765,0.411765,0.411765}%
\pgfsetstrokecolor{textcolor}%
\pgfsetfillcolor{textcolor}%
\pgftext[x=1.257115in,y=0.446875in,,top]{\color{textcolor}\rmfamily\fontsize{8.800000}{10.560000}\selectfont \(\displaystyle 47.5\)}%
\end{pgfscope}%
\begin{pgfscope}%
\pgfpathrectangle{\pgfqpoint{0.636577in}{0.562153in}}{\pgfqpoint{4.171423in}{2.103873in}}%
\pgfusepath{clip}%
\pgfsetroundcap%
\pgfsetroundjoin%
\pgfsetlinewidth{0.803000pt}%
\definecolor{currentstroke}{rgb}{1.000000,1.000000,1.000000}%
\pgfsetstrokecolor{currentstroke}%
\pgfsetdash{}{0pt}%
\pgfpathmoveto{\pgfqpoint{1.885046in}{0.562153in}}%
\pgfpathlineto{\pgfqpoint{1.885046in}{2.666026in}}%
\pgfusepath{stroke}%
\end{pgfscope}%
\begin{pgfscope}%
\pgfsetbuttcap%
\pgfsetroundjoin%
\definecolor{currentfill}{rgb}{0.411765,0.411765,0.411765}%
\pgfsetfillcolor{currentfill}%
\pgfsetlinewidth{1.003750pt}%
\definecolor{currentstroke}{rgb}{0.411765,0.411765,0.411765}%
\pgfsetstrokecolor{currentstroke}%
\pgfsetdash{}{0pt}%
\pgfsys@defobject{currentmarker}{\pgfqpoint{0.000000in}{-0.066667in}}{\pgfqpoint{0.000000in}{0.000000in}}{%
\pgfpathmoveto{\pgfqpoint{0.000000in}{0.000000in}}%
\pgfpathlineto{\pgfqpoint{0.000000in}{-0.066667in}}%
\pgfusepath{stroke,fill}%
}%
\begin{pgfscope}%
\pgfsys@transformshift{1.885046in}{0.562153in}%
\pgfsys@useobject{currentmarker}{}%
\end{pgfscope}%
\end{pgfscope}%
\begin{pgfscope}%
\definecolor{textcolor}{rgb}{0.411765,0.411765,0.411765}%
\pgfsetstrokecolor{textcolor}%
\pgfsetfillcolor{textcolor}%
\pgftext[x=1.885046in,y=0.446875in,,top]{\color{textcolor}\rmfamily\fontsize{8.800000}{10.560000}\selectfont \(\displaystyle 48.0\)}%
\end{pgfscope}%
\begin{pgfscope}%
\pgfpathrectangle{\pgfqpoint{0.636577in}{0.562153in}}{\pgfqpoint{4.171423in}{2.103873in}}%
\pgfusepath{clip}%
\pgfsetroundcap%
\pgfsetroundjoin%
\pgfsetlinewidth{0.803000pt}%
\definecolor{currentstroke}{rgb}{1.000000,1.000000,1.000000}%
\pgfsetstrokecolor{currentstroke}%
\pgfsetdash{}{0pt}%
\pgfpathmoveto{\pgfqpoint{2.512978in}{0.562153in}}%
\pgfpathlineto{\pgfqpoint{2.512978in}{2.666026in}}%
\pgfusepath{stroke}%
\end{pgfscope}%
\begin{pgfscope}%
\pgfsetbuttcap%
\pgfsetroundjoin%
\definecolor{currentfill}{rgb}{0.411765,0.411765,0.411765}%
\pgfsetfillcolor{currentfill}%
\pgfsetlinewidth{1.003750pt}%
\definecolor{currentstroke}{rgb}{0.411765,0.411765,0.411765}%
\pgfsetstrokecolor{currentstroke}%
\pgfsetdash{}{0pt}%
\pgfsys@defobject{currentmarker}{\pgfqpoint{0.000000in}{-0.066667in}}{\pgfqpoint{0.000000in}{0.000000in}}{%
\pgfpathmoveto{\pgfqpoint{0.000000in}{0.000000in}}%
\pgfpathlineto{\pgfqpoint{0.000000in}{-0.066667in}}%
\pgfusepath{stroke,fill}%
}%
\begin{pgfscope}%
\pgfsys@transformshift{2.512978in}{0.562153in}%
\pgfsys@useobject{currentmarker}{}%
\end{pgfscope}%
\end{pgfscope}%
\begin{pgfscope}%
\definecolor{textcolor}{rgb}{0.411765,0.411765,0.411765}%
\pgfsetstrokecolor{textcolor}%
\pgfsetfillcolor{textcolor}%
\pgftext[x=2.512978in,y=0.446875in,,top]{\color{textcolor}\rmfamily\fontsize{8.800000}{10.560000}\selectfont \(\displaystyle 48.5\)}%
\end{pgfscope}%
\begin{pgfscope}%
\pgfpathrectangle{\pgfqpoint{0.636577in}{0.562153in}}{\pgfqpoint{4.171423in}{2.103873in}}%
\pgfusepath{clip}%
\pgfsetroundcap%
\pgfsetroundjoin%
\pgfsetlinewidth{0.803000pt}%
\definecolor{currentstroke}{rgb}{1.000000,1.000000,1.000000}%
\pgfsetstrokecolor{currentstroke}%
\pgfsetdash{}{0pt}%
\pgfpathmoveto{\pgfqpoint{3.140910in}{0.562153in}}%
\pgfpathlineto{\pgfqpoint{3.140910in}{2.666026in}}%
\pgfusepath{stroke}%
\end{pgfscope}%
\begin{pgfscope}%
\pgfsetbuttcap%
\pgfsetroundjoin%
\definecolor{currentfill}{rgb}{0.411765,0.411765,0.411765}%
\pgfsetfillcolor{currentfill}%
\pgfsetlinewidth{1.003750pt}%
\definecolor{currentstroke}{rgb}{0.411765,0.411765,0.411765}%
\pgfsetstrokecolor{currentstroke}%
\pgfsetdash{}{0pt}%
\pgfsys@defobject{currentmarker}{\pgfqpoint{0.000000in}{-0.066667in}}{\pgfqpoint{0.000000in}{0.000000in}}{%
\pgfpathmoveto{\pgfqpoint{0.000000in}{0.000000in}}%
\pgfpathlineto{\pgfqpoint{0.000000in}{-0.066667in}}%
\pgfusepath{stroke,fill}%
}%
\begin{pgfscope}%
\pgfsys@transformshift{3.140910in}{0.562153in}%
\pgfsys@useobject{currentmarker}{}%
\end{pgfscope}%
\end{pgfscope}%
\begin{pgfscope}%
\definecolor{textcolor}{rgb}{0.411765,0.411765,0.411765}%
\pgfsetstrokecolor{textcolor}%
\pgfsetfillcolor{textcolor}%
\pgftext[x=3.140910in,y=0.446875in,,top]{\color{textcolor}\rmfamily\fontsize{8.800000}{10.560000}\selectfont \(\displaystyle 49.0\)}%
\end{pgfscope}%
\begin{pgfscope}%
\pgfpathrectangle{\pgfqpoint{0.636577in}{0.562153in}}{\pgfqpoint{4.171423in}{2.103873in}}%
\pgfusepath{clip}%
\pgfsetroundcap%
\pgfsetroundjoin%
\pgfsetlinewidth{0.803000pt}%
\definecolor{currentstroke}{rgb}{1.000000,1.000000,1.000000}%
\pgfsetstrokecolor{currentstroke}%
\pgfsetdash{}{0pt}%
\pgfpathmoveto{\pgfqpoint{3.768841in}{0.562153in}}%
\pgfpathlineto{\pgfqpoint{3.768841in}{2.666026in}}%
\pgfusepath{stroke}%
\end{pgfscope}%
\begin{pgfscope}%
\pgfsetbuttcap%
\pgfsetroundjoin%
\definecolor{currentfill}{rgb}{0.411765,0.411765,0.411765}%
\pgfsetfillcolor{currentfill}%
\pgfsetlinewidth{1.003750pt}%
\definecolor{currentstroke}{rgb}{0.411765,0.411765,0.411765}%
\pgfsetstrokecolor{currentstroke}%
\pgfsetdash{}{0pt}%
\pgfsys@defobject{currentmarker}{\pgfqpoint{0.000000in}{-0.066667in}}{\pgfqpoint{0.000000in}{0.000000in}}{%
\pgfpathmoveto{\pgfqpoint{0.000000in}{0.000000in}}%
\pgfpathlineto{\pgfqpoint{0.000000in}{-0.066667in}}%
\pgfusepath{stroke,fill}%
}%
\begin{pgfscope}%
\pgfsys@transformshift{3.768841in}{0.562153in}%
\pgfsys@useobject{currentmarker}{}%
\end{pgfscope}%
\end{pgfscope}%
\begin{pgfscope}%
\definecolor{textcolor}{rgb}{0.411765,0.411765,0.411765}%
\pgfsetstrokecolor{textcolor}%
\pgfsetfillcolor{textcolor}%
\pgftext[x=3.768841in,y=0.446875in,,top]{\color{textcolor}\rmfamily\fontsize{8.800000}{10.560000}\selectfont \(\displaystyle 49.5\)}%
\end{pgfscope}%
\begin{pgfscope}%
\pgfpathrectangle{\pgfqpoint{0.636577in}{0.562153in}}{\pgfqpoint{4.171423in}{2.103873in}}%
\pgfusepath{clip}%
\pgfsetroundcap%
\pgfsetroundjoin%
\pgfsetlinewidth{0.803000pt}%
\definecolor{currentstroke}{rgb}{1.000000,1.000000,1.000000}%
\pgfsetstrokecolor{currentstroke}%
\pgfsetdash{}{0pt}%
\pgfpathmoveto{\pgfqpoint{4.396773in}{0.562153in}}%
\pgfpathlineto{\pgfqpoint{4.396773in}{2.666026in}}%
\pgfusepath{stroke}%
\end{pgfscope}%
\begin{pgfscope}%
\pgfsetbuttcap%
\pgfsetroundjoin%
\definecolor{currentfill}{rgb}{0.411765,0.411765,0.411765}%
\pgfsetfillcolor{currentfill}%
\pgfsetlinewidth{1.003750pt}%
\definecolor{currentstroke}{rgb}{0.411765,0.411765,0.411765}%
\pgfsetstrokecolor{currentstroke}%
\pgfsetdash{}{0pt}%
\pgfsys@defobject{currentmarker}{\pgfqpoint{0.000000in}{-0.066667in}}{\pgfqpoint{0.000000in}{0.000000in}}{%
\pgfpathmoveto{\pgfqpoint{0.000000in}{0.000000in}}%
\pgfpathlineto{\pgfqpoint{0.000000in}{-0.066667in}}%
\pgfusepath{stroke,fill}%
}%
\begin{pgfscope}%
\pgfsys@transformshift{4.396773in}{0.562153in}%
\pgfsys@useobject{currentmarker}{}%
\end{pgfscope}%
\end{pgfscope}%
\begin{pgfscope}%
\definecolor{textcolor}{rgb}{0.411765,0.411765,0.411765}%
\pgfsetstrokecolor{textcolor}%
\pgfsetfillcolor{textcolor}%
\pgftext[x=4.396773in,y=0.446875in,,top]{\color{textcolor}\rmfamily\fontsize{8.800000}{10.560000}\selectfont \(\displaystyle 50.0\)}%
\end{pgfscope}%
\begin{pgfscope}%
\pgfsetbuttcap%
\pgfsetroundjoin%
\definecolor{currentfill}{rgb}{0.411765,0.411765,0.411765}%
\pgfsetfillcolor{currentfill}%
\pgfsetlinewidth{0.803000pt}%
\definecolor{currentstroke}{rgb}{0.411765,0.411765,0.411765}%
\pgfsetstrokecolor{currentstroke}%
\pgfsetdash{}{0pt}%
\pgfsys@defobject{currentmarker}{\pgfqpoint{0.000000in}{-0.044444in}}{\pgfqpoint{0.000000in}{0.000000in}}{%
\pgfpathmoveto{\pgfqpoint{0.000000in}{0.000000in}}%
\pgfpathlineto{\pgfqpoint{0.000000in}{-0.044444in}}%
\pgfusepath{stroke,fill}%
}%
\begin{pgfscope}%
\pgfsys@transformshift{0.754769in}{0.562153in}%
\pgfsys@useobject{currentmarker}{}%
\end{pgfscope}%
\end{pgfscope}%
\begin{pgfscope}%
\pgfsetbuttcap%
\pgfsetroundjoin%
\definecolor{currentfill}{rgb}{0.411765,0.411765,0.411765}%
\pgfsetfillcolor{currentfill}%
\pgfsetlinewidth{0.803000pt}%
\definecolor{currentstroke}{rgb}{0.411765,0.411765,0.411765}%
\pgfsetstrokecolor{currentstroke}%
\pgfsetdash{}{0pt}%
\pgfsys@defobject{currentmarker}{\pgfqpoint{0.000000in}{-0.044444in}}{\pgfqpoint{0.000000in}{0.000000in}}{%
\pgfpathmoveto{\pgfqpoint{0.000000in}{0.000000in}}%
\pgfpathlineto{\pgfqpoint{0.000000in}{-0.044444in}}%
\pgfusepath{stroke,fill}%
}%
\begin{pgfscope}%
\pgfsys@transformshift{0.880356in}{0.562153in}%
\pgfsys@useobject{currentmarker}{}%
\end{pgfscope}%
\end{pgfscope}%
\begin{pgfscope}%
\pgfsetbuttcap%
\pgfsetroundjoin%
\definecolor{currentfill}{rgb}{0.411765,0.411765,0.411765}%
\pgfsetfillcolor{currentfill}%
\pgfsetlinewidth{0.803000pt}%
\definecolor{currentstroke}{rgb}{0.411765,0.411765,0.411765}%
\pgfsetstrokecolor{currentstroke}%
\pgfsetdash{}{0pt}%
\pgfsys@defobject{currentmarker}{\pgfqpoint{0.000000in}{-0.044444in}}{\pgfqpoint{0.000000in}{0.000000in}}{%
\pgfpathmoveto{\pgfqpoint{0.000000in}{0.000000in}}%
\pgfpathlineto{\pgfqpoint{0.000000in}{-0.044444in}}%
\pgfusepath{stroke,fill}%
}%
\begin{pgfscope}%
\pgfsys@transformshift{1.005942in}{0.562153in}%
\pgfsys@useobject{currentmarker}{}%
\end{pgfscope}%
\end{pgfscope}%
\begin{pgfscope}%
\pgfsetbuttcap%
\pgfsetroundjoin%
\definecolor{currentfill}{rgb}{0.411765,0.411765,0.411765}%
\pgfsetfillcolor{currentfill}%
\pgfsetlinewidth{0.803000pt}%
\definecolor{currentstroke}{rgb}{0.411765,0.411765,0.411765}%
\pgfsetstrokecolor{currentstroke}%
\pgfsetdash{}{0pt}%
\pgfsys@defobject{currentmarker}{\pgfqpoint{0.000000in}{-0.044444in}}{\pgfqpoint{0.000000in}{0.000000in}}{%
\pgfpathmoveto{\pgfqpoint{0.000000in}{0.000000in}}%
\pgfpathlineto{\pgfqpoint{0.000000in}{-0.044444in}}%
\pgfusepath{stroke,fill}%
}%
\begin{pgfscope}%
\pgfsys@transformshift{1.131528in}{0.562153in}%
\pgfsys@useobject{currentmarker}{}%
\end{pgfscope}%
\end{pgfscope}%
\begin{pgfscope}%
\pgfsetbuttcap%
\pgfsetroundjoin%
\definecolor{currentfill}{rgb}{0.411765,0.411765,0.411765}%
\pgfsetfillcolor{currentfill}%
\pgfsetlinewidth{0.803000pt}%
\definecolor{currentstroke}{rgb}{0.411765,0.411765,0.411765}%
\pgfsetstrokecolor{currentstroke}%
\pgfsetdash{}{0pt}%
\pgfsys@defobject{currentmarker}{\pgfqpoint{0.000000in}{-0.044444in}}{\pgfqpoint{0.000000in}{0.000000in}}{%
\pgfpathmoveto{\pgfqpoint{0.000000in}{0.000000in}}%
\pgfpathlineto{\pgfqpoint{0.000000in}{-0.044444in}}%
\pgfusepath{stroke,fill}%
}%
\begin{pgfscope}%
\pgfsys@transformshift{1.382701in}{0.562153in}%
\pgfsys@useobject{currentmarker}{}%
\end{pgfscope}%
\end{pgfscope}%
\begin{pgfscope}%
\pgfsetbuttcap%
\pgfsetroundjoin%
\definecolor{currentfill}{rgb}{0.411765,0.411765,0.411765}%
\pgfsetfillcolor{currentfill}%
\pgfsetlinewidth{0.803000pt}%
\definecolor{currentstroke}{rgb}{0.411765,0.411765,0.411765}%
\pgfsetstrokecolor{currentstroke}%
\pgfsetdash{}{0pt}%
\pgfsys@defobject{currentmarker}{\pgfqpoint{0.000000in}{-0.044444in}}{\pgfqpoint{0.000000in}{0.000000in}}{%
\pgfpathmoveto{\pgfqpoint{0.000000in}{0.000000in}}%
\pgfpathlineto{\pgfqpoint{0.000000in}{-0.044444in}}%
\pgfusepath{stroke,fill}%
}%
\begin{pgfscope}%
\pgfsys@transformshift{1.508287in}{0.562153in}%
\pgfsys@useobject{currentmarker}{}%
\end{pgfscope}%
\end{pgfscope}%
\begin{pgfscope}%
\pgfsetbuttcap%
\pgfsetroundjoin%
\definecolor{currentfill}{rgb}{0.411765,0.411765,0.411765}%
\pgfsetfillcolor{currentfill}%
\pgfsetlinewidth{0.803000pt}%
\definecolor{currentstroke}{rgb}{0.411765,0.411765,0.411765}%
\pgfsetstrokecolor{currentstroke}%
\pgfsetdash{}{0pt}%
\pgfsys@defobject{currentmarker}{\pgfqpoint{0.000000in}{-0.044444in}}{\pgfqpoint{0.000000in}{0.000000in}}{%
\pgfpathmoveto{\pgfqpoint{0.000000in}{0.000000in}}%
\pgfpathlineto{\pgfqpoint{0.000000in}{-0.044444in}}%
\pgfusepath{stroke,fill}%
}%
\begin{pgfscope}%
\pgfsys@transformshift{1.633874in}{0.562153in}%
\pgfsys@useobject{currentmarker}{}%
\end{pgfscope}%
\end{pgfscope}%
\begin{pgfscope}%
\pgfsetbuttcap%
\pgfsetroundjoin%
\definecolor{currentfill}{rgb}{0.411765,0.411765,0.411765}%
\pgfsetfillcolor{currentfill}%
\pgfsetlinewidth{0.803000pt}%
\definecolor{currentstroke}{rgb}{0.411765,0.411765,0.411765}%
\pgfsetstrokecolor{currentstroke}%
\pgfsetdash{}{0pt}%
\pgfsys@defobject{currentmarker}{\pgfqpoint{0.000000in}{-0.044444in}}{\pgfqpoint{0.000000in}{0.000000in}}{%
\pgfpathmoveto{\pgfqpoint{0.000000in}{0.000000in}}%
\pgfpathlineto{\pgfqpoint{0.000000in}{-0.044444in}}%
\pgfusepath{stroke,fill}%
}%
\begin{pgfscope}%
\pgfsys@transformshift{1.759460in}{0.562153in}%
\pgfsys@useobject{currentmarker}{}%
\end{pgfscope}%
\end{pgfscope}%
\begin{pgfscope}%
\pgfsetbuttcap%
\pgfsetroundjoin%
\definecolor{currentfill}{rgb}{0.411765,0.411765,0.411765}%
\pgfsetfillcolor{currentfill}%
\pgfsetlinewidth{0.803000pt}%
\definecolor{currentstroke}{rgb}{0.411765,0.411765,0.411765}%
\pgfsetstrokecolor{currentstroke}%
\pgfsetdash{}{0pt}%
\pgfsys@defobject{currentmarker}{\pgfqpoint{0.000000in}{-0.044444in}}{\pgfqpoint{0.000000in}{0.000000in}}{%
\pgfpathmoveto{\pgfqpoint{0.000000in}{0.000000in}}%
\pgfpathlineto{\pgfqpoint{0.000000in}{-0.044444in}}%
\pgfusepath{stroke,fill}%
}%
\begin{pgfscope}%
\pgfsys@transformshift{2.010633in}{0.562153in}%
\pgfsys@useobject{currentmarker}{}%
\end{pgfscope}%
\end{pgfscope}%
\begin{pgfscope}%
\pgfsetbuttcap%
\pgfsetroundjoin%
\definecolor{currentfill}{rgb}{0.411765,0.411765,0.411765}%
\pgfsetfillcolor{currentfill}%
\pgfsetlinewidth{0.803000pt}%
\definecolor{currentstroke}{rgb}{0.411765,0.411765,0.411765}%
\pgfsetstrokecolor{currentstroke}%
\pgfsetdash{}{0pt}%
\pgfsys@defobject{currentmarker}{\pgfqpoint{0.000000in}{-0.044444in}}{\pgfqpoint{0.000000in}{0.000000in}}{%
\pgfpathmoveto{\pgfqpoint{0.000000in}{0.000000in}}%
\pgfpathlineto{\pgfqpoint{0.000000in}{-0.044444in}}%
\pgfusepath{stroke,fill}%
}%
\begin{pgfscope}%
\pgfsys@transformshift{2.136219in}{0.562153in}%
\pgfsys@useobject{currentmarker}{}%
\end{pgfscope}%
\end{pgfscope}%
\begin{pgfscope}%
\pgfsetbuttcap%
\pgfsetroundjoin%
\definecolor{currentfill}{rgb}{0.411765,0.411765,0.411765}%
\pgfsetfillcolor{currentfill}%
\pgfsetlinewidth{0.803000pt}%
\definecolor{currentstroke}{rgb}{0.411765,0.411765,0.411765}%
\pgfsetstrokecolor{currentstroke}%
\pgfsetdash{}{0pt}%
\pgfsys@defobject{currentmarker}{\pgfqpoint{0.000000in}{-0.044444in}}{\pgfqpoint{0.000000in}{0.000000in}}{%
\pgfpathmoveto{\pgfqpoint{0.000000in}{0.000000in}}%
\pgfpathlineto{\pgfqpoint{0.000000in}{-0.044444in}}%
\pgfusepath{stroke,fill}%
}%
\begin{pgfscope}%
\pgfsys@transformshift{2.261805in}{0.562153in}%
\pgfsys@useobject{currentmarker}{}%
\end{pgfscope}%
\end{pgfscope}%
\begin{pgfscope}%
\pgfsetbuttcap%
\pgfsetroundjoin%
\definecolor{currentfill}{rgb}{0.411765,0.411765,0.411765}%
\pgfsetfillcolor{currentfill}%
\pgfsetlinewidth{0.803000pt}%
\definecolor{currentstroke}{rgb}{0.411765,0.411765,0.411765}%
\pgfsetstrokecolor{currentstroke}%
\pgfsetdash{}{0pt}%
\pgfsys@defobject{currentmarker}{\pgfqpoint{0.000000in}{-0.044444in}}{\pgfqpoint{0.000000in}{0.000000in}}{%
\pgfpathmoveto{\pgfqpoint{0.000000in}{0.000000in}}%
\pgfpathlineto{\pgfqpoint{0.000000in}{-0.044444in}}%
\pgfusepath{stroke,fill}%
}%
\begin{pgfscope}%
\pgfsys@transformshift{2.387392in}{0.562153in}%
\pgfsys@useobject{currentmarker}{}%
\end{pgfscope}%
\end{pgfscope}%
\begin{pgfscope}%
\pgfsetbuttcap%
\pgfsetroundjoin%
\definecolor{currentfill}{rgb}{0.411765,0.411765,0.411765}%
\pgfsetfillcolor{currentfill}%
\pgfsetlinewidth{0.803000pt}%
\definecolor{currentstroke}{rgb}{0.411765,0.411765,0.411765}%
\pgfsetstrokecolor{currentstroke}%
\pgfsetdash{}{0pt}%
\pgfsys@defobject{currentmarker}{\pgfqpoint{0.000000in}{-0.044444in}}{\pgfqpoint{0.000000in}{0.000000in}}{%
\pgfpathmoveto{\pgfqpoint{0.000000in}{0.000000in}}%
\pgfpathlineto{\pgfqpoint{0.000000in}{-0.044444in}}%
\pgfusepath{stroke,fill}%
}%
\begin{pgfscope}%
\pgfsys@transformshift{2.638564in}{0.562153in}%
\pgfsys@useobject{currentmarker}{}%
\end{pgfscope}%
\end{pgfscope}%
\begin{pgfscope}%
\pgfsetbuttcap%
\pgfsetroundjoin%
\definecolor{currentfill}{rgb}{0.411765,0.411765,0.411765}%
\pgfsetfillcolor{currentfill}%
\pgfsetlinewidth{0.803000pt}%
\definecolor{currentstroke}{rgb}{0.411765,0.411765,0.411765}%
\pgfsetstrokecolor{currentstroke}%
\pgfsetdash{}{0pt}%
\pgfsys@defobject{currentmarker}{\pgfqpoint{0.000000in}{-0.044444in}}{\pgfqpoint{0.000000in}{0.000000in}}{%
\pgfpathmoveto{\pgfqpoint{0.000000in}{0.000000in}}%
\pgfpathlineto{\pgfqpoint{0.000000in}{-0.044444in}}%
\pgfusepath{stroke,fill}%
}%
\begin{pgfscope}%
\pgfsys@transformshift{2.764151in}{0.562153in}%
\pgfsys@useobject{currentmarker}{}%
\end{pgfscope}%
\end{pgfscope}%
\begin{pgfscope}%
\pgfsetbuttcap%
\pgfsetroundjoin%
\definecolor{currentfill}{rgb}{0.411765,0.411765,0.411765}%
\pgfsetfillcolor{currentfill}%
\pgfsetlinewidth{0.803000pt}%
\definecolor{currentstroke}{rgb}{0.411765,0.411765,0.411765}%
\pgfsetstrokecolor{currentstroke}%
\pgfsetdash{}{0pt}%
\pgfsys@defobject{currentmarker}{\pgfqpoint{0.000000in}{-0.044444in}}{\pgfqpoint{0.000000in}{0.000000in}}{%
\pgfpathmoveto{\pgfqpoint{0.000000in}{0.000000in}}%
\pgfpathlineto{\pgfqpoint{0.000000in}{-0.044444in}}%
\pgfusepath{stroke,fill}%
}%
\begin{pgfscope}%
\pgfsys@transformshift{2.889737in}{0.562153in}%
\pgfsys@useobject{currentmarker}{}%
\end{pgfscope}%
\end{pgfscope}%
\begin{pgfscope}%
\pgfsetbuttcap%
\pgfsetroundjoin%
\definecolor{currentfill}{rgb}{0.411765,0.411765,0.411765}%
\pgfsetfillcolor{currentfill}%
\pgfsetlinewidth{0.803000pt}%
\definecolor{currentstroke}{rgb}{0.411765,0.411765,0.411765}%
\pgfsetstrokecolor{currentstroke}%
\pgfsetdash{}{0pt}%
\pgfsys@defobject{currentmarker}{\pgfqpoint{0.000000in}{-0.044444in}}{\pgfqpoint{0.000000in}{0.000000in}}{%
\pgfpathmoveto{\pgfqpoint{0.000000in}{0.000000in}}%
\pgfpathlineto{\pgfqpoint{0.000000in}{-0.044444in}}%
\pgfusepath{stroke,fill}%
}%
\begin{pgfscope}%
\pgfsys@transformshift{3.015323in}{0.562153in}%
\pgfsys@useobject{currentmarker}{}%
\end{pgfscope}%
\end{pgfscope}%
\begin{pgfscope}%
\pgfsetbuttcap%
\pgfsetroundjoin%
\definecolor{currentfill}{rgb}{0.411765,0.411765,0.411765}%
\pgfsetfillcolor{currentfill}%
\pgfsetlinewidth{0.803000pt}%
\definecolor{currentstroke}{rgb}{0.411765,0.411765,0.411765}%
\pgfsetstrokecolor{currentstroke}%
\pgfsetdash{}{0pt}%
\pgfsys@defobject{currentmarker}{\pgfqpoint{0.000000in}{-0.044444in}}{\pgfqpoint{0.000000in}{0.000000in}}{%
\pgfpathmoveto{\pgfqpoint{0.000000in}{0.000000in}}%
\pgfpathlineto{\pgfqpoint{0.000000in}{-0.044444in}}%
\pgfusepath{stroke,fill}%
}%
\begin{pgfscope}%
\pgfsys@transformshift{3.266496in}{0.562153in}%
\pgfsys@useobject{currentmarker}{}%
\end{pgfscope}%
\end{pgfscope}%
\begin{pgfscope}%
\pgfsetbuttcap%
\pgfsetroundjoin%
\definecolor{currentfill}{rgb}{0.411765,0.411765,0.411765}%
\pgfsetfillcolor{currentfill}%
\pgfsetlinewidth{0.803000pt}%
\definecolor{currentstroke}{rgb}{0.411765,0.411765,0.411765}%
\pgfsetstrokecolor{currentstroke}%
\pgfsetdash{}{0pt}%
\pgfsys@defobject{currentmarker}{\pgfqpoint{0.000000in}{-0.044444in}}{\pgfqpoint{0.000000in}{0.000000in}}{%
\pgfpathmoveto{\pgfqpoint{0.000000in}{0.000000in}}%
\pgfpathlineto{\pgfqpoint{0.000000in}{-0.044444in}}%
\pgfusepath{stroke,fill}%
}%
\begin{pgfscope}%
\pgfsys@transformshift{3.392082in}{0.562153in}%
\pgfsys@useobject{currentmarker}{}%
\end{pgfscope}%
\end{pgfscope}%
\begin{pgfscope}%
\pgfsetbuttcap%
\pgfsetroundjoin%
\definecolor{currentfill}{rgb}{0.411765,0.411765,0.411765}%
\pgfsetfillcolor{currentfill}%
\pgfsetlinewidth{0.803000pt}%
\definecolor{currentstroke}{rgb}{0.411765,0.411765,0.411765}%
\pgfsetstrokecolor{currentstroke}%
\pgfsetdash{}{0pt}%
\pgfsys@defobject{currentmarker}{\pgfqpoint{0.000000in}{-0.044444in}}{\pgfqpoint{0.000000in}{0.000000in}}{%
\pgfpathmoveto{\pgfqpoint{0.000000in}{0.000000in}}%
\pgfpathlineto{\pgfqpoint{0.000000in}{-0.044444in}}%
\pgfusepath{stroke,fill}%
}%
\begin{pgfscope}%
\pgfsys@transformshift{3.517668in}{0.562153in}%
\pgfsys@useobject{currentmarker}{}%
\end{pgfscope}%
\end{pgfscope}%
\begin{pgfscope}%
\pgfsetbuttcap%
\pgfsetroundjoin%
\definecolor{currentfill}{rgb}{0.411765,0.411765,0.411765}%
\pgfsetfillcolor{currentfill}%
\pgfsetlinewidth{0.803000pt}%
\definecolor{currentstroke}{rgb}{0.411765,0.411765,0.411765}%
\pgfsetstrokecolor{currentstroke}%
\pgfsetdash{}{0pt}%
\pgfsys@defobject{currentmarker}{\pgfqpoint{0.000000in}{-0.044444in}}{\pgfqpoint{0.000000in}{0.000000in}}{%
\pgfpathmoveto{\pgfqpoint{0.000000in}{0.000000in}}%
\pgfpathlineto{\pgfqpoint{0.000000in}{-0.044444in}}%
\pgfusepath{stroke,fill}%
}%
\begin{pgfscope}%
\pgfsys@transformshift{3.643255in}{0.562153in}%
\pgfsys@useobject{currentmarker}{}%
\end{pgfscope}%
\end{pgfscope}%
\begin{pgfscope}%
\pgfsetbuttcap%
\pgfsetroundjoin%
\definecolor{currentfill}{rgb}{0.411765,0.411765,0.411765}%
\pgfsetfillcolor{currentfill}%
\pgfsetlinewidth{0.803000pt}%
\definecolor{currentstroke}{rgb}{0.411765,0.411765,0.411765}%
\pgfsetstrokecolor{currentstroke}%
\pgfsetdash{}{0pt}%
\pgfsys@defobject{currentmarker}{\pgfqpoint{0.000000in}{-0.044444in}}{\pgfqpoint{0.000000in}{0.000000in}}{%
\pgfpathmoveto{\pgfqpoint{0.000000in}{0.000000in}}%
\pgfpathlineto{\pgfqpoint{0.000000in}{-0.044444in}}%
\pgfusepath{stroke,fill}%
}%
\begin{pgfscope}%
\pgfsys@transformshift{3.894427in}{0.562153in}%
\pgfsys@useobject{currentmarker}{}%
\end{pgfscope}%
\end{pgfscope}%
\begin{pgfscope}%
\pgfsetbuttcap%
\pgfsetroundjoin%
\definecolor{currentfill}{rgb}{0.411765,0.411765,0.411765}%
\pgfsetfillcolor{currentfill}%
\pgfsetlinewidth{0.803000pt}%
\definecolor{currentstroke}{rgb}{0.411765,0.411765,0.411765}%
\pgfsetstrokecolor{currentstroke}%
\pgfsetdash{}{0pt}%
\pgfsys@defobject{currentmarker}{\pgfqpoint{0.000000in}{-0.044444in}}{\pgfqpoint{0.000000in}{0.000000in}}{%
\pgfpathmoveto{\pgfqpoint{0.000000in}{0.000000in}}%
\pgfpathlineto{\pgfqpoint{0.000000in}{-0.044444in}}%
\pgfusepath{stroke,fill}%
}%
\begin{pgfscope}%
\pgfsys@transformshift{4.020014in}{0.562153in}%
\pgfsys@useobject{currentmarker}{}%
\end{pgfscope}%
\end{pgfscope}%
\begin{pgfscope}%
\pgfsetbuttcap%
\pgfsetroundjoin%
\definecolor{currentfill}{rgb}{0.411765,0.411765,0.411765}%
\pgfsetfillcolor{currentfill}%
\pgfsetlinewidth{0.803000pt}%
\definecolor{currentstroke}{rgb}{0.411765,0.411765,0.411765}%
\pgfsetstrokecolor{currentstroke}%
\pgfsetdash{}{0pt}%
\pgfsys@defobject{currentmarker}{\pgfqpoint{0.000000in}{-0.044444in}}{\pgfqpoint{0.000000in}{0.000000in}}{%
\pgfpathmoveto{\pgfqpoint{0.000000in}{0.000000in}}%
\pgfpathlineto{\pgfqpoint{0.000000in}{-0.044444in}}%
\pgfusepath{stroke,fill}%
}%
\begin{pgfscope}%
\pgfsys@transformshift{4.145600in}{0.562153in}%
\pgfsys@useobject{currentmarker}{}%
\end{pgfscope}%
\end{pgfscope}%
\begin{pgfscope}%
\pgfsetbuttcap%
\pgfsetroundjoin%
\definecolor{currentfill}{rgb}{0.411765,0.411765,0.411765}%
\pgfsetfillcolor{currentfill}%
\pgfsetlinewidth{0.803000pt}%
\definecolor{currentstroke}{rgb}{0.411765,0.411765,0.411765}%
\pgfsetstrokecolor{currentstroke}%
\pgfsetdash{}{0pt}%
\pgfsys@defobject{currentmarker}{\pgfqpoint{0.000000in}{-0.044444in}}{\pgfqpoint{0.000000in}{0.000000in}}{%
\pgfpathmoveto{\pgfqpoint{0.000000in}{0.000000in}}%
\pgfpathlineto{\pgfqpoint{0.000000in}{-0.044444in}}%
\pgfusepath{stroke,fill}%
}%
\begin{pgfscope}%
\pgfsys@transformshift{4.271186in}{0.562153in}%
\pgfsys@useobject{currentmarker}{}%
\end{pgfscope}%
\end{pgfscope}%
\begin{pgfscope}%
\pgfsetbuttcap%
\pgfsetroundjoin%
\definecolor{currentfill}{rgb}{0.411765,0.411765,0.411765}%
\pgfsetfillcolor{currentfill}%
\pgfsetlinewidth{0.803000pt}%
\definecolor{currentstroke}{rgb}{0.411765,0.411765,0.411765}%
\pgfsetstrokecolor{currentstroke}%
\pgfsetdash{}{0pt}%
\pgfsys@defobject{currentmarker}{\pgfqpoint{0.000000in}{-0.044444in}}{\pgfqpoint{0.000000in}{0.000000in}}{%
\pgfpathmoveto{\pgfqpoint{0.000000in}{0.000000in}}%
\pgfpathlineto{\pgfqpoint{0.000000in}{-0.044444in}}%
\pgfusepath{stroke,fill}%
}%
\begin{pgfscope}%
\pgfsys@transformshift{4.522359in}{0.562153in}%
\pgfsys@useobject{currentmarker}{}%
\end{pgfscope}%
\end{pgfscope}%
\begin{pgfscope}%
\pgfsetbuttcap%
\pgfsetroundjoin%
\definecolor{currentfill}{rgb}{0.411765,0.411765,0.411765}%
\pgfsetfillcolor{currentfill}%
\pgfsetlinewidth{0.803000pt}%
\definecolor{currentstroke}{rgb}{0.411765,0.411765,0.411765}%
\pgfsetstrokecolor{currentstroke}%
\pgfsetdash{}{0pt}%
\pgfsys@defobject{currentmarker}{\pgfqpoint{0.000000in}{-0.044444in}}{\pgfqpoint{0.000000in}{0.000000in}}{%
\pgfpathmoveto{\pgfqpoint{0.000000in}{0.000000in}}%
\pgfpathlineto{\pgfqpoint{0.000000in}{-0.044444in}}%
\pgfusepath{stroke,fill}%
}%
\begin{pgfscope}%
\pgfsys@transformshift{4.647945in}{0.562153in}%
\pgfsys@useobject{currentmarker}{}%
\end{pgfscope}%
\end{pgfscope}%
\begin{pgfscope}%
\pgfsetbuttcap%
\pgfsetroundjoin%
\definecolor{currentfill}{rgb}{0.411765,0.411765,0.411765}%
\pgfsetfillcolor{currentfill}%
\pgfsetlinewidth{0.803000pt}%
\definecolor{currentstroke}{rgb}{0.411765,0.411765,0.411765}%
\pgfsetstrokecolor{currentstroke}%
\pgfsetdash{}{0pt}%
\pgfsys@defobject{currentmarker}{\pgfqpoint{0.000000in}{-0.044444in}}{\pgfqpoint{0.000000in}{0.000000in}}{%
\pgfpathmoveto{\pgfqpoint{0.000000in}{0.000000in}}%
\pgfpathlineto{\pgfqpoint{0.000000in}{-0.044444in}}%
\pgfusepath{stroke,fill}%
}%
\begin{pgfscope}%
\pgfsys@transformshift{4.773532in}{0.562153in}%
\pgfsys@useobject{currentmarker}{}%
\end{pgfscope}%
\end{pgfscope}%
\begin{pgfscope}%
\definecolor{textcolor}{rgb}{0.000000,0.000000,0.000000}%
\pgfsetstrokecolor{textcolor}%
\pgfsetfillcolor{textcolor}%
\pgftext[x=2.722288in,y=0.273036in,,top]{\color{textcolor}\rmfamily\fontsize{9.600000}{11.520000}\selectfont Desvio mínimo [\degree]}%
\end{pgfscope}%
\begin{pgfscope}%
\pgfpathrectangle{\pgfqpoint{0.636577in}{0.562153in}}{\pgfqpoint{4.171423in}{2.103873in}}%
\pgfusepath{clip}%
\pgfsetroundcap%
\pgfsetroundjoin%
\pgfsetlinewidth{0.803000pt}%
\definecolor{currentstroke}{rgb}{1.000000,1.000000,1.000000}%
\pgfsetstrokecolor{currentstroke}%
\pgfsetdash{}{0pt}%
\pgfpathmoveto{\pgfqpoint{0.636577in}{0.580761in}}%
\pgfpathlineto{\pgfqpoint{4.808000in}{0.580761in}}%
\pgfusepath{stroke}%
\end{pgfscope}%
\begin{pgfscope}%
\pgfsetbuttcap%
\pgfsetroundjoin%
\definecolor{currentfill}{rgb}{0.411765,0.411765,0.411765}%
\pgfsetfillcolor{currentfill}%
\pgfsetlinewidth{1.003750pt}%
\definecolor{currentstroke}{rgb}{0.411765,0.411765,0.411765}%
\pgfsetstrokecolor{currentstroke}%
\pgfsetdash{}{0pt}%
\pgfsys@defobject{currentmarker}{\pgfqpoint{-0.066667in}{0.000000in}}{\pgfqpoint{0.000000in}{0.000000in}}{%
\pgfpathmoveto{\pgfqpoint{0.000000in}{0.000000in}}%
\pgfpathlineto{\pgfqpoint{-0.066667in}{0.000000in}}%
\pgfusepath{stroke,fill}%
}%
\begin{pgfscope}%
\pgfsys@transformshift{0.636577in}{0.580761in}%
\pgfsys@useobject{currentmarker}{}%
\end{pgfscope}%
\end{pgfscope}%
\begin{pgfscope}%
\definecolor{textcolor}{rgb}{0.411765,0.411765,0.411765}%
\pgfsetstrokecolor{textcolor}%
\pgfsetfillcolor{textcolor}%
\pgftext[x=0.328592in,y=0.534331in,left,base]{\color{textcolor}\rmfamily\fontsize{8.800000}{10.560000}\selectfont \(\displaystyle 400\)}%
\end{pgfscope}%
\begin{pgfscope}%
\pgfpathrectangle{\pgfqpoint{0.636577in}{0.562153in}}{\pgfqpoint{4.171423in}{2.103873in}}%
\pgfusepath{clip}%
\pgfsetroundcap%
\pgfsetroundjoin%
\pgfsetlinewidth{0.803000pt}%
\definecolor{currentstroke}{rgb}{1.000000,1.000000,1.000000}%
\pgfsetstrokecolor{currentstroke}%
\pgfsetdash{}{0pt}%
\pgfpathmoveto{\pgfqpoint{0.636577in}{0.865046in}}%
\pgfpathlineto{\pgfqpoint{4.808000in}{0.865046in}}%
\pgfusepath{stroke}%
\end{pgfscope}%
\begin{pgfscope}%
\pgfsetbuttcap%
\pgfsetroundjoin%
\definecolor{currentfill}{rgb}{0.411765,0.411765,0.411765}%
\pgfsetfillcolor{currentfill}%
\pgfsetlinewidth{1.003750pt}%
\definecolor{currentstroke}{rgb}{0.411765,0.411765,0.411765}%
\pgfsetstrokecolor{currentstroke}%
\pgfsetdash{}{0pt}%
\pgfsys@defobject{currentmarker}{\pgfqpoint{-0.066667in}{0.000000in}}{\pgfqpoint{0.000000in}{0.000000in}}{%
\pgfpathmoveto{\pgfqpoint{0.000000in}{0.000000in}}%
\pgfpathlineto{\pgfqpoint{-0.066667in}{0.000000in}}%
\pgfusepath{stroke,fill}%
}%
\begin{pgfscope}%
\pgfsys@transformshift{0.636577in}{0.865046in}%
\pgfsys@useobject{currentmarker}{}%
\end{pgfscope}%
\end{pgfscope}%
\begin{pgfscope}%
\definecolor{textcolor}{rgb}{0.411765,0.411765,0.411765}%
\pgfsetstrokecolor{textcolor}%
\pgfsetfillcolor{textcolor}%
\pgftext[x=0.328592in,y=0.818616in,left,base]{\color{textcolor}\rmfamily\fontsize{8.800000}{10.560000}\selectfont \(\displaystyle 450\)}%
\end{pgfscope}%
\begin{pgfscope}%
\pgfpathrectangle{\pgfqpoint{0.636577in}{0.562153in}}{\pgfqpoint{4.171423in}{2.103873in}}%
\pgfusepath{clip}%
\pgfsetroundcap%
\pgfsetroundjoin%
\pgfsetlinewidth{0.803000pt}%
\definecolor{currentstroke}{rgb}{1.000000,1.000000,1.000000}%
\pgfsetstrokecolor{currentstroke}%
\pgfsetdash{}{0pt}%
\pgfpathmoveto{\pgfqpoint{0.636577in}{1.149331in}}%
\pgfpathlineto{\pgfqpoint{4.808000in}{1.149331in}}%
\pgfusepath{stroke}%
\end{pgfscope}%
\begin{pgfscope}%
\pgfsetbuttcap%
\pgfsetroundjoin%
\definecolor{currentfill}{rgb}{0.411765,0.411765,0.411765}%
\pgfsetfillcolor{currentfill}%
\pgfsetlinewidth{1.003750pt}%
\definecolor{currentstroke}{rgb}{0.411765,0.411765,0.411765}%
\pgfsetstrokecolor{currentstroke}%
\pgfsetdash{}{0pt}%
\pgfsys@defobject{currentmarker}{\pgfqpoint{-0.066667in}{0.000000in}}{\pgfqpoint{0.000000in}{0.000000in}}{%
\pgfpathmoveto{\pgfqpoint{0.000000in}{0.000000in}}%
\pgfpathlineto{\pgfqpoint{-0.066667in}{0.000000in}}%
\pgfusepath{stroke,fill}%
}%
\begin{pgfscope}%
\pgfsys@transformshift{0.636577in}{1.149331in}%
\pgfsys@useobject{currentmarker}{}%
\end{pgfscope}%
\end{pgfscope}%
\begin{pgfscope}%
\definecolor{textcolor}{rgb}{0.411765,0.411765,0.411765}%
\pgfsetstrokecolor{textcolor}%
\pgfsetfillcolor{textcolor}%
\pgftext[x=0.328592in,y=1.102901in,left,base]{\color{textcolor}\rmfamily\fontsize{8.800000}{10.560000}\selectfont \(\displaystyle 500\)}%
\end{pgfscope}%
\begin{pgfscope}%
\pgfpathrectangle{\pgfqpoint{0.636577in}{0.562153in}}{\pgfqpoint{4.171423in}{2.103873in}}%
\pgfusepath{clip}%
\pgfsetroundcap%
\pgfsetroundjoin%
\pgfsetlinewidth{0.803000pt}%
\definecolor{currentstroke}{rgb}{1.000000,1.000000,1.000000}%
\pgfsetstrokecolor{currentstroke}%
\pgfsetdash{}{0pt}%
\pgfpathmoveto{\pgfqpoint{0.636577in}{1.433616in}}%
\pgfpathlineto{\pgfqpoint{4.808000in}{1.433616in}}%
\pgfusepath{stroke}%
\end{pgfscope}%
\begin{pgfscope}%
\pgfsetbuttcap%
\pgfsetroundjoin%
\definecolor{currentfill}{rgb}{0.411765,0.411765,0.411765}%
\pgfsetfillcolor{currentfill}%
\pgfsetlinewidth{1.003750pt}%
\definecolor{currentstroke}{rgb}{0.411765,0.411765,0.411765}%
\pgfsetstrokecolor{currentstroke}%
\pgfsetdash{}{0pt}%
\pgfsys@defobject{currentmarker}{\pgfqpoint{-0.066667in}{0.000000in}}{\pgfqpoint{0.000000in}{0.000000in}}{%
\pgfpathmoveto{\pgfqpoint{0.000000in}{0.000000in}}%
\pgfpathlineto{\pgfqpoint{-0.066667in}{0.000000in}}%
\pgfusepath{stroke,fill}%
}%
\begin{pgfscope}%
\pgfsys@transformshift{0.636577in}{1.433616in}%
\pgfsys@useobject{currentmarker}{}%
\end{pgfscope}%
\end{pgfscope}%
\begin{pgfscope}%
\definecolor{textcolor}{rgb}{0.411765,0.411765,0.411765}%
\pgfsetstrokecolor{textcolor}%
\pgfsetfillcolor{textcolor}%
\pgftext[x=0.328592in,y=1.387186in,left,base]{\color{textcolor}\rmfamily\fontsize{8.800000}{10.560000}\selectfont \(\displaystyle 550\)}%
\end{pgfscope}%
\begin{pgfscope}%
\pgfpathrectangle{\pgfqpoint{0.636577in}{0.562153in}}{\pgfqpoint{4.171423in}{2.103873in}}%
\pgfusepath{clip}%
\pgfsetroundcap%
\pgfsetroundjoin%
\pgfsetlinewidth{0.803000pt}%
\definecolor{currentstroke}{rgb}{1.000000,1.000000,1.000000}%
\pgfsetstrokecolor{currentstroke}%
\pgfsetdash{}{0pt}%
\pgfpathmoveto{\pgfqpoint{0.636577in}{1.717901in}}%
\pgfpathlineto{\pgfqpoint{4.808000in}{1.717901in}}%
\pgfusepath{stroke}%
\end{pgfscope}%
\begin{pgfscope}%
\pgfsetbuttcap%
\pgfsetroundjoin%
\definecolor{currentfill}{rgb}{0.411765,0.411765,0.411765}%
\pgfsetfillcolor{currentfill}%
\pgfsetlinewidth{1.003750pt}%
\definecolor{currentstroke}{rgb}{0.411765,0.411765,0.411765}%
\pgfsetstrokecolor{currentstroke}%
\pgfsetdash{}{0pt}%
\pgfsys@defobject{currentmarker}{\pgfqpoint{-0.066667in}{0.000000in}}{\pgfqpoint{0.000000in}{0.000000in}}{%
\pgfpathmoveto{\pgfqpoint{0.000000in}{0.000000in}}%
\pgfpathlineto{\pgfqpoint{-0.066667in}{0.000000in}}%
\pgfusepath{stroke,fill}%
}%
\begin{pgfscope}%
\pgfsys@transformshift{0.636577in}{1.717901in}%
\pgfsys@useobject{currentmarker}{}%
\end{pgfscope}%
\end{pgfscope}%
\begin{pgfscope}%
\definecolor{textcolor}{rgb}{0.411765,0.411765,0.411765}%
\pgfsetstrokecolor{textcolor}%
\pgfsetfillcolor{textcolor}%
\pgftext[x=0.328592in,y=1.671471in,left,base]{\color{textcolor}\rmfamily\fontsize{8.800000}{10.560000}\selectfont \(\displaystyle 600\)}%
\end{pgfscope}%
\begin{pgfscope}%
\pgfpathrectangle{\pgfqpoint{0.636577in}{0.562153in}}{\pgfqpoint{4.171423in}{2.103873in}}%
\pgfusepath{clip}%
\pgfsetroundcap%
\pgfsetroundjoin%
\pgfsetlinewidth{0.803000pt}%
\definecolor{currentstroke}{rgb}{1.000000,1.000000,1.000000}%
\pgfsetstrokecolor{currentstroke}%
\pgfsetdash{}{0pt}%
\pgfpathmoveto{\pgfqpoint{0.636577in}{2.002186in}}%
\pgfpathlineto{\pgfqpoint{4.808000in}{2.002186in}}%
\pgfusepath{stroke}%
\end{pgfscope}%
\begin{pgfscope}%
\pgfsetbuttcap%
\pgfsetroundjoin%
\definecolor{currentfill}{rgb}{0.411765,0.411765,0.411765}%
\pgfsetfillcolor{currentfill}%
\pgfsetlinewidth{1.003750pt}%
\definecolor{currentstroke}{rgb}{0.411765,0.411765,0.411765}%
\pgfsetstrokecolor{currentstroke}%
\pgfsetdash{}{0pt}%
\pgfsys@defobject{currentmarker}{\pgfqpoint{-0.066667in}{0.000000in}}{\pgfqpoint{0.000000in}{0.000000in}}{%
\pgfpathmoveto{\pgfqpoint{0.000000in}{0.000000in}}%
\pgfpathlineto{\pgfqpoint{-0.066667in}{0.000000in}}%
\pgfusepath{stroke,fill}%
}%
\begin{pgfscope}%
\pgfsys@transformshift{0.636577in}{2.002186in}%
\pgfsys@useobject{currentmarker}{}%
\end{pgfscope}%
\end{pgfscope}%
\begin{pgfscope}%
\definecolor{textcolor}{rgb}{0.411765,0.411765,0.411765}%
\pgfsetstrokecolor{textcolor}%
\pgfsetfillcolor{textcolor}%
\pgftext[x=0.328592in,y=1.955756in,left,base]{\color{textcolor}\rmfamily\fontsize{8.800000}{10.560000}\selectfont \(\displaystyle 650\)}%
\end{pgfscope}%
\begin{pgfscope}%
\pgfpathrectangle{\pgfqpoint{0.636577in}{0.562153in}}{\pgfqpoint{4.171423in}{2.103873in}}%
\pgfusepath{clip}%
\pgfsetroundcap%
\pgfsetroundjoin%
\pgfsetlinewidth{0.803000pt}%
\definecolor{currentstroke}{rgb}{1.000000,1.000000,1.000000}%
\pgfsetstrokecolor{currentstroke}%
\pgfsetdash{}{0pt}%
\pgfpathmoveto{\pgfqpoint{0.636577in}{2.286471in}}%
\pgfpathlineto{\pgfqpoint{4.808000in}{2.286471in}}%
\pgfusepath{stroke}%
\end{pgfscope}%
\begin{pgfscope}%
\pgfsetbuttcap%
\pgfsetroundjoin%
\definecolor{currentfill}{rgb}{0.411765,0.411765,0.411765}%
\pgfsetfillcolor{currentfill}%
\pgfsetlinewidth{1.003750pt}%
\definecolor{currentstroke}{rgb}{0.411765,0.411765,0.411765}%
\pgfsetstrokecolor{currentstroke}%
\pgfsetdash{}{0pt}%
\pgfsys@defobject{currentmarker}{\pgfqpoint{-0.066667in}{0.000000in}}{\pgfqpoint{0.000000in}{0.000000in}}{%
\pgfpathmoveto{\pgfqpoint{0.000000in}{0.000000in}}%
\pgfpathlineto{\pgfqpoint{-0.066667in}{0.000000in}}%
\pgfusepath{stroke,fill}%
}%
\begin{pgfscope}%
\pgfsys@transformshift{0.636577in}{2.286471in}%
\pgfsys@useobject{currentmarker}{}%
\end{pgfscope}%
\end{pgfscope}%
\begin{pgfscope}%
\definecolor{textcolor}{rgb}{0.411765,0.411765,0.411765}%
\pgfsetstrokecolor{textcolor}%
\pgfsetfillcolor{textcolor}%
\pgftext[x=0.328592in,y=2.240041in,left,base]{\color{textcolor}\rmfamily\fontsize{8.800000}{10.560000}\selectfont \(\displaystyle 700\)}%
\end{pgfscope}%
\begin{pgfscope}%
\pgfpathrectangle{\pgfqpoint{0.636577in}{0.562153in}}{\pgfqpoint{4.171423in}{2.103873in}}%
\pgfusepath{clip}%
\pgfsetroundcap%
\pgfsetroundjoin%
\pgfsetlinewidth{0.803000pt}%
\definecolor{currentstroke}{rgb}{1.000000,1.000000,1.000000}%
\pgfsetstrokecolor{currentstroke}%
\pgfsetdash{}{0pt}%
\pgfpathmoveto{\pgfqpoint{0.636577in}{2.570756in}}%
\pgfpathlineto{\pgfqpoint{4.808000in}{2.570756in}}%
\pgfusepath{stroke}%
\end{pgfscope}%
\begin{pgfscope}%
\pgfsetbuttcap%
\pgfsetroundjoin%
\definecolor{currentfill}{rgb}{0.411765,0.411765,0.411765}%
\pgfsetfillcolor{currentfill}%
\pgfsetlinewidth{1.003750pt}%
\definecolor{currentstroke}{rgb}{0.411765,0.411765,0.411765}%
\pgfsetstrokecolor{currentstroke}%
\pgfsetdash{}{0pt}%
\pgfsys@defobject{currentmarker}{\pgfqpoint{-0.066667in}{0.000000in}}{\pgfqpoint{0.000000in}{0.000000in}}{%
\pgfpathmoveto{\pgfqpoint{0.000000in}{0.000000in}}%
\pgfpathlineto{\pgfqpoint{-0.066667in}{0.000000in}}%
\pgfusepath{stroke,fill}%
}%
\begin{pgfscope}%
\pgfsys@transformshift{0.636577in}{2.570756in}%
\pgfsys@useobject{currentmarker}{}%
\end{pgfscope}%
\end{pgfscope}%
\begin{pgfscope}%
\definecolor{textcolor}{rgb}{0.411765,0.411765,0.411765}%
\pgfsetstrokecolor{textcolor}%
\pgfsetfillcolor{textcolor}%
\pgftext[x=0.328592in,y=2.524326in,left,base]{\color{textcolor}\rmfamily\fontsize{8.800000}{10.560000}\selectfont \(\displaystyle 750\)}%
\end{pgfscope}%
\begin{pgfscope}%
\pgfsetbuttcap%
\pgfsetroundjoin%
\definecolor{currentfill}{rgb}{0.411765,0.411765,0.411765}%
\pgfsetfillcolor{currentfill}%
\pgfsetlinewidth{0.803000pt}%
\definecolor{currentstroke}{rgb}{0.411765,0.411765,0.411765}%
\pgfsetstrokecolor{currentstroke}%
\pgfsetdash{}{0pt}%
\pgfsys@defobject{currentmarker}{\pgfqpoint{-0.044444in}{0.000000in}}{\pgfqpoint{0.000000in}{0.000000in}}{%
\pgfpathmoveto{\pgfqpoint{0.000000in}{0.000000in}}%
\pgfpathlineto{\pgfqpoint{-0.044444in}{0.000000in}}%
\pgfusepath{stroke,fill}%
}%
\begin{pgfscope}%
\pgfsys@transformshift{0.636577in}{0.637618in}%
\pgfsys@useobject{currentmarker}{}%
\end{pgfscope}%
\end{pgfscope}%
\begin{pgfscope}%
\pgfsetbuttcap%
\pgfsetroundjoin%
\definecolor{currentfill}{rgb}{0.411765,0.411765,0.411765}%
\pgfsetfillcolor{currentfill}%
\pgfsetlinewidth{0.803000pt}%
\definecolor{currentstroke}{rgb}{0.411765,0.411765,0.411765}%
\pgfsetstrokecolor{currentstroke}%
\pgfsetdash{}{0pt}%
\pgfsys@defobject{currentmarker}{\pgfqpoint{-0.044444in}{0.000000in}}{\pgfqpoint{0.000000in}{0.000000in}}{%
\pgfpathmoveto{\pgfqpoint{0.000000in}{0.000000in}}%
\pgfpathlineto{\pgfqpoint{-0.044444in}{0.000000in}}%
\pgfusepath{stroke,fill}%
}%
\begin{pgfscope}%
\pgfsys@transformshift{0.636577in}{0.694475in}%
\pgfsys@useobject{currentmarker}{}%
\end{pgfscope}%
\end{pgfscope}%
\begin{pgfscope}%
\pgfsetbuttcap%
\pgfsetroundjoin%
\definecolor{currentfill}{rgb}{0.411765,0.411765,0.411765}%
\pgfsetfillcolor{currentfill}%
\pgfsetlinewidth{0.803000pt}%
\definecolor{currentstroke}{rgb}{0.411765,0.411765,0.411765}%
\pgfsetstrokecolor{currentstroke}%
\pgfsetdash{}{0pt}%
\pgfsys@defobject{currentmarker}{\pgfqpoint{-0.044444in}{0.000000in}}{\pgfqpoint{0.000000in}{0.000000in}}{%
\pgfpathmoveto{\pgfqpoint{0.000000in}{0.000000in}}%
\pgfpathlineto{\pgfqpoint{-0.044444in}{0.000000in}}%
\pgfusepath{stroke,fill}%
}%
\begin{pgfscope}%
\pgfsys@transformshift{0.636577in}{0.751332in}%
\pgfsys@useobject{currentmarker}{}%
\end{pgfscope}%
\end{pgfscope}%
\begin{pgfscope}%
\pgfsetbuttcap%
\pgfsetroundjoin%
\definecolor{currentfill}{rgb}{0.411765,0.411765,0.411765}%
\pgfsetfillcolor{currentfill}%
\pgfsetlinewidth{0.803000pt}%
\definecolor{currentstroke}{rgb}{0.411765,0.411765,0.411765}%
\pgfsetstrokecolor{currentstroke}%
\pgfsetdash{}{0pt}%
\pgfsys@defobject{currentmarker}{\pgfqpoint{-0.044444in}{0.000000in}}{\pgfqpoint{0.000000in}{0.000000in}}{%
\pgfpathmoveto{\pgfqpoint{0.000000in}{0.000000in}}%
\pgfpathlineto{\pgfqpoint{-0.044444in}{0.000000in}}%
\pgfusepath{stroke,fill}%
}%
\begin{pgfscope}%
\pgfsys@transformshift{0.636577in}{0.808189in}%
\pgfsys@useobject{currentmarker}{}%
\end{pgfscope}%
\end{pgfscope}%
\begin{pgfscope}%
\pgfsetbuttcap%
\pgfsetroundjoin%
\definecolor{currentfill}{rgb}{0.411765,0.411765,0.411765}%
\pgfsetfillcolor{currentfill}%
\pgfsetlinewidth{0.803000pt}%
\definecolor{currentstroke}{rgb}{0.411765,0.411765,0.411765}%
\pgfsetstrokecolor{currentstroke}%
\pgfsetdash{}{0pt}%
\pgfsys@defobject{currentmarker}{\pgfqpoint{-0.044444in}{0.000000in}}{\pgfqpoint{0.000000in}{0.000000in}}{%
\pgfpathmoveto{\pgfqpoint{0.000000in}{0.000000in}}%
\pgfpathlineto{\pgfqpoint{-0.044444in}{0.000000in}}%
\pgfusepath{stroke,fill}%
}%
\begin{pgfscope}%
\pgfsys@transformshift{0.636577in}{0.921903in}%
\pgfsys@useobject{currentmarker}{}%
\end{pgfscope}%
\end{pgfscope}%
\begin{pgfscope}%
\pgfsetbuttcap%
\pgfsetroundjoin%
\definecolor{currentfill}{rgb}{0.411765,0.411765,0.411765}%
\pgfsetfillcolor{currentfill}%
\pgfsetlinewidth{0.803000pt}%
\definecolor{currentstroke}{rgb}{0.411765,0.411765,0.411765}%
\pgfsetstrokecolor{currentstroke}%
\pgfsetdash{}{0pt}%
\pgfsys@defobject{currentmarker}{\pgfqpoint{-0.044444in}{0.000000in}}{\pgfqpoint{0.000000in}{0.000000in}}{%
\pgfpathmoveto{\pgfqpoint{0.000000in}{0.000000in}}%
\pgfpathlineto{\pgfqpoint{-0.044444in}{0.000000in}}%
\pgfusepath{stroke,fill}%
}%
\begin{pgfscope}%
\pgfsys@transformshift{0.636577in}{0.978760in}%
\pgfsys@useobject{currentmarker}{}%
\end{pgfscope}%
\end{pgfscope}%
\begin{pgfscope}%
\pgfsetbuttcap%
\pgfsetroundjoin%
\definecolor{currentfill}{rgb}{0.411765,0.411765,0.411765}%
\pgfsetfillcolor{currentfill}%
\pgfsetlinewidth{0.803000pt}%
\definecolor{currentstroke}{rgb}{0.411765,0.411765,0.411765}%
\pgfsetstrokecolor{currentstroke}%
\pgfsetdash{}{0pt}%
\pgfsys@defobject{currentmarker}{\pgfqpoint{-0.044444in}{0.000000in}}{\pgfqpoint{0.000000in}{0.000000in}}{%
\pgfpathmoveto{\pgfqpoint{0.000000in}{0.000000in}}%
\pgfpathlineto{\pgfqpoint{-0.044444in}{0.000000in}}%
\pgfusepath{stroke,fill}%
}%
\begin{pgfscope}%
\pgfsys@transformshift{0.636577in}{1.035617in}%
\pgfsys@useobject{currentmarker}{}%
\end{pgfscope}%
\end{pgfscope}%
\begin{pgfscope}%
\pgfsetbuttcap%
\pgfsetroundjoin%
\definecolor{currentfill}{rgb}{0.411765,0.411765,0.411765}%
\pgfsetfillcolor{currentfill}%
\pgfsetlinewidth{0.803000pt}%
\definecolor{currentstroke}{rgb}{0.411765,0.411765,0.411765}%
\pgfsetstrokecolor{currentstroke}%
\pgfsetdash{}{0pt}%
\pgfsys@defobject{currentmarker}{\pgfqpoint{-0.044444in}{0.000000in}}{\pgfqpoint{0.000000in}{0.000000in}}{%
\pgfpathmoveto{\pgfqpoint{0.000000in}{0.000000in}}%
\pgfpathlineto{\pgfqpoint{-0.044444in}{0.000000in}}%
\pgfusepath{stroke,fill}%
}%
\begin{pgfscope}%
\pgfsys@transformshift{0.636577in}{1.092474in}%
\pgfsys@useobject{currentmarker}{}%
\end{pgfscope}%
\end{pgfscope}%
\begin{pgfscope}%
\pgfsetbuttcap%
\pgfsetroundjoin%
\definecolor{currentfill}{rgb}{0.411765,0.411765,0.411765}%
\pgfsetfillcolor{currentfill}%
\pgfsetlinewidth{0.803000pt}%
\definecolor{currentstroke}{rgb}{0.411765,0.411765,0.411765}%
\pgfsetstrokecolor{currentstroke}%
\pgfsetdash{}{0pt}%
\pgfsys@defobject{currentmarker}{\pgfqpoint{-0.044444in}{0.000000in}}{\pgfqpoint{0.000000in}{0.000000in}}{%
\pgfpathmoveto{\pgfqpoint{0.000000in}{0.000000in}}%
\pgfpathlineto{\pgfqpoint{-0.044444in}{0.000000in}}%
\pgfusepath{stroke,fill}%
}%
\begin{pgfscope}%
\pgfsys@transformshift{0.636577in}{1.206188in}%
\pgfsys@useobject{currentmarker}{}%
\end{pgfscope}%
\end{pgfscope}%
\begin{pgfscope}%
\pgfsetbuttcap%
\pgfsetroundjoin%
\definecolor{currentfill}{rgb}{0.411765,0.411765,0.411765}%
\pgfsetfillcolor{currentfill}%
\pgfsetlinewidth{0.803000pt}%
\definecolor{currentstroke}{rgb}{0.411765,0.411765,0.411765}%
\pgfsetstrokecolor{currentstroke}%
\pgfsetdash{}{0pt}%
\pgfsys@defobject{currentmarker}{\pgfqpoint{-0.044444in}{0.000000in}}{\pgfqpoint{0.000000in}{0.000000in}}{%
\pgfpathmoveto{\pgfqpoint{0.000000in}{0.000000in}}%
\pgfpathlineto{\pgfqpoint{-0.044444in}{0.000000in}}%
\pgfusepath{stroke,fill}%
}%
\begin{pgfscope}%
\pgfsys@transformshift{0.636577in}{1.263045in}%
\pgfsys@useobject{currentmarker}{}%
\end{pgfscope}%
\end{pgfscope}%
\begin{pgfscope}%
\pgfsetbuttcap%
\pgfsetroundjoin%
\definecolor{currentfill}{rgb}{0.411765,0.411765,0.411765}%
\pgfsetfillcolor{currentfill}%
\pgfsetlinewidth{0.803000pt}%
\definecolor{currentstroke}{rgb}{0.411765,0.411765,0.411765}%
\pgfsetstrokecolor{currentstroke}%
\pgfsetdash{}{0pt}%
\pgfsys@defobject{currentmarker}{\pgfqpoint{-0.044444in}{0.000000in}}{\pgfqpoint{0.000000in}{0.000000in}}{%
\pgfpathmoveto{\pgfqpoint{0.000000in}{0.000000in}}%
\pgfpathlineto{\pgfqpoint{-0.044444in}{0.000000in}}%
\pgfusepath{stroke,fill}%
}%
\begin{pgfscope}%
\pgfsys@transformshift{0.636577in}{1.319902in}%
\pgfsys@useobject{currentmarker}{}%
\end{pgfscope}%
\end{pgfscope}%
\begin{pgfscope}%
\pgfsetbuttcap%
\pgfsetroundjoin%
\definecolor{currentfill}{rgb}{0.411765,0.411765,0.411765}%
\pgfsetfillcolor{currentfill}%
\pgfsetlinewidth{0.803000pt}%
\definecolor{currentstroke}{rgb}{0.411765,0.411765,0.411765}%
\pgfsetstrokecolor{currentstroke}%
\pgfsetdash{}{0pt}%
\pgfsys@defobject{currentmarker}{\pgfqpoint{-0.044444in}{0.000000in}}{\pgfqpoint{0.000000in}{0.000000in}}{%
\pgfpathmoveto{\pgfqpoint{0.000000in}{0.000000in}}%
\pgfpathlineto{\pgfqpoint{-0.044444in}{0.000000in}}%
\pgfusepath{stroke,fill}%
}%
\begin{pgfscope}%
\pgfsys@transformshift{0.636577in}{1.376759in}%
\pgfsys@useobject{currentmarker}{}%
\end{pgfscope}%
\end{pgfscope}%
\begin{pgfscope}%
\pgfsetbuttcap%
\pgfsetroundjoin%
\definecolor{currentfill}{rgb}{0.411765,0.411765,0.411765}%
\pgfsetfillcolor{currentfill}%
\pgfsetlinewidth{0.803000pt}%
\definecolor{currentstroke}{rgb}{0.411765,0.411765,0.411765}%
\pgfsetstrokecolor{currentstroke}%
\pgfsetdash{}{0pt}%
\pgfsys@defobject{currentmarker}{\pgfqpoint{-0.044444in}{0.000000in}}{\pgfqpoint{0.000000in}{0.000000in}}{%
\pgfpathmoveto{\pgfqpoint{0.000000in}{0.000000in}}%
\pgfpathlineto{\pgfqpoint{-0.044444in}{0.000000in}}%
\pgfusepath{stroke,fill}%
}%
\begin{pgfscope}%
\pgfsys@transformshift{0.636577in}{1.490473in}%
\pgfsys@useobject{currentmarker}{}%
\end{pgfscope}%
\end{pgfscope}%
\begin{pgfscope}%
\pgfsetbuttcap%
\pgfsetroundjoin%
\definecolor{currentfill}{rgb}{0.411765,0.411765,0.411765}%
\pgfsetfillcolor{currentfill}%
\pgfsetlinewidth{0.803000pt}%
\definecolor{currentstroke}{rgb}{0.411765,0.411765,0.411765}%
\pgfsetstrokecolor{currentstroke}%
\pgfsetdash{}{0pt}%
\pgfsys@defobject{currentmarker}{\pgfqpoint{-0.044444in}{0.000000in}}{\pgfqpoint{0.000000in}{0.000000in}}{%
\pgfpathmoveto{\pgfqpoint{0.000000in}{0.000000in}}%
\pgfpathlineto{\pgfqpoint{-0.044444in}{0.000000in}}%
\pgfusepath{stroke,fill}%
}%
\begin{pgfscope}%
\pgfsys@transformshift{0.636577in}{1.547330in}%
\pgfsys@useobject{currentmarker}{}%
\end{pgfscope}%
\end{pgfscope}%
\begin{pgfscope}%
\pgfsetbuttcap%
\pgfsetroundjoin%
\definecolor{currentfill}{rgb}{0.411765,0.411765,0.411765}%
\pgfsetfillcolor{currentfill}%
\pgfsetlinewidth{0.803000pt}%
\definecolor{currentstroke}{rgb}{0.411765,0.411765,0.411765}%
\pgfsetstrokecolor{currentstroke}%
\pgfsetdash{}{0pt}%
\pgfsys@defobject{currentmarker}{\pgfqpoint{-0.044444in}{0.000000in}}{\pgfqpoint{0.000000in}{0.000000in}}{%
\pgfpathmoveto{\pgfqpoint{0.000000in}{0.000000in}}%
\pgfpathlineto{\pgfqpoint{-0.044444in}{0.000000in}}%
\pgfusepath{stroke,fill}%
}%
\begin{pgfscope}%
\pgfsys@transformshift{0.636577in}{1.604187in}%
\pgfsys@useobject{currentmarker}{}%
\end{pgfscope}%
\end{pgfscope}%
\begin{pgfscope}%
\pgfsetbuttcap%
\pgfsetroundjoin%
\definecolor{currentfill}{rgb}{0.411765,0.411765,0.411765}%
\pgfsetfillcolor{currentfill}%
\pgfsetlinewidth{0.803000pt}%
\definecolor{currentstroke}{rgb}{0.411765,0.411765,0.411765}%
\pgfsetstrokecolor{currentstroke}%
\pgfsetdash{}{0pt}%
\pgfsys@defobject{currentmarker}{\pgfqpoint{-0.044444in}{0.000000in}}{\pgfqpoint{0.000000in}{0.000000in}}{%
\pgfpathmoveto{\pgfqpoint{0.000000in}{0.000000in}}%
\pgfpathlineto{\pgfqpoint{-0.044444in}{0.000000in}}%
\pgfusepath{stroke,fill}%
}%
\begin{pgfscope}%
\pgfsys@transformshift{0.636577in}{1.661044in}%
\pgfsys@useobject{currentmarker}{}%
\end{pgfscope}%
\end{pgfscope}%
\begin{pgfscope}%
\pgfsetbuttcap%
\pgfsetroundjoin%
\definecolor{currentfill}{rgb}{0.411765,0.411765,0.411765}%
\pgfsetfillcolor{currentfill}%
\pgfsetlinewidth{0.803000pt}%
\definecolor{currentstroke}{rgb}{0.411765,0.411765,0.411765}%
\pgfsetstrokecolor{currentstroke}%
\pgfsetdash{}{0pt}%
\pgfsys@defobject{currentmarker}{\pgfqpoint{-0.044444in}{0.000000in}}{\pgfqpoint{0.000000in}{0.000000in}}{%
\pgfpathmoveto{\pgfqpoint{0.000000in}{0.000000in}}%
\pgfpathlineto{\pgfqpoint{-0.044444in}{0.000000in}}%
\pgfusepath{stroke,fill}%
}%
\begin{pgfscope}%
\pgfsys@transformshift{0.636577in}{1.774758in}%
\pgfsys@useobject{currentmarker}{}%
\end{pgfscope}%
\end{pgfscope}%
\begin{pgfscope}%
\pgfsetbuttcap%
\pgfsetroundjoin%
\definecolor{currentfill}{rgb}{0.411765,0.411765,0.411765}%
\pgfsetfillcolor{currentfill}%
\pgfsetlinewidth{0.803000pt}%
\definecolor{currentstroke}{rgb}{0.411765,0.411765,0.411765}%
\pgfsetstrokecolor{currentstroke}%
\pgfsetdash{}{0pt}%
\pgfsys@defobject{currentmarker}{\pgfqpoint{-0.044444in}{0.000000in}}{\pgfqpoint{0.000000in}{0.000000in}}{%
\pgfpathmoveto{\pgfqpoint{0.000000in}{0.000000in}}%
\pgfpathlineto{\pgfqpoint{-0.044444in}{0.000000in}}%
\pgfusepath{stroke,fill}%
}%
\begin{pgfscope}%
\pgfsys@transformshift{0.636577in}{1.831615in}%
\pgfsys@useobject{currentmarker}{}%
\end{pgfscope}%
\end{pgfscope}%
\begin{pgfscope}%
\pgfsetbuttcap%
\pgfsetroundjoin%
\definecolor{currentfill}{rgb}{0.411765,0.411765,0.411765}%
\pgfsetfillcolor{currentfill}%
\pgfsetlinewidth{0.803000pt}%
\definecolor{currentstroke}{rgb}{0.411765,0.411765,0.411765}%
\pgfsetstrokecolor{currentstroke}%
\pgfsetdash{}{0pt}%
\pgfsys@defobject{currentmarker}{\pgfqpoint{-0.044444in}{0.000000in}}{\pgfqpoint{0.000000in}{0.000000in}}{%
\pgfpathmoveto{\pgfqpoint{0.000000in}{0.000000in}}%
\pgfpathlineto{\pgfqpoint{-0.044444in}{0.000000in}}%
\pgfusepath{stroke,fill}%
}%
\begin{pgfscope}%
\pgfsys@transformshift{0.636577in}{1.888472in}%
\pgfsys@useobject{currentmarker}{}%
\end{pgfscope}%
\end{pgfscope}%
\begin{pgfscope}%
\pgfsetbuttcap%
\pgfsetroundjoin%
\definecolor{currentfill}{rgb}{0.411765,0.411765,0.411765}%
\pgfsetfillcolor{currentfill}%
\pgfsetlinewidth{0.803000pt}%
\definecolor{currentstroke}{rgb}{0.411765,0.411765,0.411765}%
\pgfsetstrokecolor{currentstroke}%
\pgfsetdash{}{0pt}%
\pgfsys@defobject{currentmarker}{\pgfqpoint{-0.044444in}{0.000000in}}{\pgfqpoint{0.000000in}{0.000000in}}{%
\pgfpathmoveto{\pgfqpoint{0.000000in}{0.000000in}}%
\pgfpathlineto{\pgfqpoint{-0.044444in}{0.000000in}}%
\pgfusepath{stroke,fill}%
}%
\begin{pgfscope}%
\pgfsys@transformshift{0.636577in}{1.945329in}%
\pgfsys@useobject{currentmarker}{}%
\end{pgfscope}%
\end{pgfscope}%
\begin{pgfscope}%
\pgfsetbuttcap%
\pgfsetroundjoin%
\definecolor{currentfill}{rgb}{0.411765,0.411765,0.411765}%
\pgfsetfillcolor{currentfill}%
\pgfsetlinewidth{0.803000pt}%
\definecolor{currentstroke}{rgb}{0.411765,0.411765,0.411765}%
\pgfsetstrokecolor{currentstroke}%
\pgfsetdash{}{0pt}%
\pgfsys@defobject{currentmarker}{\pgfqpoint{-0.044444in}{0.000000in}}{\pgfqpoint{0.000000in}{0.000000in}}{%
\pgfpathmoveto{\pgfqpoint{0.000000in}{0.000000in}}%
\pgfpathlineto{\pgfqpoint{-0.044444in}{0.000000in}}%
\pgfusepath{stroke,fill}%
}%
\begin{pgfscope}%
\pgfsys@transformshift{0.636577in}{2.059043in}%
\pgfsys@useobject{currentmarker}{}%
\end{pgfscope}%
\end{pgfscope}%
\begin{pgfscope}%
\pgfsetbuttcap%
\pgfsetroundjoin%
\definecolor{currentfill}{rgb}{0.411765,0.411765,0.411765}%
\pgfsetfillcolor{currentfill}%
\pgfsetlinewidth{0.803000pt}%
\definecolor{currentstroke}{rgb}{0.411765,0.411765,0.411765}%
\pgfsetstrokecolor{currentstroke}%
\pgfsetdash{}{0pt}%
\pgfsys@defobject{currentmarker}{\pgfqpoint{-0.044444in}{0.000000in}}{\pgfqpoint{0.000000in}{0.000000in}}{%
\pgfpathmoveto{\pgfqpoint{0.000000in}{0.000000in}}%
\pgfpathlineto{\pgfqpoint{-0.044444in}{0.000000in}}%
\pgfusepath{stroke,fill}%
}%
\begin{pgfscope}%
\pgfsys@transformshift{0.636577in}{2.115900in}%
\pgfsys@useobject{currentmarker}{}%
\end{pgfscope}%
\end{pgfscope}%
\begin{pgfscope}%
\pgfsetbuttcap%
\pgfsetroundjoin%
\definecolor{currentfill}{rgb}{0.411765,0.411765,0.411765}%
\pgfsetfillcolor{currentfill}%
\pgfsetlinewidth{0.803000pt}%
\definecolor{currentstroke}{rgb}{0.411765,0.411765,0.411765}%
\pgfsetstrokecolor{currentstroke}%
\pgfsetdash{}{0pt}%
\pgfsys@defobject{currentmarker}{\pgfqpoint{-0.044444in}{0.000000in}}{\pgfqpoint{0.000000in}{0.000000in}}{%
\pgfpathmoveto{\pgfqpoint{0.000000in}{0.000000in}}%
\pgfpathlineto{\pgfqpoint{-0.044444in}{0.000000in}}%
\pgfusepath{stroke,fill}%
}%
\begin{pgfscope}%
\pgfsys@transformshift{0.636577in}{2.172757in}%
\pgfsys@useobject{currentmarker}{}%
\end{pgfscope}%
\end{pgfscope}%
\begin{pgfscope}%
\pgfsetbuttcap%
\pgfsetroundjoin%
\definecolor{currentfill}{rgb}{0.411765,0.411765,0.411765}%
\pgfsetfillcolor{currentfill}%
\pgfsetlinewidth{0.803000pt}%
\definecolor{currentstroke}{rgb}{0.411765,0.411765,0.411765}%
\pgfsetstrokecolor{currentstroke}%
\pgfsetdash{}{0pt}%
\pgfsys@defobject{currentmarker}{\pgfqpoint{-0.044444in}{0.000000in}}{\pgfqpoint{0.000000in}{0.000000in}}{%
\pgfpathmoveto{\pgfqpoint{0.000000in}{0.000000in}}%
\pgfpathlineto{\pgfqpoint{-0.044444in}{0.000000in}}%
\pgfusepath{stroke,fill}%
}%
\begin{pgfscope}%
\pgfsys@transformshift{0.636577in}{2.229614in}%
\pgfsys@useobject{currentmarker}{}%
\end{pgfscope}%
\end{pgfscope}%
\begin{pgfscope}%
\pgfsetbuttcap%
\pgfsetroundjoin%
\definecolor{currentfill}{rgb}{0.411765,0.411765,0.411765}%
\pgfsetfillcolor{currentfill}%
\pgfsetlinewidth{0.803000pt}%
\definecolor{currentstroke}{rgb}{0.411765,0.411765,0.411765}%
\pgfsetstrokecolor{currentstroke}%
\pgfsetdash{}{0pt}%
\pgfsys@defobject{currentmarker}{\pgfqpoint{-0.044444in}{0.000000in}}{\pgfqpoint{0.000000in}{0.000000in}}{%
\pgfpathmoveto{\pgfqpoint{0.000000in}{0.000000in}}%
\pgfpathlineto{\pgfqpoint{-0.044444in}{0.000000in}}%
\pgfusepath{stroke,fill}%
}%
\begin{pgfscope}%
\pgfsys@transformshift{0.636577in}{2.343328in}%
\pgfsys@useobject{currentmarker}{}%
\end{pgfscope}%
\end{pgfscope}%
\begin{pgfscope}%
\pgfsetbuttcap%
\pgfsetroundjoin%
\definecolor{currentfill}{rgb}{0.411765,0.411765,0.411765}%
\pgfsetfillcolor{currentfill}%
\pgfsetlinewidth{0.803000pt}%
\definecolor{currentstroke}{rgb}{0.411765,0.411765,0.411765}%
\pgfsetstrokecolor{currentstroke}%
\pgfsetdash{}{0pt}%
\pgfsys@defobject{currentmarker}{\pgfqpoint{-0.044444in}{0.000000in}}{\pgfqpoint{0.000000in}{0.000000in}}{%
\pgfpathmoveto{\pgfqpoint{0.000000in}{0.000000in}}%
\pgfpathlineto{\pgfqpoint{-0.044444in}{0.000000in}}%
\pgfusepath{stroke,fill}%
}%
\begin{pgfscope}%
\pgfsys@transformshift{0.636577in}{2.400185in}%
\pgfsys@useobject{currentmarker}{}%
\end{pgfscope}%
\end{pgfscope}%
\begin{pgfscope}%
\pgfsetbuttcap%
\pgfsetroundjoin%
\definecolor{currentfill}{rgb}{0.411765,0.411765,0.411765}%
\pgfsetfillcolor{currentfill}%
\pgfsetlinewidth{0.803000pt}%
\definecolor{currentstroke}{rgb}{0.411765,0.411765,0.411765}%
\pgfsetstrokecolor{currentstroke}%
\pgfsetdash{}{0pt}%
\pgfsys@defobject{currentmarker}{\pgfqpoint{-0.044444in}{0.000000in}}{\pgfqpoint{0.000000in}{0.000000in}}{%
\pgfpathmoveto{\pgfqpoint{0.000000in}{0.000000in}}%
\pgfpathlineto{\pgfqpoint{-0.044444in}{0.000000in}}%
\pgfusepath{stroke,fill}%
}%
\begin{pgfscope}%
\pgfsys@transformshift{0.636577in}{2.457042in}%
\pgfsys@useobject{currentmarker}{}%
\end{pgfscope}%
\end{pgfscope}%
\begin{pgfscope}%
\pgfsetbuttcap%
\pgfsetroundjoin%
\definecolor{currentfill}{rgb}{0.411765,0.411765,0.411765}%
\pgfsetfillcolor{currentfill}%
\pgfsetlinewidth{0.803000pt}%
\definecolor{currentstroke}{rgb}{0.411765,0.411765,0.411765}%
\pgfsetstrokecolor{currentstroke}%
\pgfsetdash{}{0pt}%
\pgfsys@defobject{currentmarker}{\pgfqpoint{-0.044444in}{0.000000in}}{\pgfqpoint{0.000000in}{0.000000in}}{%
\pgfpathmoveto{\pgfqpoint{0.000000in}{0.000000in}}%
\pgfpathlineto{\pgfqpoint{-0.044444in}{0.000000in}}%
\pgfusepath{stroke,fill}%
}%
\begin{pgfscope}%
\pgfsys@transformshift{0.636577in}{2.513899in}%
\pgfsys@useobject{currentmarker}{}%
\end{pgfscope}%
\end{pgfscope}%
\begin{pgfscope}%
\pgfsetbuttcap%
\pgfsetroundjoin%
\definecolor{currentfill}{rgb}{0.411765,0.411765,0.411765}%
\pgfsetfillcolor{currentfill}%
\pgfsetlinewidth{0.803000pt}%
\definecolor{currentstroke}{rgb}{0.411765,0.411765,0.411765}%
\pgfsetstrokecolor{currentstroke}%
\pgfsetdash{}{0pt}%
\pgfsys@defobject{currentmarker}{\pgfqpoint{-0.044444in}{0.000000in}}{\pgfqpoint{0.000000in}{0.000000in}}{%
\pgfpathmoveto{\pgfqpoint{0.000000in}{0.000000in}}%
\pgfpathlineto{\pgfqpoint{-0.044444in}{0.000000in}}%
\pgfusepath{stroke,fill}%
}%
\begin{pgfscope}%
\pgfsys@transformshift{0.636577in}{2.627613in}%
\pgfsys@useobject{currentmarker}{}%
\end{pgfscope}%
\end{pgfscope}%
\begin{pgfscope}%
\definecolor{textcolor}{rgb}{0.000000,0.000000,0.000000}%
\pgfsetstrokecolor{textcolor}%
\pgfsetfillcolor{textcolor}%
\pgftext[x=0.273036in,y=1.614090in,,bottom,rotate=90.000000]{\color{textcolor}\rmfamily\fontsize{9.600000}{11.520000}\selectfont Comprimento de onde [\SI{}{\nano\meter}]}%
\end{pgfscope}%
\begin{pgfscope}%
\pgfpathrectangle{\pgfqpoint{0.636577in}{0.562153in}}{\pgfqpoint{4.171423in}{2.103873in}}%
\pgfusepath{clip}%
\pgfsetbuttcap%
\pgfsetroundjoin%
\definecolor{currentfill}{rgb}{0.933333,0.521569,0.290196}%
\pgfsetfillcolor{currentfill}%
\pgfsetlinewidth{0.752812pt}%
\definecolor{currentstroke}{rgb}{1.000000,1.000000,1.000000}%
\pgfsetstrokecolor{currentstroke}%
\pgfsetdash{}{0pt}%
\pgfpathmoveto{\pgfqpoint{1.780391in}{1.625152in}}%
\pgfpathcurveto{\pgfqpoint{1.789231in}{1.625152in}}{\pgfqpoint{1.797710in}{1.628664in}}{\pgfqpoint{1.803961in}{1.634915in}}%
\pgfpathcurveto{\pgfqpoint{1.810212in}{1.641166in}}{\pgfqpoint{1.813724in}{1.649645in}}{\pgfqpoint{1.813724in}{1.658486in}}%
\pgfpathcurveto{\pgfqpoint{1.813724in}{1.667326in}}{\pgfqpoint{1.810212in}{1.675805in}}{\pgfqpoint{1.803961in}{1.682056in}}%
\pgfpathcurveto{\pgfqpoint{1.797710in}{1.688307in}}{\pgfqpoint{1.789231in}{1.691819in}}{\pgfqpoint{1.780391in}{1.691819in}}%
\pgfpathcurveto{\pgfqpoint{1.771551in}{1.691819in}}{\pgfqpoint{1.763072in}{1.688307in}}{\pgfqpoint{1.756821in}{1.682056in}}%
\pgfpathcurveto{\pgfqpoint{1.750570in}{1.675805in}}{\pgfqpoint{1.747058in}{1.667326in}}{\pgfqpoint{1.747058in}{1.658486in}}%
\pgfpathcurveto{\pgfqpoint{1.747058in}{1.649645in}}{\pgfqpoint{1.750570in}{1.641166in}}{\pgfqpoint{1.756821in}{1.634915in}}%
\pgfpathcurveto{\pgfqpoint{1.763072in}{1.628664in}}{\pgfqpoint{1.771551in}{1.625152in}}{\pgfqpoint{1.780391in}{1.625152in}}%
\pgfpathclose%
\pgfusepath{stroke,fill}%
\end{pgfscope}%
\begin{pgfscope}%
\pgfpathrectangle{\pgfqpoint{0.636577in}{0.562153in}}{\pgfqpoint{4.171423in}{2.103873in}}%
\pgfusepath{clip}%
\pgfsetbuttcap%
\pgfsetroundjoin%
\definecolor{currentfill}{rgb}{0.933333,0.521569,0.290196}%
\pgfsetfillcolor{currentfill}%
\pgfsetlinewidth{0.752812pt}%
\definecolor{currentstroke}{rgb}{1.000000,1.000000,1.000000}%
\pgfsetstrokecolor{currentstroke}%
\pgfsetdash{}{0pt}%
\pgfpathmoveto{\pgfqpoint{1.989701in}{1.502341in}}%
\pgfpathcurveto{\pgfqpoint{1.998542in}{1.502341in}}{\pgfqpoint{2.007021in}{1.505853in}}{\pgfqpoint{2.013272in}{1.512104in}}%
\pgfpathcurveto{\pgfqpoint{2.019523in}{1.518355in}}{\pgfqpoint{2.023035in}{1.526834in}}{\pgfqpoint{2.023035in}{1.535674in}}%
\pgfpathcurveto{\pgfqpoint{2.023035in}{1.544515in}}{\pgfqpoint{2.019523in}{1.552994in}}{\pgfqpoint{2.013272in}{1.559245in}}%
\pgfpathcurveto{\pgfqpoint{2.007021in}{1.565496in}}{\pgfqpoint{1.998542in}{1.569008in}}{\pgfqpoint{1.989701in}{1.569008in}}%
\pgfpathcurveto{\pgfqpoint{1.980861in}{1.569008in}}{\pgfqpoint{1.972382in}{1.565496in}}{\pgfqpoint{1.966131in}{1.559245in}}%
\pgfpathcurveto{\pgfqpoint{1.959880in}{1.552994in}}{\pgfqpoint{1.956368in}{1.544515in}}{\pgfqpoint{1.956368in}{1.535674in}}%
\pgfpathcurveto{\pgfqpoint{1.956368in}{1.526834in}}{\pgfqpoint{1.959880in}{1.518355in}}{\pgfqpoint{1.966131in}{1.512104in}}%
\pgfpathcurveto{\pgfqpoint{1.972382in}{1.505853in}}{\pgfqpoint{1.980861in}{1.502341in}}{\pgfqpoint{1.989701in}{1.502341in}}%
\pgfpathclose%
\pgfusepath{stroke,fill}%
\end{pgfscope}%
\begin{pgfscope}%
\pgfpathrectangle{\pgfqpoint{0.636577in}{0.562153in}}{\pgfqpoint{4.171423in}{2.103873in}}%
\pgfusepath{clip}%
\pgfsetbuttcap%
\pgfsetroundjoin%
\definecolor{currentfill}{rgb}{0.933333,0.521569,0.290196}%
\pgfsetfillcolor{currentfill}%
\pgfsetlinewidth{0.752812pt}%
\definecolor{currentstroke}{rgb}{1.000000,1.000000,1.000000}%
\pgfsetstrokecolor{currentstroke}%
\pgfsetdash{}{0pt}%
\pgfpathmoveto{\pgfqpoint{3.664186in}{0.815224in}}%
\pgfpathcurveto{\pgfqpoint{3.673026in}{0.815224in}}{\pgfqpoint{3.681505in}{0.818736in}}{\pgfqpoint{3.687756in}{0.824987in}}%
\pgfpathcurveto{\pgfqpoint{3.694007in}{0.831238in}}{\pgfqpoint{3.697519in}{0.839717in}}{\pgfqpoint{3.697519in}{0.848558in}}%
\pgfpathcurveto{\pgfqpoint{3.697519in}{0.857398in}}{\pgfqpoint{3.694007in}{0.865877in}}{\pgfqpoint{3.687756in}{0.872128in}}%
\pgfpathcurveto{\pgfqpoint{3.681505in}{0.878379in}}{\pgfqpoint{3.673026in}{0.881891in}}{\pgfqpoint{3.664186in}{0.881891in}}%
\pgfpathcurveto{\pgfqpoint{3.655346in}{0.881891in}}{\pgfqpoint{3.646867in}{0.878379in}}{\pgfqpoint{3.640616in}{0.872128in}}%
\pgfpathcurveto{\pgfqpoint{3.634365in}{0.865877in}}{\pgfqpoint{3.630853in}{0.857398in}}{\pgfqpoint{3.630853in}{0.848558in}}%
\pgfpathcurveto{\pgfqpoint{3.630853in}{0.839717in}}{\pgfqpoint{3.634365in}{0.831238in}}{\pgfqpoint{3.640616in}{0.824987in}}%
\pgfpathcurveto{\pgfqpoint{3.646867in}{0.818736in}}{\pgfqpoint{3.655346in}{0.815224in}}{\pgfqpoint{3.664186in}{0.815224in}}%
\pgfpathclose%
\pgfusepath{stroke,fill}%
\end{pgfscope}%
\begin{pgfscope}%
\pgfpathrectangle{\pgfqpoint{0.636577in}{0.562153in}}{\pgfqpoint{4.171423in}{2.103873in}}%
\pgfusepath{clip}%
\pgfsetbuttcap%
\pgfsetroundjoin%
\definecolor{currentfill}{rgb}{0.415686,0.800000,0.392157}%
\pgfsetfillcolor{currentfill}%
\pgfsetlinewidth{0.752812pt}%
\definecolor{currentstroke}{rgb}{1.000000,1.000000,1.000000}%
\pgfsetstrokecolor{currentstroke}%
\pgfsetdash{}{0pt}%
\pgfpathmoveto{\pgfqpoint{2.408323in}{1.374583in}}%
\pgfpathcurveto{\pgfqpoint{2.417163in}{1.374583in}}{\pgfqpoint{2.425642in}{1.378096in}}{\pgfqpoint{2.431893in}{1.384347in}}%
\pgfpathcurveto{\pgfqpoint{2.438144in}{1.390597in}}{\pgfqpoint{2.441656in}{1.399077in}}{\pgfqpoint{2.441656in}{1.407917in}}%
\pgfpathcurveto{\pgfqpoint{2.441656in}{1.416757in}}{\pgfqpoint{2.438144in}{1.425236in}}{\pgfqpoint{2.431893in}{1.431487in}}%
\pgfpathcurveto{\pgfqpoint{2.425642in}{1.437738in}}{\pgfqpoint{2.417163in}{1.441250in}}{\pgfqpoint{2.408323in}{1.441250in}}%
\pgfpathcurveto{\pgfqpoint{2.399482in}{1.441250in}}{\pgfqpoint{2.391003in}{1.437738in}}{\pgfqpoint{2.384752in}{1.431487in}}%
\pgfpathcurveto{\pgfqpoint{2.378501in}{1.425236in}}{\pgfqpoint{2.374989in}{1.416757in}}{\pgfqpoint{2.374989in}{1.407917in}}%
\pgfpathcurveto{\pgfqpoint{2.374989in}{1.399077in}}{\pgfqpoint{2.378501in}{1.390597in}}{\pgfqpoint{2.384752in}{1.384347in}}%
\pgfpathcurveto{\pgfqpoint{2.391003in}{1.378096in}}{\pgfqpoint{2.399482in}{1.374583in}}{\pgfqpoint{2.408323in}{1.374583in}}%
\pgfpathclose%
\pgfusepath{stroke,fill}%
\end{pgfscope}%
\begin{pgfscope}%
\pgfpathrectangle{\pgfqpoint{0.636577in}{0.562153in}}{\pgfqpoint{4.171423in}{2.103873in}}%
\pgfusepath{clip}%
\pgfsetbuttcap%
\pgfsetroundjoin%
\definecolor{currentfill}{rgb}{0.415686,0.800000,0.392157}%
\pgfsetfillcolor{currentfill}%
\pgfsetlinewidth{0.752812pt}%
\definecolor{currentstroke}{rgb}{1.000000,1.000000,1.000000}%
\pgfsetstrokecolor{currentstroke}%
\pgfsetdash{}{0pt}%
\pgfpathmoveto{\pgfqpoint{2.199012in}{1.556640in}}%
\pgfpathcurveto{\pgfqpoint{2.207852in}{1.556640in}}{\pgfqpoint{2.216331in}{1.560152in}}{\pgfqpoint{2.222582in}{1.566403in}}%
\pgfpathcurveto{\pgfqpoint{2.228833in}{1.572654in}}{\pgfqpoint{2.232345in}{1.581133in}}{\pgfqpoint{2.232345in}{1.589973in}}%
\pgfpathcurveto{\pgfqpoint{2.232345in}{1.598813in}}{\pgfqpoint{2.228833in}{1.607292in}}{\pgfqpoint{2.222582in}{1.613543in}}%
\pgfpathcurveto{\pgfqpoint{2.216331in}{1.619794in}}{\pgfqpoint{2.207852in}{1.623306in}}{\pgfqpoint{2.199012in}{1.623306in}}%
\pgfpathcurveto{\pgfqpoint{2.190172in}{1.623306in}}{\pgfqpoint{2.181693in}{1.619794in}}{\pgfqpoint{2.175442in}{1.613543in}}%
\pgfpathcurveto{\pgfqpoint{2.169191in}{1.607292in}}{\pgfqpoint{2.165679in}{1.598813in}}{\pgfqpoint{2.165679in}{1.589973in}}%
\pgfpathcurveto{\pgfqpoint{2.165679in}{1.581133in}}{\pgfqpoint{2.169191in}{1.572654in}}{\pgfqpoint{2.175442in}{1.566403in}}%
\pgfpathcurveto{\pgfqpoint{2.181693in}{1.560152in}}{\pgfqpoint{2.190172in}{1.556640in}}{\pgfqpoint{2.199012in}{1.556640in}}%
\pgfpathclose%
\pgfusepath{stroke,fill}%
\end{pgfscope}%
\begin{pgfscope}%
\pgfpathrectangle{\pgfqpoint{0.636577in}{0.562153in}}{\pgfqpoint{4.171423in}{2.103873in}}%
\pgfusepath{clip}%
\pgfsetbuttcap%
\pgfsetroundjoin%
\definecolor{currentfill}{rgb}{0.415686,0.800000,0.392157}%
\pgfsetfillcolor{currentfill}%
\pgfsetlinewidth{0.752812pt}%
\definecolor{currentstroke}{rgb}{1.000000,1.000000,1.000000}%
\pgfsetstrokecolor{currentstroke}%
\pgfsetdash{}{0pt}%
\pgfpathmoveto{\pgfqpoint{4.292118in}{0.747053in}}%
\pgfpathcurveto{\pgfqpoint{4.300958in}{0.747053in}}{\pgfqpoint{4.309437in}{0.750565in}}{\pgfqpoint{4.315688in}{0.756816in}}%
\pgfpathcurveto{\pgfqpoint{4.321939in}{0.763067in}}{\pgfqpoint{4.325451in}{0.771546in}}{\pgfqpoint{4.325451in}{0.780386in}}%
\pgfpathcurveto{\pgfqpoint{4.325451in}{0.789226in}}{\pgfqpoint{4.321939in}{0.797705in}}{\pgfqpoint{4.315688in}{0.803956in}}%
\pgfpathcurveto{\pgfqpoint{4.309437in}{0.810207in}}{\pgfqpoint{4.300958in}{0.813719in}}{\pgfqpoint{4.292118in}{0.813719in}}%
\pgfpathcurveto{\pgfqpoint{4.283277in}{0.813719in}}{\pgfqpoint{4.274798in}{0.810207in}}{\pgfqpoint{4.268547in}{0.803956in}}%
\pgfpathcurveto{\pgfqpoint{4.262296in}{0.797705in}}{\pgfqpoint{4.258784in}{0.789226in}}{\pgfqpoint{4.258784in}{0.780386in}}%
\pgfpathcurveto{\pgfqpoint{4.258784in}{0.771546in}}{\pgfqpoint{4.262296in}{0.763067in}}{\pgfqpoint{4.268547in}{0.756816in}}%
\pgfpathcurveto{\pgfqpoint{4.274798in}{0.750565in}}{\pgfqpoint{4.283277in}{0.747053in}}{\pgfqpoint{4.292118in}{0.747053in}}%
\pgfpathclose%
\pgfusepath{stroke,fill}%
\end{pgfscope}%
\begin{pgfscope}%
\pgfpathrectangle{\pgfqpoint{0.636577in}{0.562153in}}{\pgfqpoint{4.171423in}{2.103873in}}%
\pgfusepath{clip}%
\pgfsetbuttcap%
\pgfsetroundjoin%
\definecolor{currentfill}{rgb}{0.839216,0.372549,0.372549}%
\pgfsetfillcolor{currentfill}%
\pgfsetlinewidth{0.752812pt}%
\definecolor{currentstroke}{rgb}{1.000000,1.000000,1.000000}%
\pgfsetstrokecolor{currentstroke}%
\pgfsetdash{}{0pt}%
\pgfpathmoveto{\pgfqpoint{1.989701in}{1.609062in}}%
\pgfpathcurveto{\pgfqpoint{1.998542in}{1.609062in}}{\pgfqpoint{2.007021in}{1.612574in}}{\pgfqpoint{2.013272in}{1.618825in}}%
\pgfpathcurveto{\pgfqpoint{2.019523in}{1.625076in}}{\pgfqpoint{2.023035in}{1.633555in}}{\pgfqpoint{2.023035in}{1.642395in}}%
\pgfpathcurveto{\pgfqpoint{2.023035in}{1.651235in}}{\pgfqpoint{2.019523in}{1.659714in}}{\pgfqpoint{2.013272in}{1.665965in}}%
\pgfpathcurveto{\pgfqpoint{2.007021in}{1.672216in}}{\pgfqpoint{1.998542in}{1.675728in}}{\pgfqpoint{1.989701in}{1.675728in}}%
\pgfpathcurveto{\pgfqpoint{1.980861in}{1.675728in}}{\pgfqpoint{1.972382in}{1.672216in}}{\pgfqpoint{1.966131in}{1.665965in}}%
\pgfpathcurveto{\pgfqpoint{1.959880in}{1.659714in}}{\pgfqpoint{1.956368in}{1.651235in}}{\pgfqpoint{1.956368in}{1.642395in}}%
\pgfpathcurveto{\pgfqpoint{1.956368in}{1.633555in}}{\pgfqpoint{1.959880in}{1.625076in}}{\pgfqpoint{1.966131in}{1.618825in}}%
\pgfpathcurveto{\pgfqpoint{1.972382in}{1.612574in}}{\pgfqpoint{1.980861in}{1.609062in}}{\pgfqpoint{1.989701in}{1.609062in}}%
\pgfpathclose%
\pgfusepath{stroke,fill}%
\end{pgfscope}%
\begin{pgfscope}%
\pgfpathrectangle{\pgfqpoint{0.636577in}{0.562153in}}{\pgfqpoint{4.171423in}{2.103873in}}%
\pgfusepath{clip}%
\pgfsetbuttcap%
\pgfsetroundjoin%
\definecolor{currentfill}{rgb}{0.839216,0.372549,0.372549}%
\pgfsetfillcolor{currentfill}%
\pgfsetlinewidth{0.752812pt}%
\definecolor{currentstroke}{rgb}{1.000000,1.000000,1.000000}%
\pgfsetstrokecolor{currentstroke}%
\pgfsetdash{}{0pt}%
\pgfpathmoveto{\pgfqpoint{2.931599in}{1.120376in}}%
\pgfpathcurveto{\pgfqpoint{2.940439in}{1.120376in}}{\pgfqpoint{2.948918in}{1.123888in}}{\pgfqpoint{2.955169in}{1.130139in}}%
\pgfpathcurveto{\pgfqpoint{2.961420in}{1.136390in}}{\pgfqpoint{2.964932in}{1.144869in}}{\pgfqpoint{2.964932in}{1.153709in}}%
\pgfpathcurveto{\pgfqpoint{2.964932in}{1.162549in}}{\pgfqpoint{2.961420in}{1.171028in}}{\pgfqpoint{2.955169in}{1.177279in}}%
\pgfpathcurveto{\pgfqpoint{2.948918in}{1.183530in}}{\pgfqpoint{2.940439in}{1.187042in}}{\pgfqpoint{2.931599in}{1.187042in}}%
\pgfpathcurveto{\pgfqpoint{2.922759in}{1.187042in}}{\pgfqpoint{2.914280in}{1.183530in}}{\pgfqpoint{2.908029in}{1.177279in}}%
\pgfpathcurveto{\pgfqpoint{2.901778in}{1.171028in}}{\pgfqpoint{2.898266in}{1.162549in}}{\pgfqpoint{2.898266in}{1.153709in}}%
\pgfpathcurveto{\pgfqpoint{2.898266in}{1.144869in}}{\pgfqpoint{2.901778in}{1.136390in}}{\pgfqpoint{2.908029in}{1.130139in}}%
\pgfpathcurveto{\pgfqpoint{2.914280in}{1.123888in}}{\pgfqpoint{2.922759in}{1.120376in}}{\pgfqpoint{2.931599in}{1.120376in}}%
\pgfpathclose%
\pgfusepath{stroke,fill}%
\end{pgfscope}%
\begin{pgfscope}%
\pgfpathrectangle{\pgfqpoint{0.636577in}{0.562153in}}{\pgfqpoint{4.171423in}{2.103873in}}%
\pgfusepath{clip}%
\pgfsetbuttcap%
\pgfsetroundjoin%
\definecolor{currentfill}{rgb}{0.839216,0.372549,0.372549}%
\pgfsetfillcolor{currentfill}%
\pgfsetlinewidth{0.752812pt}%
\definecolor{currentstroke}{rgb}{1.000000,1.000000,1.000000}%
\pgfsetstrokecolor{currentstroke}%
\pgfsetdash{}{0pt}%
\pgfpathmoveto{\pgfqpoint{3.873496in}{0.812211in}}%
\pgfpathcurveto{\pgfqpoint{3.882337in}{0.812211in}}{\pgfqpoint{3.890816in}{0.815723in}}{\pgfqpoint{3.897067in}{0.821974in}}%
\pgfpathcurveto{\pgfqpoint{3.903318in}{0.828225in}}{\pgfqpoint{3.906830in}{0.836704in}}{\pgfqpoint{3.906830in}{0.845544in}}%
\pgfpathcurveto{\pgfqpoint{3.906830in}{0.854384in}}{\pgfqpoint{3.903318in}{0.862863in}}{\pgfqpoint{3.897067in}{0.869114in}}%
\pgfpathcurveto{\pgfqpoint{3.890816in}{0.875365in}}{\pgfqpoint{3.882337in}{0.878877in}}{\pgfqpoint{3.873496in}{0.878877in}}%
\pgfpathcurveto{\pgfqpoint{3.864656in}{0.878877in}}{\pgfqpoint{3.856177in}{0.875365in}}{\pgfqpoint{3.849926in}{0.869114in}}%
\pgfpathcurveto{\pgfqpoint{3.843675in}{0.862863in}}{\pgfqpoint{3.840163in}{0.854384in}}{\pgfqpoint{3.840163in}{0.845544in}}%
\pgfpathcurveto{\pgfqpoint{3.840163in}{0.836704in}}{\pgfqpoint{3.843675in}{0.828225in}}{\pgfqpoint{3.849926in}{0.821974in}}%
\pgfpathcurveto{\pgfqpoint{3.856177in}{0.815723in}}{\pgfqpoint{3.864656in}{0.812211in}}{\pgfqpoint{3.873496in}{0.812211in}}%
\pgfpathclose%
\pgfusepath{stroke,fill}%
\end{pgfscope}%
\begin{pgfscope}%
\pgfpathrectangle{\pgfqpoint{0.636577in}{0.562153in}}{\pgfqpoint{4.171423in}{2.103873in}}%
\pgfusepath{clip}%
\pgfsetbuttcap%
\pgfsetroundjoin%
\definecolor{currentfill}{rgb}{0.839216,0.372549,0.372549}%
\pgfsetfillcolor{currentfill}%
\pgfsetlinewidth{0.752812pt}%
\definecolor{currentstroke}{rgb}{1.000000,1.000000,1.000000}%
\pgfsetstrokecolor{currentstroke}%
\pgfsetdash{}{0pt}%
\pgfpathmoveto{\pgfqpoint{1.152459in}{2.070058in}}%
\pgfpathcurveto{\pgfqpoint{1.161299in}{2.070058in}}{\pgfqpoint{1.169779in}{2.073571in}}{\pgfqpoint{1.176030in}{2.079821in}}%
\pgfpathcurveto{\pgfqpoint{1.182280in}{2.086072in}}{\pgfqpoint{1.185793in}{2.094552in}}{\pgfqpoint{1.185793in}{2.103392in}}%
\pgfpathcurveto{\pgfqpoint{1.185793in}{2.112232in}}{\pgfqpoint{1.182280in}{2.120711in}}{\pgfqpoint{1.176030in}{2.126962in}}%
\pgfpathcurveto{\pgfqpoint{1.169779in}{2.133213in}}{\pgfqpoint{1.161299in}{2.136725in}}{\pgfqpoint{1.152459in}{2.136725in}}%
\pgfpathcurveto{\pgfqpoint{1.143619in}{2.136725in}}{\pgfqpoint{1.135140in}{2.133213in}}{\pgfqpoint{1.128889in}{2.126962in}}%
\pgfpathcurveto{\pgfqpoint{1.122638in}{2.120711in}}{\pgfqpoint{1.119126in}{2.112232in}}{\pgfqpoint{1.119126in}{2.103392in}}%
\pgfpathcurveto{\pgfqpoint{1.119126in}{2.094552in}}{\pgfqpoint{1.122638in}{2.086072in}}{\pgfqpoint{1.128889in}{2.079821in}}%
\pgfpathcurveto{\pgfqpoint{1.135140in}{2.073571in}}{\pgfqpoint{1.143619in}{2.070058in}}{\pgfqpoint{1.152459in}{2.070058in}}%
\pgfpathclose%
\pgfusepath{stroke,fill}%
\end{pgfscope}%
\begin{pgfscope}%
\pgfpathrectangle{\pgfqpoint{0.636577in}{0.562153in}}{\pgfqpoint{4.171423in}{2.103873in}}%
\pgfusepath{clip}%
\pgfsetbuttcap%
\pgfsetroundjoin%
\definecolor{currentfill}{rgb}{0.282353,0.470588,0.815686}%
\pgfsetfillcolor{currentfill}%
\pgfsetlinewidth{0.752812pt}%
\definecolor{currentstroke}{rgb}{1.000000,1.000000,1.000000}%
\pgfsetstrokecolor{currentstroke}%
\pgfsetdash{}{0pt}%
\pgfpathmoveto{\pgfqpoint{2.931599in}{1.006434in}}%
\pgfpathcurveto{\pgfqpoint{2.940439in}{1.006434in}}{\pgfqpoint{2.948918in}{1.009946in}}{\pgfqpoint{2.955169in}{1.016197in}}%
\pgfpathcurveto{\pgfqpoint{2.961420in}{1.022448in}}{\pgfqpoint{2.964932in}{1.030928in}}{\pgfqpoint{2.964932in}{1.039768in}}%
\pgfpathcurveto{\pgfqpoint{2.964932in}{1.048608in}}{\pgfqpoint{2.961420in}{1.057087in}}{\pgfqpoint{2.955169in}{1.063338in}}%
\pgfpathcurveto{\pgfqpoint{2.948918in}{1.069589in}}{\pgfqpoint{2.940439in}{1.073101in}}{\pgfqpoint{2.931599in}{1.073101in}}%
\pgfpathcurveto{\pgfqpoint{2.922759in}{1.073101in}}{\pgfqpoint{2.914280in}{1.069589in}}{\pgfqpoint{2.908029in}{1.063338in}}%
\pgfpathcurveto{\pgfqpoint{2.901778in}{1.057087in}}{\pgfqpoint{2.898266in}{1.048608in}}{\pgfqpoint{2.898266in}{1.039768in}}%
\pgfpathcurveto{\pgfqpoint{2.898266in}{1.030928in}}{\pgfqpoint{2.901778in}{1.022448in}}{\pgfqpoint{2.908029in}{1.016197in}}%
\pgfpathcurveto{\pgfqpoint{2.914280in}{1.009946in}}{\pgfqpoint{2.922759in}{1.006434in}}{\pgfqpoint{2.931599in}{1.006434in}}%
\pgfpathclose%
\pgfusepath{stroke,fill}%
\end{pgfscope}%
\begin{pgfscope}%
\pgfpathrectangle{\pgfqpoint{0.636577in}{0.562153in}}{\pgfqpoint{4.171423in}{2.103873in}}%
\pgfusepath{clip}%
\pgfsetbuttcap%
\pgfsetroundjoin%
\definecolor{currentfill}{rgb}{0.282353,0.470588,0.815686}%
\pgfsetfillcolor{currentfill}%
\pgfsetlinewidth{0.752812pt}%
\definecolor{currentstroke}{rgb}{1.000000,1.000000,1.000000}%
\pgfsetstrokecolor{currentstroke}%
\pgfsetdash{}{0pt}%
\pgfpathmoveto{\pgfqpoint{1.361770in}{1.889537in}}%
\pgfpathcurveto{\pgfqpoint{1.370610in}{1.889537in}}{\pgfqpoint{1.379089in}{1.893050in}}{\pgfqpoint{1.385340in}{1.899300in}}%
\pgfpathcurveto{\pgfqpoint{1.391591in}{1.905551in}}{\pgfqpoint{1.395103in}{1.914031in}}{\pgfqpoint{1.395103in}{1.922871in}}%
\pgfpathcurveto{\pgfqpoint{1.395103in}{1.931711in}}{\pgfqpoint{1.391591in}{1.940190in}}{\pgfqpoint{1.385340in}{1.946441in}}%
\pgfpathcurveto{\pgfqpoint{1.379089in}{1.952692in}}{\pgfqpoint{1.370610in}{1.956204in}}{\pgfqpoint{1.361770in}{1.956204in}}%
\pgfpathcurveto{\pgfqpoint{1.352930in}{1.956204in}}{\pgfqpoint{1.344451in}{1.952692in}}{\pgfqpoint{1.338200in}{1.946441in}}%
\pgfpathcurveto{\pgfqpoint{1.331949in}{1.940190in}}{\pgfqpoint{1.328437in}{1.931711in}}{\pgfqpoint{1.328437in}{1.922871in}}%
\pgfpathcurveto{\pgfqpoint{1.328437in}{1.914031in}}{\pgfqpoint{1.331949in}{1.905551in}}{\pgfqpoint{1.338200in}{1.899300in}}%
\pgfpathcurveto{\pgfqpoint{1.344451in}{1.893050in}}{\pgfqpoint{1.352930in}{1.889537in}}{\pgfqpoint{1.361770in}{1.889537in}}%
\pgfpathclose%
\pgfusepath{stroke,fill}%
\end{pgfscope}%
\begin{pgfscope}%
\pgfpathrectangle{\pgfqpoint{0.636577in}{0.562153in}}{\pgfqpoint{4.171423in}{2.103873in}}%
\pgfusepath{clip}%
\pgfsetbuttcap%
\pgfsetroundjoin%
\definecolor{currentfill}{rgb}{0.282353,0.470588,0.815686}%
\pgfsetfillcolor{currentfill}%
\pgfsetlinewidth{0.752812pt}%
\definecolor{currentstroke}{rgb}{1.000000,1.000000,1.000000}%
\pgfsetstrokecolor{currentstroke}%
\pgfsetdash{}{0pt}%
\pgfpathmoveto{\pgfqpoint{3.245565in}{0.956514in}}%
\pgfpathcurveto{\pgfqpoint{3.254405in}{0.956514in}}{\pgfqpoint{3.262884in}{0.960026in}}{\pgfqpoint{3.269135in}{0.966277in}}%
\pgfpathcurveto{\pgfqpoint{3.275386in}{0.972528in}}{\pgfqpoint{3.278898in}{0.981007in}}{\pgfqpoint{3.278898in}{0.989847in}}%
\pgfpathcurveto{\pgfqpoint{3.278898in}{0.998687in}}{\pgfqpoint{3.275386in}{1.007166in}}{\pgfqpoint{3.269135in}{1.013417in}}%
\pgfpathcurveto{\pgfqpoint{3.262884in}{1.019668in}}{\pgfqpoint{3.254405in}{1.023180in}}{\pgfqpoint{3.245565in}{1.023180in}}%
\pgfpathcurveto{\pgfqpoint{3.236725in}{1.023180in}}{\pgfqpoint{3.228245in}{1.019668in}}{\pgfqpoint{3.221995in}{1.013417in}}%
\pgfpathcurveto{\pgfqpoint{3.215744in}{1.007166in}}{\pgfqpoint{3.212231in}{0.998687in}}{\pgfqpoint{3.212231in}{0.989847in}}%
\pgfpathcurveto{\pgfqpoint{3.212231in}{0.981007in}}{\pgfqpoint{3.215744in}{0.972528in}}{\pgfqpoint{3.221995in}{0.966277in}}%
\pgfpathcurveto{\pgfqpoint{3.228245in}{0.960026in}}{\pgfqpoint{3.236725in}{0.956514in}}{\pgfqpoint{3.245565in}{0.956514in}}%
\pgfpathclose%
\pgfusepath{stroke,fill}%
\end{pgfscope}%
\begin{pgfscope}%
\pgfpathrectangle{\pgfqpoint{0.636577in}{0.562153in}}{\pgfqpoint{4.171423in}{2.103873in}}%
\pgfusepath{clip}%
\pgfsetbuttcap%
\pgfsetroundjoin%
\definecolor{currentfill}{rgb}{1.000000,1.000000,1.000000}%
\pgfsetfillcolor{currentfill}%
\pgfsetlinewidth{1.204500pt}%
\definecolor{currentstroke}{rgb}{1.000000,1.000000,1.000000}%
\pgfsetstrokecolor{currentstroke}%
\pgfsetdash{}{0pt}%
\pgfsys@defobject{currentmarker}{\pgfqpoint{infin}{infin}}{\pgfqpoint{-infin}{-infin}}{%
\pgfusepath{stroke,fill}%
}%
\end{pgfscope}%
\begin{pgfscope}%
\pgfpathrectangle{\pgfqpoint{0.636577in}{0.562153in}}{\pgfqpoint{4.171423in}{2.103873in}}%
\pgfusepath{clip}%
\pgfsetbuttcap%
\pgfsetroundjoin%
\definecolor{currentfill}{rgb}{0.282353,0.470588,0.815686}%
\pgfsetfillcolor{currentfill}%
\pgfsetlinewidth{0.803000pt}%
\definecolor{currentstroke}{rgb}{0.282353,0.470588,0.815686}%
\pgfsetstrokecolor{currentstroke}%
\pgfsetdash{}{0pt}%
\pgfpathmoveto{\pgfqpoint{0.000000in}{-0.033333in}}%
\pgfpathcurveto{\pgfqpoint{0.008840in}{-0.033333in}}{\pgfqpoint{0.017319in}{-0.029821in}}{\pgfqpoint{0.023570in}{-0.023570in}}%
\pgfpathcurveto{\pgfqpoint{0.029821in}{-0.017319in}}{\pgfqpoint{0.033333in}{-0.008840in}}{\pgfqpoint{0.033333in}{0.000000in}}%
\pgfpathcurveto{\pgfqpoint{0.033333in}{0.008840in}}{\pgfqpoint{0.029821in}{0.017319in}}{\pgfqpoint{0.023570in}{0.023570in}}%
\pgfpathcurveto{\pgfqpoint{0.017319in}{0.029821in}}{\pgfqpoint{0.008840in}{0.033333in}}{\pgfqpoint{0.000000in}{0.033333in}}%
\pgfpathcurveto{\pgfqpoint{-0.008840in}{0.033333in}}{\pgfqpoint{-0.017319in}{0.029821in}}{\pgfqpoint{-0.023570in}{0.023570in}}%
\pgfpathcurveto{\pgfqpoint{-0.029821in}{0.017319in}}{\pgfqpoint{-0.033333in}{0.008840in}}{\pgfqpoint{-0.033333in}{0.000000in}}%
\pgfpathcurveto{\pgfqpoint{-0.033333in}{-0.008840in}}{\pgfqpoint{-0.029821in}{-0.017319in}}{\pgfqpoint{-0.023570in}{-0.023570in}}%
\pgfpathcurveto{\pgfqpoint{-0.017319in}{-0.029821in}}{\pgfqpoint{-0.008840in}{-0.033333in}}{\pgfqpoint{0.000000in}{-0.033333in}}%
\pgfpathclose%
\pgfusepath{stroke,fill}%
\end{pgfscope}%
\begin{pgfscope}%
\pgfpathrectangle{\pgfqpoint{0.636577in}{0.562153in}}{\pgfqpoint{4.171423in}{2.103873in}}%
\pgfusepath{clip}%
\pgfsetbuttcap%
\pgfsetroundjoin%
\definecolor{currentfill}{rgb}{0.933333,0.521569,0.290196}%
\pgfsetfillcolor{currentfill}%
\pgfsetlinewidth{0.803000pt}%
\definecolor{currentstroke}{rgb}{0.933333,0.521569,0.290196}%
\pgfsetstrokecolor{currentstroke}%
\pgfsetdash{}{0pt}%
\pgfpathmoveto{\pgfqpoint{0.000000in}{-0.033333in}}%
\pgfpathcurveto{\pgfqpoint{0.008840in}{-0.033333in}}{\pgfqpoint{0.017319in}{-0.029821in}}{\pgfqpoint{0.023570in}{-0.023570in}}%
\pgfpathcurveto{\pgfqpoint{0.029821in}{-0.017319in}}{\pgfqpoint{0.033333in}{-0.008840in}}{\pgfqpoint{0.033333in}{0.000000in}}%
\pgfpathcurveto{\pgfqpoint{0.033333in}{0.008840in}}{\pgfqpoint{0.029821in}{0.017319in}}{\pgfqpoint{0.023570in}{0.023570in}}%
\pgfpathcurveto{\pgfqpoint{0.017319in}{0.029821in}}{\pgfqpoint{0.008840in}{0.033333in}}{\pgfqpoint{0.000000in}{0.033333in}}%
\pgfpathcurveto{\pgfqpoint{-0.008840in}{0.033333in}}{\pgfqpoint{-0.017319in}{0.029821in}}{\pgfqpoint{-0.023570in}{0.023570in}}%
\pgfpathcurveto{\pgfqpoint{-0.029821in}{0.017319in}}{\pgfqpoint{-0.033333in}{0.008840in}}{\pgfqpoint{-0.033333in}{0.000000in}}%
\pgfpathcurveto{\pgfqpoint{-0.033333in}{-0.008840in}}{\pgfqpoint{-0.029821in}{-0.017319in}}{\pgfqpoint{-0.023570in}{-0.023570in}}%
\pgfpathcurveto{\pgfqpoint{-0.017319in}{-0.029821in}}{\pgfqpoint{-0.008840in}{-0.033333in}}{\pgfqpoint{0.000000in}{-0.033333in}}%
\pgfpathclose%
\pgfusepath{stroke,fill}%
\end{pgfscope}%
\begin{pgfscope}%
\pgfpathrectangle{\pgfqpoint{0.636577in}{0.562153in}}{\pgfqpoint{4.171423in}{2.103873in}}%
\pgfusepath{clip}%
\pgfsetbuttcap%
\pgfsetroundjoin%
\definecolor{currentfill}{rgb}{0.415686,0.800000,0.392157}%
\pgfsetfillcolor{currentfill}%
\pgfsetlinewidth{0.803000pt}%
\definecolor{currentstroke}{rgb}{0.415686,0.800000,0.392157}%
\pgfsetstrokecolor{currentstroke}%
\pgfsetdash{}{0pt}%
\pgfpathmoveto{\pgfqpoint{0.000000in}{-0.033333in}}%
\pgfpathcurveto{\pgfqpoint{0.008840in}{-0.033333in}}{\pgfqpoint{0.017319in}{-0.029821in}}{\pgfqpoint{0.023570in}{-0.023570in}}%
\pgfpathcurveto{\pgfqpoint{0.029821in}{-0.017319in}}{\pgfqpoint{0.033333in}{-0.008840in}}{\pgfqpoint{0.033333in}{0.000000in}}%
\pgfpathcurveto{\pgfqpoint{0.033333in}{0.008840in}}{\pgfqpoint{0.029821in}{0.017319in}}{\pgfqpoint{0.023570in}{0.023570in}}%
\pgfpathcurveto{\pgfqpoint{0.017319in}{0.029821in}}{\pgfqpoint{0.008840in}{0.033333in}}{\pgfqpoint{0.000000in}{0.033333in}}%
\pgfpathcurveto{\pgfqpoint{-0.008840in}{0.033333in}}{\pgfqpoint{-0.017319in}{0.029821in}}{\pgfqpoint{-0.023570in}{0.023570in}}%
\pgfpathcurveto{\pgfqpoint{-0.029821in}{0.017319in}}{\pgfqpoint{-0.033333in}{0.008840in}}{\pgfqpoint{-0.033333in}{0.000000in}}%
\pgfpathcurveto{\pgfqpoint{-0.033333in}{-0.008840in}}{\pgfqpoint{-0.029821in}{-0.017319in}}{\pgfqpoint{-0.023570in}{-0.023570in}}%
\pgfpathcurveto{\pgfqpoint{-0.017319in}{-0.029821in}}{\pgfqpoint{-0.008840in}{-0.033333in}}{\pgfqpoint{0.000000in}{-0.033333in}}%
\pgfpathclose%
\pgfusepath{stroke,fill}%
\end{pgfscope}%
\begin{pgfscope}%
\pgfpathrectangle{\pgfqpoint{0.636577in}{0.562153in}}{\pgfqpoint{4.171423in}{2.103873in}}%
\pgfusepath{clip}%
\pgfsetbuttcap%
\pgfsetroundjoin%
\definecolor{currentfill}{rgb}{0.839216,0.372549,0.372549}%
\pgfsetfillcolor{currentfill}%
\pgfsetlinewidth{0.803000pt}%
\definecolor{currentstroke}{rgb}{0.839216,0.372549,0.372549}%
\pgfsetstrokecolor{currentstroke}%
\pgfsetdash{}{0pt}%
\pgfpathmoveto{\pgfqpoint{0.000000in}{-0.033333in}}%
\pgfpathcurveto{\pgfqpoint{0.008840in}{-0.033333in}}{\pgfqpoint{0.017319in}{-0.029821in}}{\pgfqpoint{0.023570in}{-0.023570in}}%
\pgfpathcurveto{\pgfqpoint{0.029821in}{-0.017319in}}{\pgfqpoint{0.033333in}{-0.008840in}}{\pgfqpoint{0.033333in}{0.000000in}}%
\pgfpathcurveto{\pgfqpoint{0.033333in}{0.008840in}}{\pgfqpoint{0.029821in}{0.017319in}}{\pgfqpoint{0.023570in}{0.023570in}}%
\pgfpathcurveto{\pgfqpoint{0.017319in}{0.029821in}}{\pgfqpoint{0.008840in}{0.033333in}}{\pgfqpoint{0.000000in}{0.033333in}}%
\pgfpathcurveto{\pgfqpoint{-0.008840in}{0.033333in}}{\pgfqpoint{-0.017319in}{0.029821in}}{\pgfqpoint{-0.023570in}{0.023570in}}%
\pgfpathcurveto{\pgfqpoint{-0.029821in}{0.017319in}}{\pgfqpoint{-0.033333in}{0.008840in}}{\pgfqpoint{-0.033333in}{0.000000in}}%
\pgfpathcurveto{\pgfqpoint{-0.033333in}{-0.008840in}}{\pgfqpoint{-0.029821in}{-0.017319in}}{\pgfqpoint{-0.023570in}{-0.023570in}}%
\pgfpathcurveto{\pgfqpoint{-0.017319in}{-0.029821in}}{\pgfqpoint{-0.008840in}{-0.033333in}}{\pgfqpoint{0.000000in}{-0.033333in}}%
\pgfpathclose%
\pgfusepath{stroke,fill}%
\end{pgfscope}%
\begin{pgfscope}%
\pgfpathrectangle{\pgfqpoint{0.636577in}{0.562153in}}{\pgfqpoint{4.171423in}{2.103873in}}%
\pgfusepath{clip}%
\pgfsetroundcap%
\pgfsetroundjoin%
\pgfsetlinewidth{1.204500pt}%
\definecolor{currentstroke}{rgb}{0.100000,0.100000,0.100000}%
\pgfsetstrokecolor{currentstroke}%
\pgfsetstrokeopacity{0.600000}%
\pgfsetdash{}{0pt}%
\pgfpathmoveto{\pgfqpoint{0.826187in}{2.570395in}}%
\pgfpathlineto{\pgfqpoint{0.883356in}{2.497454in}}%
\pgfpathlineto{\pgfqpoint{0.940525in}{2.428168in}}%
\pgfpathlineto{\pgfqpoint{0.997694in}{2.362243in}}%
\pgfpathlineto{\pgfqpoint{1.054863in}{2.299416in}}%
\pgfpathlineto{\pgfqpoint{1.112031in}{2.239452in}}%
\pgfpathlineto{\pgfqpoint{1.169200in}{2.182141in}}%
\pgfpathlineto{\pgfqpoint{1.226369in}{2.127293in}}%
\pgfpathlineto{\pgfqpoint{1.302594in}{2.057700in}}%
\pgfpathlineto{\pgfqpoint{1.378820in}{1.991823in}}%
\pgfpathlineto{\pgfqpoint{1.455045in}{1.929341in}}%
\pgfpathlineto{\pgfqpoint{1.531270in}{1.869975in}}%
\pgfpathlineto{\pgfqpoint{1.607495in}{1.813474in}}%
\pgfpathlineto{\pgfqpoint{1.683720in}{1.759616in}}%
\pgfpathlineto{\pgfqpoint{1.759945in}{1.708201in}}%
\pgfpathlineto{\pgfqpoint{1.836171in}{1.659051in}}%
\pgfpathlineto{\pgfqpoint{1.931452in}{1.600554in}}%
\pgfpathlineto{\pgfqpoint{2.026734in}{1.545065in}}%
\pgfpathlineto{\pgfqpoint{2.122015in}{1.492335in}}%
\pgfpathlineto{\pgfqpoint{2.217297in}{1.442143in}}%
\pgfpathlineto{\pgfqpoint{2.312578in}{1.394294in}}%
\pgfpathlineto{\pgfqpoint{2.407860in}{1.348610in}}%
\pgfpathlineto{\pgfqpoint{2.522197in}{1.296428in}}%
\pgfpathlineto{\pgfqpoint{2.636535in}{1.246894in}}%
\pgfpathlineto{\pgfqpoint{2.750873in}{1.199793in}}%
\pgfpathlineto{\pgfqpoint{2.865211in}{1.154934in}}%
\pgfpathlineto{\pgfqpoint{2.998605in}{1.105202in}}%
\pgfpathlineto{\pgfqpoint{3.131999in}{1.058046in}}%
\pgfpathlineto{\pgfqpoint{3.265393in}{1.013252in}}%
\pgfpathlineto{\pgfqpoint{3.398787in}{0.970630in}}%
\pgfpathlineto{\pgfqpoint{3.551237in}{0.924363in}}%
\pgfpathlineto{\pgfqpoint{3.703688in}{0.880488in}}%
\pgfpathlineto{\pgfqpoint{3.856138in}{0.838808in}}%
\pgfpathlineto{\pgfqpoint{4.027645in}{0.794322in}}%
\pgfpathlineto{\pgfqpoint{4.199151in}{0.752170in}}%
\pgfpathlineto{\pgfqpoint{4.370658in}{0.712156in}}%
\pgfpathlineto{\pgfqpoint{4.561221in}{0.669992in}}%
\pgfpathlineto{\pgfqpoint{4.618390in}{0.657784in}}%
\pgfpathlineto{\pgfqpoint{4.618390in}{0.657784in}}%
\pgfusepath{stroke}%
\end{pgfscope}%
\begin{pgfscope}%
\pgfsetrectcap%
\pgfsetmiterjoin%
\pgfsetlinewidth{1.003750pt}%
\definecolor{currentstroke}{rgb}{1.000000,1.000000,1.000000}%
\pgfsetstrokecolor{currentstroke}%
\pgfsetdash{}{0pt}%
\pgfpathmoveto{\pgfqpoint{0.636577in}{0.562153in}}%
\pgfpathlineto{\pgfqpoint{0.636577in}{2.666026in}}%
\pgfusepath{stroke}%
\end{pgfscope}%
\begin{pgfscope}%
\pgfsetrectcap%
\pgfsetmiterjoin%
\pgfsetlinewidth{1.003750pt}%
\definecolor{currentstroke}{rgb}{1.000000,1.000000,1.000000}%
\pgfsetstrokecolor{currentstroke}%
\pgfsetdash{}{0pt}%
\pgfpathmoveto{\pgfqpoint{4.808000in}{0.562153in}}%
\pgfpathlineto{\pgfqpoint{4.808000in}{2.666026in}}%
\pgfusepath{stroke}%
\end{pgfscope}%
\begin{pgfscope}%
\pgfsetrectcap%
\pgfsetmiterjoin%
\pgfsetlinewidth{1.003750pt}%
\definecolor{currentstroke}{rgb}{1.000000,1.000000,1.000000}%
\pgfsetstrokecolor{currentstroke}%
\pgfsetdash{}{0pt}%
\pgfpathmoveto{\pgfqpoint{0.636577in}{0.562153in}}%
\pgfpathlineto{\pgfqpoint{4.808000in}{0.562153in}}%
\pgfusepath{stroke}%
\end{pgfscope}%
\begin{pgfscope}%
\pgfsetrectcap%
\pgfsetmiterjoin%
\pgfsetlinewidth{1.003750pt}%
\definecolor{currentstroke}{rgb}{1.000000,1.000000,1.000000}%
\pgfsetstrokecolor{currentstroke}%
\pgfsetdash{}{0pt}%
\pgfpathmoveto{\pgfqpoint{0.636577in}{2.666026in}}%
\pgfpathlineto{\pgfqpoint{4.808000in}{2.666026in}}%
\pgfusepath{stroke}%
\end{pgfscope}%
\begin{pgfscope}%
\definecolor{textcolor}{rgb}{0.150000,0.150000,0.150000}%
\pgfsetstrokecolor{textcolor}%
\pgfsetfillcolor{textcolor}%
\pgftext[x=2.722288in,y=2.749359in,,base]{\color{textcolor}\rmfamily\fontsize{9.600000}{11.520000}\selectfont Relação para o espectrômetro}%
\end{pgfscope}%
\begin{pgfscope}%
\pgfsetbuttcap%
\pgfsetmiterjoin%
\definecolor{currentfill}{rgb}{0.917647,0.917647,0.949020}%
\pgfsetfillcolor{currentfill}%
\pgfsetfillopacity{0.800000}%
\pgfsetlinewidth{0.803000pt}%
\definecolor{currentstroke}{rgb}{0.800000,0.800000,0.800000}%
\pgfsetstrokecolor{currentstroke}%
\pgfsetstrokeopacity{0.800000}%
\pgfsetdash{}{0pt}%
\pgfpathmoveto{\pgfqpoint{3.659440in}{1.607621in}}%
\pgfpathlineto{\pgfqpoint{4.710778in}{1.607621in}}%
\pgfpathquadraticcurveto{\pgfqpoint{4.738556in}{1.607621in}}{\pgfqpoint{4.738556in}{1.635398in}}%
\pgfpathlineto{\pgfqpoint{4.738556in}{2.568804in}}%
\pgfpathquadraticcurveto{\pgfqpoint{4.738556in}{2.596582in}}{\pgfqpoint{4.710778in}{2.596582in}}%
\pgfpathlineto{\pgfqpoint{3.659440in}{2.596582in}}%
\pgfpathquadraticcurveto{\pgfqpoint{3.631663in}{2.596582in}}{\pgfqpoint{3.631663in}{2.568804in}}%
\pgfpathlineto{\pgfqpoint{3.631663in}{1.635398in}}%
\pgfpathquadraticcurveto{\pgfqpoint{3.631663in}{1.607621in}}{\pgfqpoint{3.659440in}{1.607621in}}%
\pgfpathclose%
\pgfusepath{stroke,fill}%
\end{pgfscope}%
\begin{pgfscope}%
\definecolor{textcolor}{rgb}{0.150000,0.150000,0.150000}%
\pgfsetstrokecolor{textcolor}%
\pgfsetfillcolor{textcolor}%
\pgftext[x=3.838820in,y=2.434385in,left,base]{\color{textcolor}\rmfamily\fontsize{9.600000}{11.520000}\selectfont Lâmpadas}%
\end{pgfscope}%
\begin{pgfscope}%
\pgfsetbuttcap%
\pgfsetroundjoin%
\definecolor{currentfill}{rgb}{1.000000,1.000000,1.000000}%
\pgfsetfillcolor{currentfill}%
\pgfsetlinewidth{1.204500pt}%
\definecolor{currentstroke}{rgb}{1.000000,1.000000,1.000000}%
\pgfsetstrokecolor{currentstroke}%
\pgfsetdash{}{0pt}%
\pgfsys@defobject{currentmarker}{\pgfqpoint{infin}{infin}}{\pgfqpoint{-infin}{-infin}}{%
\pgfusepath{stroke,fill}%
}%
\begin{pgfscope}%
\pgfsys@transformshift{3.791385in}{2.323697in}%
\pgfsys@useobject{currentmarker}{}%
\end{pgfscope}%
\end{pgfscope}%
\begin{pgfscope}%
\definecolor{textcolor}{rgb}{0.150000,0.150000,0.150000}%
\pgfsetstrokecolor{textcolor}%
\pgfsetfillcolor{textcolor}%
\pgftext[x=4.006663in,y=2.287239in,left,base]{\color{textcolor}\rmfamily\fontsize{10.000000}{12.000000}\selectfont composto}%
\end{pgfscope}%
\begin{pgfscope}%
\pgfsetbuttcap%
\pgfsetroundjoin%
\definecolor{currentfill}{rgb}{0.282353,0.470588,0.815686}%
\pgfsetfillcolor{currentfill}%
\pgfsetlinewidth{0.803000pt}%
\definecolor{currentstroke}{rgb}{0.282353,0.470588,0.815686}%
\pgfsetstrokecolor{currentstroke}%
\pgfsetdash{}{0pt}%
\pgfpathmoveto{\pgfqpoint{3.791385in}{2.142062in}}%
\pgfpathcurveto{\pgfqpoint{3.800225in}{2.142062in}}{\pgfqpoint{3.808704in}{2.145575in}}{\pgfqpoint{3.814955in}{2.151826in}}%
\pgfpathcurveto{\pgfqpoint{3.821206in}{2.158077in}}{\pgfqpoint{3.824718in}{2.166556in}}{\pgfqpoint{3.824718in}{2.175396in}}%
\pgfpathcurveto{\pgfqpoint{3.824718in}{2.184236in}}{\pgfqpoint{3.821206in}{2.192715in}}{\pgfqpoint{3.814955in}{2.198966in}}%
\pgfpathcurveto{\pgfqpoint{3.808704in}{2.205217in}}{\pgfqpoint{3.800225in}{2.208729in}}{\pgfqpoint{3.791385in}{2.208729in}}%
\pgfpathcurveto{\pgfqpoint{3.782545in}{2.208729in}}{\pgfqpoint{3.774066in}{2.205217in}}{\pgfqpoint{3.767815in}{2.198966in}}%
\pgfpathcurveto{\pgfqpoint{3.761564in}{2.192715in}}{\pgfqpoint{3.758052in}{2.184236in}}{\pgfqpoint{3.758052in}{2.175396in}}%
\pgfpathcurveto{\pgfqpoint{3.758052in}{2.166556in}}{\pgfqpoint{3.761564in}{2.158077in}}{\pgfqpoint{3.767815in}{2.151826in}}%
\pgfpathcurveto{\pgfqpoint{3.774066in}{2.145575in}}{\pgfqpoint{3.782545in}{2.142062in}}{\pgfqpoint{3.791385in}{2.142062in}}%
\pgfpathclose%
\pgfusepath{stroke,fill}%
\end{pgfscope}%
\begin{pgfscope}%
\definecolor{textcolor}{rgb}{0.150000,0.150000,0.150000}%
\pgfsetstrokecolor{textcolor}%
\pgfsetfillcolor{textcolor}%
\pgftext[x=4.006663in,y=2.138937in,left,base]{\color{textcolor}\rmfamily\fontsize{10.000000}{12.000000}\selectfont Cd}%
\end{pgfscope}%
\begin{pgfscope}%
\pgfsetbuttcap%
\pgfsetroundjoin%
\definecolor{currentfill}{rgb}{0.933333,0.521569,0.290196}%
\pgfsetfillcolor{currentfill}%
\pgfsetlinewidth{0.803000pt}%
\definecolor{currentstroke}{rgb}{0.933333,0.521569,0.290196}%
\pgfsetstrokecolor{currentstroke}%
\pgfsetdash{}{0pt}%
\pgfpathmoveto{\pgfqpoint{3.791385in}{1.993761in}}%
\pgfpathcurveto{\pgfqpoint{3.800225in}{1.993761in}}{\pgfqpoint{3.808704in}{1.997273in}}{\pgfqpoint{3.814955in}{2.003524in}}%
\pgfpathcurveto{\pgfqpoint{3.821206in}{2.009775in}}{\pgfqpoint{3.824718in}{2.018254in}}{\pgfqpoint{3.824718in}{2.027094in}}%
\pgfpathcurveto{\pgfqpoint{3.824718in}{2.035934in}}{\pgfqpoint{3.821206in}{2.044413in}}{\pgfqpoint{3.814955in}{2.050664in}}%
\pgfpathcurveto{\pgfqpoint{3.808704in}{2.056915in}}{\pgfqpoint{3.800225in}{2.060427in}}{\pgfqpoint{3.791385in}{2.060427in}}%
\pgfpathcurveto{\pgfqpoint{3.782545in}{2.060427in}}{\pgfqpoint{3.774066in}{2.056915in}}{\pgfqpoint{3.767815in}{2.050664in}}%
\pgfpathcurveto{\pgfqpoint{3.761564in}{2.044413in}}{\pgfqpoint{3.758052in}{2.035934in}}{\pgfqpoint{3.758052in}{2.027094in}}%
\pgfpathcurveto{\pgfqpoint{3.758052in}{2.018254in}}{\pgfqpoint{3.761564in}{2.009775in}}{\pgfqpoint{3.767815in}{2.003524in}}%
\pgfpathcurveto{\pgfqpoint{3.774066in}{1.997273in}}{\pgfqpoint{3.782545in}{1.993761in}}{\pgfqpoint{3.791385in}{1.993761in}}%
\pgfpathclose%
\pgfusepath{stroke,fill}%
\end{pgfscope}%
\begin{pgfscope}%
\definecolor{textcolor}{rgb}{0.150000,0.150000,0.150000}%
\pgfsetstrokecolor{textcolor}%
\pgfsetfillcolor{textcolor}%
\pgftext[x=4.006663in,y=1.990636in,left,base]{\color{textcolor}\rmfamily\fontsize{10.000000}{12.000000}\selectfont Na}%
\end{pgfscope}%
\begin{pgfscope}%
\pgfsetbuttcap%
\pgfsetroundjoin%
\definecolor{currentfill}{rgb}{0.415686,0.800000,0.392157}%
\pgfsetfillcolor{currentfill}%
\pgfsetlinewidth{0.803000pt}%
\definecolor{currentstroke}{rgb}{0.415686,0.800000,0.392157}%
\pgfsetstrokecolor{currentstroke}%
\pgfsetdash{}{0pt}%
\pgfpathmoveto{\pgfqpoint{3.791385in}{1.845459in}}%
\pgfpathcurveto{\pgfqpoint{3.800225in}{1.845459in}}{\pgfqpoint{3.808704in}{1.848971in}}{\pgfqpoint{3.814955in}{1.855222in}}%
\pgfpathcurveto{\pgfqpoint{3.821206in}{1.861473in}}{\pgfqpoint{3.824718in}{1.869952in}}{\pgfqpoint{3.824718in}{1.878792in}}%
\pgfpathcurveto{\pgfqpoint{3.824718in}{1.887633in}}{\pgfqpoint{3.821206in}{1.896112in}}{\pgfqpoint{3.814955in}{1.902363in}}%
\pgfpathcurveto{\pgfqpoint{3.808704in}{1.908614in}}{\pgfqpoint{3.800225in}{1.912126in}}{\pgfqpoint{3.791385in}{1.912126in}}%
\pgfpathcurveto{\pgfqpoint{3.782545in}{1.912126in}}{\pgfqpoint{3.774066in}{1.908614in}}{\pgfqpoint{3.767815in}{1.902363in}}%
\pgfpathcurveto{\pgfqpoint{3.761564in}{1.896112in}}{\pgfqpoint{3.758052in}{1.887633in}}{\pgfqpoint{3.758052in}{1.878792in}}%
\pgfpathcurveto{\pgfqpoint{3.758052in}{1.869952in}}{\pgfqpoint{3.761564in}{1.861473in}}{\pgfqpoint{3.767815in}{1.855222in}}%
\pgfpathcurveto{\pgfqpoint{3.774066in}{1.848971in}}{\pgfqpoint{3.782545in}{1.845459in}}{\pgfqpoint{3.791385in}{1.845459in}}%
\pgfpathclose%
\pgfusepath{stroke,fill}%
\end{pgfscope}%
\begin{pgfscope}%
\definecolor{textcolor}{rgb}{0.150000,0.150000,0.150000}%
\pgfsetstrokecolor{textcolor}%
\pgfsetfillcolor{textcolor}%
\pgftext[x=4.006663in,y=1.842334in,left,base]{\color{textcolor}\rmfamily\fontsize{10.000000}{12.000000}\selectfont Hg}%
\end{pgfscope}%
\begin{pgfscope}%
\pgfsetbuttcap%
\pgfsetroundjoin%
\definecolor{currentfill}{rgb}{0.839216,0.372549,0.372549}%
\pgfsetfillcolor{currentfill}%
\pgfsetlinewidth{0.803000pt}%
\definecolor{currentstroke}{rgb}{0.839216,0.372549,0.372549}%
\pgfsetstrokecolor{currentstroke}%
\pgfsetdash{}{0pt}%
\pgfpathmoveto{\pgfqpoint{3.791385in}{1.695191in}}%
\pgfpathcurveto{\pgfqpoint{3.800225in}{1.695191in}}{\pgfqpoint{3.808704in}{1.698703in}}{\pgfqpoint{3.814955in}{1.704954in}}%
\pgfpathcurveto{\pgfqpoint{3.821206in}{1.711205in}}{\pgfqpoint{3.824718in}{1.719684in}}{\pgfqpoint{3.824718in}{1.728524in}}%
\pgfpathcurveto{\pgfqpoint{3.824718in}{1.737364in}}{\pgfqpoint{3.821206in}{1.745843in}}{\pgfqpoint{3.814955in}{1.752094in}}%
\pgfpathcurveto{\pgfqpoint{3.808704in}{1.758345in}}{\pgfqpoint{3.800225in}{1.761857in}}{\pgfqpoint{3.791385in}{1.761857in}}%
\pgfpathcurveto{\pgfqpoint{3.782545in}{1.761857in}}{\pgfqpoint{3.774066in}{1.758345in}}{\pgfqpoint{3.767815in}{1.752094in}}%
\pgfpathcurveto{\pgfqpoint{3.761564in}{1.745843in}}{\pgfqpoint{3.758052in}{1.737364in}}{\pgfqpoint{3.758052in}{1.728524in}}%
\pgfpathcurveto{\pgfqpoint{3.758052in}{1.719684in}}{\pgfqpoint{3.761564in}{1.711205in}}{\pgfqpoint{3.767815in}{1.704954in}}%
\pgfpathcurveto{\pgfqpoint{3.774066in}{1.698703in}}{\pgfqpoint{3.782545in}{1.695191in}}{\pgfqpoint{3.791385in}{1.695191in}}%
\pgfpathclose%
\pgfusepath{stroke,fill}%
\end{pgfscope}%
\begin{pgfscope}%
\definecolor{textcolor}{rgb}{0.150000,0.150000,0.150000}%
\pgfsetstrokecolor{textcolor}%
\pgfsetfillcolor{textcolor}%
\pgftext[x=4.006663in,y=1.692066in,left,base]{\color{textcolor}\rmfamily\fontsize{10.000000}{12.000000}\selectfont He}%
\end{pgfscope}%
\end{pgfpicture}%
\makeatother%
\endgroup%


	\caption{Cauchy}
	\label{fig:cauchy}
\end{figure}
