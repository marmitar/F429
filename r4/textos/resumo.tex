A decomposição da luz em ondas de diferentes frequências decorre da dependência do índice de refração de um material do comprimento de onda da onda incidente.

Neste experimento, foi medida a relação de dispersão de um prisma de vidro a partir dos ângulos de desvio mínimo decorrentes da decomposição da luz emitida por lâmpadas de Hélio (\ce{He}), Cádmio (\ce{Cd}), Mercúrio (\ce{Hg}) e Sódio (\ce{Na}). Partindo das medidas de ângulo de desvio mínimo, este experimento teve como objetivo anlisar a precisão deste deste aparato todo para medições desse tipo de natureza, utilizando como modelo a equação de Cauchy.

O resultado de tudo isso foi uma resolução espectral de \SI[detect-all = true]{32}{\nano\meter} na faixa de luz visível, capaz de diferenciar até 11 faixas distintas de comprimentos de onda nesse espectro. Isso já seria uma quantia razoável para esse equipamentos, porém eles têm potencial para muito mais se o modelo teórico for calibrado mais precisamente.
