Para a realização da coleta de dados, o experimento foi dividido em duas partes. sendo a primeira parte dedicada à analise dos filtros RC e segunda do filtro RLC.

\subsection{Filtros RC}

    Para a análise de filtros RC como circuitos integradores e diferenciadores, primeiramente montou-se um circuito RC com componentes $R_1=\SI{150}{\ohm}$ e $C_1=\SI{220}{\nano\farad}$, em valores nominais. Para a integração, foi utilizado o circuito \ref{fig:integrador}, um filtro passa-baixas. Analogamente, para a diferenciação, o circuito \ref{fig:diferenciador}, um filtro passa-altas, foi empregado. É importante citar que ambos circuitos foram construídos com os mesmos componentes, em disposições diferentes, questão melhor abordada na seção \ref{discussao}
    
    Em sequência, o gerador de onda \texttt{BK Precision 4052} \cite{ref:gerador} foi ajustado para tensão $V_{pp}=\SI{1}{\volt}$ e \textit{offset} de $\SI{0,5}{\volt}$ e os sinais de entrada e saída, $V_{in}$ e $V_{out}$, foram ligados aos osciloscópio \texttt{Tektronix TBS 1062} \cite{ref:osciloscopio}. Com o sistema em questão devidamente montado, valores de tensão foram coletados para frequências de $\SI{50}{\kilo\hertz}$ e $\SI{500}{\hertz}$, para um intervalo de tempo de $\SI{40}{\milli\second}$. Três situações foram observadas, ondas senoidais, quadradas e triangulares. Por fim, os dados de tensão $V_{in}$ e $V_{out}$ foram coletados utilizando a interface gráfica \texttt{Pylab} da biblioteca  \texttt{pylef}\cite{ref:pylef} de \textit{Python}.

\subsection{Filtro RLC}

    Para a análise do filtro RLC, foi montado um circuito como pode ser visto em \ref{fig:amortecido}, com componentes de valor nominal $L=\SI{2,42}{\milli\henry}$ e $C_{2}=\SI{47}{\nano\farad}$. Para este circuito, o resistor de década $R_{d}$, uma assumiu valores variados com o decorrer da coleta de dados.
    
    Na sequência, o circuito foi conectado com os mesmos gerador de onda e osciloscópio utilizados anteriormente, com o gerador ajustado no modo pulso, com frequência de $\SI{500}{\hertz}$ e tensão de saída $V_{in}=\SI{1}{\volt}$. A fim de obter os regimes oscilatórios mencionados na seção \ref{introducao}, os valores de $R_d$ foram variados até que se pudesse observar no osciloscópio, os regimes desejados. Também permitindo a observação da variação no transiente em decorrência da mudança de impedância. Para o caso sub-amortecido e sobreamortecido, $\SI{40}{\ohm}$ e $\SI{1}{\kilo\ohm}$, respectivamente, mostraram, de forma bem definida, os regimes em questão. 
    
    É importante mencionar que, de forma que a tensão de saída se estabilizasse ao redor da tensão de entrada, cumprindo os objetivos do experimento, $V_{out}$ se referiria à tensão no capacitor, já que após o transiente, o mesmo estaria carregado e produziria uma tensão próxima de $V_{in}$, segundo a teoria.

    Por fim, novamente, as tensões $V_{in}$ e $V_{out}$ foram coletados utilizando o \textit{software} \texttt{Pylab}.

    \begin{figure}[H]
    \centering

    \begin{subfigure}[t]{0.3\textwidth}
        \centering
        \begin{circuitikz}[scale=1.2]

  \draw (0, 2)	% linha superior
  node[above] {$V_{in}$}
  to [capacitor, o-, l=$C$] ++(2,0)
  to [short, -o] ++(1,0)
  to ++(0,0) node[above] {$V_{out}$};

  \draw (2,0)		% aterramento
  node[ground] {}
  to [resistor, -, l_=$R$] ++(0,2);

\end{circuitikz}

        \caption{Passa-altas}
        \label{fig:passaalta}
    \end{subfigure}
    \qquad
    \begin{subfigure}[t]{0.3\textwidth}
        \centering
        \begin{circuitikz}[scale=1.2]

  \draw (0, 2)    % linha superior
  node[above] {$V_{in}$}
  to [resistor, o-, l=$R$] ++(2,0)
  to [short, -o] ++(1,0)
  to ++(0,0) node[above] {$V_{out}$};

  \draw (2,0)	    % aterramento
  node[ground] {}
  to [inductor, l_=$L$] ++(0,1)
  to [capacitor, l_=$C$] ++(0,1);

\end{circuitikz}

        \caption{Rejeita-banda}
        \label{fig:rejeitabanda}
    \end{subfigure}

    \caption{Circuitaria.}
    \label{fig:circuitos}
\end{figure}


\pagebreak