Ao final do experimento, concluiu-se que filtros passivos de um componente reativo podem atuar de forma efetiva como integradores e diferenciadores de um sinal original. Percebeu-se que os sinais de saída foram bem similares ao esperado, mostrando que os circuitos testados eram fiéis aos modelos teóricos e que, na construção de sistemas que cumprissem os objetivos do experimento, os circuitos empregados foram bem sucedidos.

Em relação ao filtro ressonante, foi construído um circuito capaz de operar nos dois regimes de amortecimento desejado, além de, no regime estacionário, apresentar um sinal de saída que se estabilizasse muito próximo ao redor do sinal de entrada. Nos quesitos anteriores, o circuito utilizado se mostrou eficiente e cumpriu os objetivos propostos. 

Em suma, em ambas as partes do experimento, a aplicação de filtros passivos nas situações anteriores foi eficiente e seguiu as previsões teóricas, mostrando a conciliação da teoria e prática atingindo todos os objetivos propostos.