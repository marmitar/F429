Ao fim do experimento, concluiu-se que ambos filtros construídos pelo grupo forma extremamente eficazes para atingir os objetivos propostos. Também observou-se que os modelos teóricos empregados foram coerentes, dado que as frequências de corte e central calculadas corresponderam com exatidão com os dados coletados. Assim, mostrando coesão entre os modelos teóricos e o comportamento real do sistemas estudados.
