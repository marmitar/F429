Este experimento tem por objetivo o estudo dos circuitos RC, RL e RLC, bem como suas propriedades e funções. Circuitos RC e RL, estudados na primeira etapa deste experimento, podem ser utilizados como filtros de frequências, a serem classificados como passa-altas ou passa-baixas. Especificamente, busca-se construir um filtro passa-alta com intuito de filtrar um ruido de \SI{120}{\hertz} de um sinal de \SI{8}{\kilo\hertz}. Na segunda etapa deste experimento, serão estudados os circuitos RLC, os chamados filtros ressonantes, o que permitirá classificá-los como passa-bandas ou rejeita-bandas. Para tal estudo, sera analisado um filtro ressonante que permita a passagem de sinais de  \SI{100}{\hertz} e \SI{10}{\kilo\hertz}, filtrando um sinal de frequência média \SI{1}{\kilo\hertz}.

