%-----------------%
% Filtros Simples %
%-----------------%
\centering
%% CIRCUITO 1, subfigura 1
\subfigure[\label{fig:rc_pa}Passa-altas RC]{
	\begin{circuitikz}[scale=1]
		\node (Xi) at (0.7,0.7) {$V_1$};
		\node (Xf) at (3.7,0.7) {$V_2$};
		\draw [semithick,->] (Xi) -- (0.1,0.1);
		\draw [semithick,->] (Xf) -- (3.1,0.1);
		\draw to [resistor, o-, l_=$R$] ++(2,0)
			(2,0) to [short, -o] ++(1,0)
			(2,0) to [capacitor, -, l=$C$] ++(0,-2)
			node[ground] {};
	\end{circuitikz}
}
%% CIRCUITO 2, subfigura 1
\subfigure[\label{fig:rc_pb}Passa-baixas RC]{
	\begin{circuitikz}[scale=1]
		\node (Xi) at (0.7,0.7) {$V_1$};
		\node (Xf) at (3.7,0.7) {$V_2$};
		\draw [semithick,->] (Xi) -- (0.1,0.1);
		\draw [semithick,->] (Xf) -- (3.1,0.1);
		\draw to [capacitor, o-, l_=$C$] ++(2,0)
			(2,0) to [short, -o] ++(1,0)
			(2,0) to [resistor, -, l=$R$] ++(0,-2)
			node[ground] {};
	\end{circuitikz}
}
%% CIRCUITO 3, subfigura 3
\subfigure[\label{fig:rl_pa}Passa-altas RL]{
	\begin{circuitikz}[scale=1]
		\node (Xi) at (0.7,0.7) {$V_1$};
		\node (Xf) at (3.7,0.7) {$V_2$};
		\draw [semithick,->] (Xi) -- (0.1,0.1);
		\draw [semithick,->] (Xf) -- (3.1,0.1);
		\draw to [resistor, o-, l_=$R$] ++(2,0)
			(2,0) to [short, -o] ++(1,0)
			(2,0) to [inductor, -, l=$L$] ++(0,-2)
			node[ground] {};
	\end{circuitikz}
}
\caption{Filtros simplesinhos.}